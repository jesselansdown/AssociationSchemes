% generated by GAPDoc2LaTeX from XML source (Frank Luebeck)
\documentclass[a4paper,11pt]{report}

\usepackage[top=37mm,bottom=37mm,left=27mm,right=27mm]{geometry}
\sloppy
\pagestyle{myheadings}
\usepackage{amssymb}
\usepackage[utf8]{inputenc}
\usepackage{makeidx}
\makeindex
\usepackage{color}
\definecolor{FireBrick}{rgb}{0.5812,0.0074,0.0083}
\definecolor{RoyalBlue}{rgb}{0.0236,0.0894,0.6179}
\definecolor{RoyalGreen}{rgb}{0.0236,0.6179,0.0894}
\definecolor{RoyalRed}{rgb}{0.6179,0.0236,0.0894}
\definecolor{LightBlue}{rgb}{0.8544,0.9511,1.0000}
\definecolor{Black}{rgb}{0.0,0.0,0.0}

\definecolor{linkColor}{rgb}{0.0,0.0,0.554}
\definecolor{citeColor}{rgb}{0.0,0.0,0.554}
\definecolor{fileColor}{rgb}{0.0,0.0,0.554}
\definecolor{urlColor}{rgb}{0.0,0.0,0.554}
\definecolor{promptColor}{rgb}{0.0,0.0,0.589}
\definecolor{brkpromptColor}{rgb}{0.589,0.0,0.0}
\definecolor{gapinputColor}{rgb}{0.589,0.0,0.0}
\definecolor{gapoutputColor}{rgb}{0.0,0.0,0.0}

%%  for a long time these were red and blue by default,
%%  now black, but keep variables to overwrite
\definecolor{FuncColor}{rgb}{0.0,0.0,0.0}
%% strange name because of pdflatex bug:
\definecolor{Chapter }{rgb}{0.0,0.0,0.0}
\definecolor{DarkOlive}{rgb}{0.1047,0.2412,0.0064}


\usepackage{fancyvrb}

\usepackage{mathptmx,helvet}
\usepackage[T1]{fontenc}
\usepackage{textcomp}


\usepackage[
            pdftex=true,
            bookmarks=true,        
            a4paper=true,
            pdftitle={Written with GAPDoc},
            pdfcreator={LaTeX with hyperref package / GAPDoc},
            colorlinks=true,
            backref=page,
            breaklinks=true,
            linkcolor=linkColor,
            citecolor=citeColor,
            filecolor=fileColor,
            urlcolor=urlColor,
            pdfpagemode={UseNone}, 
           ]{hyperref}

\newcommand{\maintitlesize}{\fontsize{50}{55}\selectfont}

% write page numbers to a .pnr log file for online help
\newwrite\pagenrlog
\immediate\openout\pagenrlog =\jobname.pnr
\immediate\write\pagenrlog{PAGENRS := [}
\newcommand{\logpage}[1]{\protect\write\pagenrlog{#1, \thepage,}}
%% were never documented, give conflicts with some additional packages

\newcommand{\GAP}{\textsf{GAP}}

%% nicer description environments, allows long labels
\usepackage{enumitem}
\setdescription{style=nextline}

%% depth of toc
\setcounter{tocdepth}{1}





%% command for ColorPrompt style examples
\newcommand{\gapprompt}[1]{\color{promptColor}{\bfseries #1}}
\newcommand{\gapbrkprompt}[1]{\color{brkpromptColor}{\bfseries #1}}
\newcommand{\gapinput}[1]{\color{gapinputColor}{#1}}


\begin{document}

\logpage{[ 0, 0, 0 ]}
\begin{titlepage}
\mbox{}\vfill

\begin{center}{\maintitlesize \textbf{ AssociationSchemes \mbox{}}}\\
\vfill

\hypersetup{pdftitle= AssociationSchemes }
\markright{\scriptsize \mbox{}\hfill  AssociationSchemes  \hfill\mbox{}}
{\Huge \textbf{ A \textsf{GAP} package for working with association schemes and coherent configurations \mbox{}}}\\
\vfill

{\Huge  0.1 \mbox{}}\\[1cm]
{ 5 February 2019 \mbox{}}\\[1cm]
\mbox{}\\[2cm]
{\Large \textbf{ John Bamberg\\
    \mbox{}}}\\
{\Large \textbf{ Akihide Hanaki\\
    \mbox{}}}\\
{\Large \textbf{ Jesse Lansdown\\
    \mbox{}}}\\
\hypersetup{pdfauthor= John Bamberg\\
    ;  Akihide Hanaki\\
    ;  Jesse Lansdown\\
    }
\end{center}\vfill

\mbox{}\\
{\mbox{}\\
\small \noindent \textbf{ John Bamberg\\
    }  Email: \href{mailto://john.bamberg@uwa.edu.au} {\texttt{john.bamberg@uwa.edu.au}}\\
  Homepage: \href{http://school.maths.uwa.edu.au/~bamberg/} {\texttt{http://school.maths.uwa.edu.au/\texttt{\symbol{126}}bamberg/}}\\
  Address: \begin{minipage}[t]{8cm}\noindent
 John Bamberg\\
 School of Mathematics and Statistics\\
 The University of Western Australia\\
 35 Stirling Highway\\
 Crawley WA 6009, Perth\\
 Australia\\
 \end{minipage}
}\\
{\mbox{}\\
\small \noindent \textbf{ Akihide Hanaki\\
    }  Email: \href{mailto://hanaki@shinshu-u.ac.jp} {\texttt{hanaki@shinshu-u.ac.jp}}\\
  Homepage: \href{http://math.shinshu-u.ac.jp/~hanaki/} {\texttt{http://math.shinshu-u.ac.jp/\texttt{\symbol{126}}hanaki/}}\\
  Address: \begin{minipage}[t]{8cm}\noindent
 Akihide Hanaki\\
 Department of Mathematics\\
 Faculty of Science, Shinshu University\\
 Matsumoto 390-8621, Japan\\
 \end{minipage}
}\\
{\mbox{}\\
\small \noindent \textbf{ Jesse Lansdown\\
    }  Email: \href{mailto://jesse.lansdown@research.uwa.edu.au} {\texttt{jesse.lansdown@research.uwa.edu.au}}\\
  Homepage: \href{http://www.jesselansdown.com} {\texttt{http://www.jesselansdown.com}}\\
  Address: \begin{minipage}[t]{8cm}\noindent
 Jesse Lansdown\\
 School of Mathematics and Statistics\\
 The University of Western Australia\\
 35 Stirling Highway\\
 Crawley WA 6009, Perth\\
 Australia\\
 \end{minipage}
}\\
\end{titlepage}

\newpage\setcounter{page}{2}
\newpage

\def\contentsname{Contents\logpage{[ 0, 0, 1 ]}}

\tableofcontents
\newpage

     
\chapter{\textcolor{Chapter }{Introduction}}\label{Chapter_Introduction}
\logpage{[ 1, 0, 0 ]}
\hyperdef{L}{X7DFB63A97E67C0A1}{}
{
  

 AssociationSchemes is a GAP package for working with association schemes and
coherent configurations. Currently only methods for homogeneous coherent
configurations are implemented. 
\section{\textcolor{Chapter }{Installation}}\label{Chapter_Introduction_Section_Installation}
\logpage{[ 1, 1, 0 ]}
\hyperdef{L}{X8360C04082558A12}{}
{
  To install AssociationSchemes, simply copy to the "pkg" directory of your GAP
installation. }

 
\section{\textcolor{Chapter }{Prerequisite packages}}\label{Chapter_Introduction_Section_Prerequisite_packages}
\logpage{[ 1, 2, 0 ]}
\hyperdef{L}{X8490395B79917BA1}{}
{
  "Digraphs" is needed if automorphism groups are to be computed. }

 }

   
\chapter{\textcolor{Chapter }{Functionality}}\label{Chapter_Functionality}
\logpage{[ 2, 0, 0 ]}
\hyperdef{L}{X87F1120883F5B4D0}{}
{
  
\section{\textcolor{Chapter }{Constructor Methods}}\label{Chapter_Functionality_Section_Constructor_Methods}
\logpage{[ 2, 1, 0 ]}
\hyperdef{L}{X820CD05F85142F0A}{}
{
  

\subsection{\textcolor{Chapter }{CoherentConfiguration (for IsMatrix)}}
\logpage{[ 2, 1, 1 ]}\nobreak
\hyperdef{L}{X847E43E07EC6FD4A}{}
{\noindent\textcolor{FuncColor}{$\triangleright$\enspace\texttt{CoherentConfiguration({\mdseries\slshape M})\index{CoherentConfiguration@\texttt{CoherentConfiguration}!for IsMatrix}
\label{CoherentConfiguration:for IsMatrix}
}\hfill{\scriptsize (operation)}}\\
\textbf{\indent Returns:\ }
coherent configuration 



 Takes the relationship matrix, $M$, describing a coherent configuration and returns a CoherentConfiguration
object. The matrix $M = \sum_{i=0}^d i A_i$, where $A_i$ are the adjacency matrices describing a coherent configuration. Checks that
the matrix satisfies the coherent configuration axioms. }

 

\subsection{\textcolor{Chapter }{CoherentConfigurationNC (for IsMatrix)}}
\logpage{[ 2, 1, 2 ]}\nobreak
\hyperdef{L}{X85B5EA717BDC514A}{}
{\noindent\textcolor{FuncColor}{$\triangleright$\enspace\texttt{CoherentConfigurationNC({\mdseries\slshape M})\index{CoherentConfigurationNC@\texttt{CoherentConfigurationNC}!for IsMatrix}
\label{CoherentConfigurationNC:for IsMatrix}
}\hfill{\scriptsize (operation)}}\\
\textbf{\indent Returns:\ }
coherent configuration 



 Same as CoherentConfiguration but without performing any checks. Use this
method only if you know with certainty that $M$ describes a coherent configuration. }

 

\subsection{\textcolor{Chapter }{HomogeneousCoherentConfiguration (for IsPosInt, IsPosInt)}}
\logpage{[ 2, 1, 3 ]}\nobreak
\hyperdef{L}{X8610716786531117}{}
{\noindent\textcolor{FuncColor}{$\triangleright$\enspace\texttt{HomogeneousCoherentConfiguration({\mdseries\slshape n, k})\index{HomogeneousCoherentConfiguration@\texttt{HomogeneousCoherentConfiguration}!for IsPosInt, IsPosInt}
\label{HomogeneousCoherentConfiguration:for IsPosInt, IsPosInt}
}\hfill{\scriptsize (operation)}}\\
\textbf{\indent Returns:\ }
coherent configuration 



 Returns the $k$-th homogeneous coherent configuration of order $n$. Library is complete for $5 \le n \le 38$ excluding $n \in \{31, 35, 36, 37\}$. (Put reference). }

 

\subsection{\textcolor{Chapter }{FusionScheme (for IsCoherentConfiguration, IsList)}}
\logpage{[ 2, 1, 4 ]}\nobreak
\hyperdef{L}{X87EC62747CACA134}{}
{\noindent\textcolor{FuncColor}{$\triangleright$\enspace\texttt{FusionScheme({\mdseries\slshape CC, L})\index{FusionScheme@\texttt{FusionScheme}!for IsCoherentConfiguration, IsList}
\label{FusionScheme:for IsCoherentConfiguration, IsList}
}\hfill{\scriptsize (operation)}}\\
\textbf{\indent Returns:\ }
coherent configuration 



 Takes a $d$-class coherent configuration CC and returns a fusion scheme corresponding to
L, where L is a partion of $\{0, \ldots, d\}$. Returns fail if $L$ is not a valid partition. }

 

\subsection{\textcolor{Chapter }{CoherentConfigurationByOrbitals (for IsPermGroup)}}
\logpage{[ 2, 1, 5 ]}\nobreak
\hyperdef{L}{X7D6309BE79A40B00}{}
{\noindent\textcolor{FuncColor}{$\triangleright$\enspace\texttt{CoherentConfigurationByOrbitals({\mdseries\slshape G})\index{CoherentConfigurationByOrbitals@\texttt{CoherentConfigurationByOrbitals}!for IsPermGroup}
\label{CoherentConfigurationByOrbitals:for IsPermGroup}
}\hfill{\scriptsize (operation)}}\\
\textbf{\indent Returns:\ }
coherent configuration 



 Constructs a "group-case" coherent configuration, where the relations are
defined by the orbitals of $G$ on $\{1, \ldots, n\} \times \{1, \ldots, n\}$. $G$ must be a permutation group which is transitive on $\{1, \ldots, n\}$. }

 

\subsection{\textcolor{Chapter }{CoherentConfigurationByOrbitals (for IsGroup, IsGroup)}}
\logpage{[ 2, 1, 6 ]}\nobreak
\hyperdef{L}{X85D8E4AC79E68860}{}
{\noindent\textcolor{FuncColor}{$\triangleright$\enspace\texttt{CoherentConfigurationByOrbitals({\mdseries\slshape G, H})\index{CoherentConfigurationByOrbitals@\texttt{CoherentConfigurationByOrbitals}!for IsGroup, IsGroup}
\label{CoherentConfigurationByOrbitals:for IsGroup, IsGroup}
}\hfill{\scriptsize (operation)}}\\
\textbf{\indent Returns:\ }
coherent configuration 



 Constructs a "group-case" coherent configuration, where the relations are
defined by the orbitals of $G$ on $G/H$. $G$ is a group, $H$ is a subgroup of $G$, $G/H$ is the set of right cosets of $G$ on $H$, and $G$ must be transitive on $G/H$. }

 

\subsection{\textcolor{Chapter }{JohnsonScheme (for IsPosInt, IsPosInt)}}
\logpage{[ 2, 1, 7 ]}\nobreak
\hyperdef{L}{X87F73F177F55A0A6}{}
{\noindent\textcolor{FuncColor}{$\triangleright$\enspace\texttt{JohnsonScheme({\mdseries\slshape n, k})\index{JohnsonScheme@\texttt{JohnsonScheme}!for IsPosInt, IsPosInt}
\label{JohnsonScheme:for IsPosInt, IsPosInt}
}\hfill{\scriptsize (operation)}}\\
\textbf{\indent Returns:\ }
Johnson scheme 



 Returns the Johnson scheme, $J(n, k)$. }

 

\subsection{\textcolor{Chapter }{SchurianScheme (for IsPermGroup)}}
\logpage{[ 2, 1, 8 ]}\nobreak
\hyperdef{L}{X7873613B7C723A02}{}
{\noindent\textcolor{FuncColor}{$\triangleright$\enspace\texttt{SchurianScheme({\mdseries\slshape G})\index{SchurianScheme@\texttt{SchurianScheme}!for IsPermGroup}
\label{SchurianScheme:for IsPermGroup}
}\hfill{\scriptsize (operation)}}\\
\textbf{\indent Returns:\ }
Schurian scheme 



 Returns the Schurian scheme defined by $G$, where $G$ is a generously transitive permutation group. A Schurian scheme is a special
case of CoherentConfigurationByOrbitals and is symmetric. }

 }

 
\section{\textcolor{Chapter }{Matrices describing coherent configurations}}\label{Chapter_Functionality_Section_Matrices_describing_coherent_configurations}
\logpage{[ 2, 2, 0 ]}
\hyperdef{L}{X7EEC9A0F84879711}{}
{
  

\subsection{\textcolor{Chapter }{RelationMatrix (for IsCoherentConfiguration)}}
\logpage{[ 2, 2, 1 ]}\nobreak
\hyperdef{L}{X7D32B7ED7A84A314}{}
{\noindent\textcolor{FuncColor}{$\triangleright$\enspace\texttt{RelationMatrix({\mdseries\slshape CC})\index{RelationMatrix@\texttt{RelationMatrix}!for IsCoherentConfiguration}
\label{RelationMatrix:for IsCoherentConfiguration}
}\hfill{\scriptsize (operation)}}\\
\textbf{\indent Returns:\ }
relation matrix $M$ 



 Takes a coherent configuration and returns the underlying relation matrix $M = \sum_{i=0}^d i A_i$, where $A_i$ are the adjacency matrices of the coherent configuration }

 

\subsection{\textcolor{Chapter }{AdjacencyMatrices (for IsCoherentConfiguration)}}
\logpage{[ 2, 2, 2 ]}\nobreak
\hyperdef{L}{X85C23F3D7A3291EB}{}
{\noindent\textcolor{FuncColor}{$\triangleright$\enspace\texttt{AdjacencyMatrices({\mdseries\slshape CC})\index{AdjacencyMatrices@\texttt{AdjacencyMatrices}!for IsCoherentConfiguration}
\label{AdjacencyMatrices:for IsCoherentConfiguration}
}\hfill{\scriptsize (attribute)}}\\
\textbf{\indent Returns:\ }
L 



 Returns a list $L$, where the $i$-th entry of $L$ is the adjacency matrix $A_{i-1}$, where $(A_i)_{xy} =1$ if $(x,y) \in R_i$ and $(A_i)_{xy} =0$ otherwise. }

 

\subsection{\textcolor{Chapter }{IntersectionMatrices (for IsCoherentConfiguration)}}
\logpage{[ 2, 2, 3 ]}\nobreak
\hyperdef{L}{X82D252C185061232}{}
{\noindent\textcolor{FuncColor}{$\triangleright$\enspace\texttt{IntersectionMatrices({\mdseries\slshape CC})\index{IntersectionMatrices@\texttt{IntersectionMatrices}!for IsCoherentConfiguration}
\label{IntersectionMatrices:for IsCoherentConfiguration}
}\hfill{\scriptsize (attribute)}}\\
\textbf{\indent Returns:\ }
L 



 Returns a list L of the intersection matrices of a coherent configuration $CC$, where the $i$-th entry of $L$ is $B_{i-1}$ and $B_{i}_{jk} = p_{ji}^k$. }

 

\subsection{\textcolor{Chapter }{MatrixOfEigenvalues (for IsCoherentConfiguration)}}
\logpage{[ 2, 2, 4 ]}\nobreak
\hyperdef{L}{X8508511D8747A7D5}{}
{\noindent\textcolor{FuncColor}{$\triangleright$\enspace\texttt{MatrixOfEigenvalues({\mdseries\slshape CC})\index{MatrixOfEigenvalues@\texttt{MatrixOfEigenvalues}!for IsCoherentConfiguration}
\label{MatrixOfEigenvalues:for IsCoherentConfiguration}
}\hfill{\scriptsize (attribute)}}\\
\textbf{\indent Returns:\ }
P 



 Returns a the matrix of eigenvalues (or character table), $P$, for a coherent configuration CC. }

 

\subsection{\textcolor{Chapter }{DualMatrixOfEigenvalues (for IsCoherentConfiguration)}}
\logpage{[ 2, 2, 5 ]}\nobreak
\hyperdef{L}{X85484A2278D5276C}{}
{\noindent\textcolor{FuncColor}{$\triangleright$\enspace\texttt{DualMatrixOfEigenvalues({\mdseries\slshape CC})\index{DualMatrixOfEigenvalues@\texttt{DualMatrixOfEigenvalues}!for IsCoherentConfiguration}
\label{DualMatrixOfEigenvalues:for IsCoherentConfiguration}
}\hfill{\scriptsize (attribute)}}\\
\textbf{\indent Returns:\ }
Q 



 Returns a the dual matrix of eigenvalues, $Q$, for a coherent configuration CC. }

 

\subsection{\textcolor{Chapter }{MinimalIdempotents (for IsCoherentConfiguration)}}
\logpage{[ 2, 2, 6 ]}\nobreak
\hyperdef{L}{X78DE09E686836C4C}{}
{\noindent\textcolor{FuncColor}{$\triangleright$\enspace\texttt{MinimalIdempotents({\mdseries\slshape CC})\index{MinimalIdempotents@\texttt{MinimalIdempotents}!for IsCoherentConfiguration}
\label{MinimalIdempotents:for IsCoherentConfiguration}
}\hfill{\scriptsize (attribute)}}\\
\textbf{\indent Returns:\ }
L 



 Returns a list $L$ which is a basis of minimal idempotents for the adjacency algebra of a
coherent configuration CC. The $i$-th entry of $L$ is $E_{i-1}$. }

 }

 
\section{\textcolor{Chapter }{Properties of coherent configurations}}\label{Chapter_Functionality_Section_Properties_of_coherent_configurations}
\logpage{[ 2, 3, 0 ]}
\hyperdef{L}{X86E712CB7B567E54}{}
{
  

\subsection{\textcolor{Chapter }{IsHomogeneous (for IsCoherentConfiguration)}}
\logpage{[ 2, 3, 1 ]}\nobreak
\hyperdef{L}{X7A7AB8E784E5AF9C}{}
{\noindent\textcolor{FuncColor}{$\triangleright$\enspace\texttt{IsHomogeneous({\mdseries\slshape CC})\index{IsHomogeneous@\texttt{IsHomogeneous}!for IsCoherentConfiguration}
\label{IsHomogeneous:for IsCoherentConfiguration}
}\hfill{\scriptsize (property)}}\\
\textbf{\indent Returns:\ }
true or false 



 Checks if the input is a homogeneous coherent configuration. }

 

\subsection{\textcolor{Chapter }{IsCommutative (for IsCoherentConfiguration)}}
\logpage{[ 2, 3, 2 ]}\nobreak
\hyperdef{L}{X7E31B5D47D38B4FC}{}
{\noindent\textcolor{FuncColor}{$\triangleright$\enspace\texttt{IsCommutative({\mdseries\slshape CC})\index{IsCommutative@\texttt{IsCommutative}!for IsCoherentConfiguration}
\label{IsCommutative:for IsCoherentConfiguration}
}\hfill{\scriptsize (property)}}\\
\textbf{\indent Returns:\ }
true or false 



 Checks if the input is a commutative coherent configuration. }

 

\subsection{\textcolor{Chapter }{IsSymmetricCoherentConfiguration (for IsCoherentConfiguration)}}
\logpage{[ 2, 3, 3 ]}\nobreak
\hyperdef{L}{X7F01D8A587A3CABF}{}
{\noindent\textcolor{FuncColor}{$\triangleright$\enspace\texttt{IsSymmetricCoherentConfiguration({\mdseries\slshape CC})\index{IsSymmetricCoherentConfiguration@\texttt{IsSymmetricCoherentConfiguration}!for IsCoherentConfiguration}
\label{IsSymmetricCoherentConfiguration:for IsCoherentConfiguration}
}\hfill{\scriptsize (property)}}\\
\textbf{\indent Returns:\ }
true or false 



 Checks if the input is a symmetric coherent configuration. }

 

\subsection{\textcolor{Chapter }{IsCoherentConfigurationByOrbitals (for IsCoherentConfiguration)}}
\logpage{[ 2, 3, 4 ]}\nobreak
\hyperdef{L}{X7E1FCA0580ABB86F}{}
{\noindent\textcolor{FuncColor}{$\triangleright$\enspace\texttt{IsCoherentConfigurationByOrbitals({\mdseries\slshape CC})\index{IsCoherentConfigurationByOrbitals@\texttt{IsCoherentConfigurationByOrbitals}!for IsCoherentConfiguration}
\label{IsCoherentConfigurationByOrbitals:for IsCoherentConfiguration}
}\hfill{\scriptsize (property)}}\\
\textbf{\indent Returns:\ }
true or false 



 Checks if the coherent configuration $CC$ can be constructed from relations defined the orbitals of a transitive group $G$ acting a set $X$. }

 

\subsection{\textcolor{Chapter }{IsGenerouslyTransitive (for IsPermGroup)}}
\logpage{[ 2, 3, 5 ]}\nobreak
\hyperdef{L}{X7E9D8FA485C36A2B}{}
{\noindent\textcolor{FuncColor}{$\triangleright$\enspace\texttt{IsGenerouslyTransitive({\mdseries\slshape G})\index{IsGenerouslyTransitive@\texttt{IsGenerouslyTransitive}!for IsPermGroup}
\label{IsGenerouslyTransitive:for IsPermGroup}
}\hfill{\scriptsize (property)}}\\
\textbf{\indent Returns:\ }
true or false 



 Checks if $G$ is generously transitive. }

 

\subsection{\textcolor{Chapter }{IsGenerouslyTransitive (for IsPermGroup, IsList)}}
\logpage{[ 2, 3, 6 ]}\nobreak
\hyperdef{L}{X7DB4FFBC7C0A2173}{}
{\noindent\textcolor{FuncColor}{$\triangleright$\enspace\texttt{IsGenerouslyTransitive({\mdseries\slshape G, L})\index{IsGenerouslyTransitive@\texttt{IsGenerouslyTransitive}!for IsPermGroup, IsList}
\label{IsGenerouslyTransitive:for IsPermGroup, IsList}
}\hfill{\scriptsize (operation)}}\\
\textbf{\indent Returns:\ }
true or false 



 Checks that the group $G$ acts generously transitive on the set $L$. }

 

\subsection{\textcolor{Chapter }{IsSchurian (for IsCoherentConfiguration)}}
\logpage{[ 2, 3, 7 ]}\nobreak
\hyperdef{L}{X8664FCA57C5D6B5A}{}
{\noindent\textcolor{FuncColor}{$\triangleright$\enspace\texttt{IsSchurian({\mdseries\slshape CC})\index{IsSchurian@\texttt{IsSchurian}!for IsCoherentConfiguration}
\label{IsSchurian:for IsCoherentConfiguration}
}\hfill{\scriptsize (property)}}\\
\textbf{\indent Returns:\ }
true or false 



 Checks if the input is a Schurian coherent configuration, that is, if the
automorphism group is generously transitive. }

 }

 
\section{\textcolor{Chapter }{Attributes of coherent configurations}}\label{Chapter_Functionality_Section_Attributes_of_coherent_configurations}
\logpage{[ 2, 4, 0 ]}
\hyperdef{L}{X823FDE0D7B41826F}{}
{
  

\subsection{\textcolor{Chapter }{ClassOfAssociationScheme (for IsCoherentConfiguration)}}
\logpage{[ 2, 4, 1 ]}\nobreak
\hyperdef{L}{X7E7B3AE7810B2930}{}
{\noindent\textcolor{FuncColor}{$\triangleright$\enspace\texttt{ClassOfAssociationScheme({\mdseries\slshape CC})\index{ClassOfAssociationScheme@\texttt{ClassOfAssociationScheme}!for IsCoherentConfiguration}
\label{ClassOfAssociationScheme:for IsCoherentConfiguration}
}\hfill{\scriptsize (attribute)}}\\
\textbf{\indent Returns:\ }
d 



 Returns $d$ for a $d$-class association scheme. }

 

\subsection{\textcolor{Chapter }{Order (for IsCoherentConfiguration)}}
\logpage{[ 2, 4, 2 ]}\nobreak
\hyperdef{L}{X821670F48367E341}{}
{\noindent\textcolor{FuncColor}{$\triangleright$\enspace\texttt{Order({\mdseries\slshape CC})\index{Order@\texttt{Order}!for IsCoherentConfiguration}
\label{Order:for IsCoherentConfiguration}
}\hfill{\scriptsize (attribute)}}\\
\textbf{\indent Returns:\ }
n 



 Returns the order $n$ (number of vertices) of the coherent configuration. }

 

\subsection{\textcolor{Chapter }{Valencies (for IsCoherentConfiguration)}}
\logpage{[ 2, 4, 3 ]}\nobreak
\hyperdef{L}{X787AEF2279B59D51}{}
{\noindent\textcolor{FuncColor}{$\triangleright$\enspace\texttt{Valencies({\mdseries\slshape CC})\index{Valencies@\texttt{Valencies}!for IsCoherentConfiguration}
\label{Valencies:for IsCoherentConfiguration}
}\hfill{\scriptsize (attribute)}}\\
\textbf{\indent Returns:\ }
L 



 Returns a list L of valencies of a coherent configuration CC. The $i$-th entry of $L$ is $k_{i-1}$. (Check this for nonsymmetric CCs) }

 

\subsection{\textcolor{Chapter }{AutomorphismGroup (for IsCoherentConfiguration)}}
\logpage{[ 2, 4, 4 ]}\nobreak
\hyperdef{L}{X83F116ED7EFE71B0}{}
{\noindent\textcolor{FuncColor}{$\triangleright$\enspace\texttt{AutomorphismGroup({\mdseries\slshape CC})\index{AutomorphismGroup@\texttt{AutomorphismGroup}!for IsCoherentConfiguration}
\label{AutomorphismGroup:for IsCoherentConfiguration}
}\hfill{\scriptsize (attribute)}}\\
\textbf{\indent Returns:\ }
G 



 Returns the automorphism group $G$ of the coherent configuration CC. $G$ is a permutation group acting on the index set of the veritices. If $G$ is not already known and must be computed, then the package "Digraphs" is
required. }

 }

 }

 \def\indexname{Index\logpage{[ "Ind", 0, 0 ]}
\hyperdef{L}{X83A0356F839C696F}{}
}

\cleardoublepage
\phantomsection
\addcontentsline{toc}{chapter}{Index}


\printindex

\immediate\write\pagenrlog{["Ind", 0, 0], \arabic{page},}
\newpage
\immediate\write\pagenrlog{["End"], \arabic{page}];}
\immediate\closeout\pagenrlog
\end{document}

% generated by GAPDoc2LaTeX from XML source (Frank Luebeck)
\documentclass[a4paper,11pt]{report}

\usepackage[top=37mm,bottom=37mm,left=27mm,right=27mm]{geometry}
\sloppy
\pagestyle{myheadings}
\usepackage{amssymb}
\usepackage[utf8]{inputenc}
\usepackage{makeidx}
\makeindex
\usepackage{color}
\definecolor{FireBrick}{rgb}{0.5812,0.0074,0.0083}
\definecolor{RoyalBlue}{rgb}{0.0236,0.0894,0.6179}
\definecolor{RoyalGreen}{rgb}{0.0236,0.6179,0.0894}
\definecolor{RoyalRed}{rgb}{0.6179,0.0236,0.0894}
\definecolor{LightBlue}{rgb}{0.8544,0.9511,1.0000}
\definecolor{Black}{rgb}{0.0,0.0,0.0}

\definecolor{linkColor}{rgb}{0.0,0.0,0.554}
\definecolor{citeColor}{rgb}{0.0,0.0,0.554}
\definecolor{fileColor}{rgb}{0.0,0.0,0.554}
\definecolor{urlColor}{rgb}{0.0,0.0,0.554}
\definecolor{promptColor}{rgb}{0.0,0.0,0.589}
\definecolor{brkpromptColor}{rgb}{0.589,0.0,0.0}
\definecolor{gapinputColor}{rgb}{0.589,0.0,0.0}
\definecolor{gapoutputColor}{rgb}{0.0,0.0,0.0}

%%  for a long time these were red and blue by default,
%%  now black, but keep variables to overwrite
\definecolor{FuncColor}{rgb}{0.0,0.0,0.0}
%% strange name because of pdflatex bug:
\definecolor{Chapter }{rgb}{0.0,0.0,0.0}
\definecolor{DarkOlive}{rgb}{0.1047,0.2412,0.0064}


\usepackage{fancyvrb}

\usepackage{mathptmx,helvet}
\usepackage[T1]{fontenc}
\usepackage{textcomp}


\usepackage[
            pdftex=true,
            bookmarks=true,        
            a4paper=true,
            pdftitle={Written with GAPDoc},
            pdfcreator={LaTeX with hyperref package / GAPDoc},
            colorlinks=true,
            backref=page,
            breaklinks=true,
            linkcolor=linkColor,
            citecolor=citeColor,
            filecolor=fileColor,
            urlcolor=urlColor,
            pdfpagemode={UseNone}, 
           ]{hyperref}

\newcommand{\maintitlesize}{\fontsize{50}{55}\selectfont}

% write page numbers to a .pnr log file for online help
\newwrite\pagenrlog
\immediate\openout\pagenrlog =\jobname.pnr
\immediate\write\pagenrlog{PAGENRS := [}
\newcommand{\logpage}[1]{\protect\write\pagenrlog{#1, \thepage,}}
%% were never documented, give conflicts with some additional packages

\newcommand{\GAP}{\textsf{GAP}}

%% nicer description environments, allows long labels
\usepackage{enumitem}
\setdescription{style=nextline}

%% depth of toc
\setcounter{tocdepth}{1}





%% command for ColorPrompt style examples
\newcommand{\gapprompt}[1]{\color{promptColor}{\bfseries #1}}
\newcommand{\gapbrkprompt}[1]{\color{brkpromptColor}{\bfseries #1}}
\newcommand{\gapinput}[1]{\color{gapinputColor}{#1}}


\begin{document}

\logpage{[ 0, 0, 0 ]}
\begin{titlepage}
\mbox{}\vfill

\begin{center}{\maintitlesize \textbf{ AssociationSchemes \mbox{}}}\\
\vfill

\hypersetup{pdftitle= AssociationSchemes }
\markright{\scriptsize \mbox{}\hfill  AssociationSchemes  \hfill\mbox{}}
{\Huge \textbf{ A \textsf{GAP} package for working with association schemes and homogeneous coherent
configurations \mbox{}}}\\
\vfill

{\Huge  2.0.0 \mbox{}}\\[1cm]
{ 2 February 2022 \mbox{}}\\[1cm]
\mbox{}\\[2cm]
{\Large \textbf{ John Bamberg\\
    \mbox{}}}\\
{\Large \textbf{ Akihide Hanaki\\
    \mbox{}}}\\
{\Large \textbf{ Jesse Lansdown\\
    \mbox{}}}\\
\hypersetup{pdfauthor= John Bamberg\\
    ;  Akihide Hanaki\\
    ;  Jesse Lansdown\\
    }
\end{center}\vfill

\mbox{}\\
{\mbox{}\\
\small \noindent \textbf{ John Bamberg\\
    }  Email: \href{mailto://john.bamberg@uwa.edu.au} {\texttt{john.bamberg@uwa.edu.au}}\\
  Homepage: \href{http://school.maths.uwa.edu.au/~bamberg/} {\texttt{http://school.maths.uwa.edu.au/\texttt{\symbol{126}}bamberg/}}\\
  Address: \begin{minipage}[t]{8cm}\noindent
 John Bamberg\\
 School of Mathematics and Statistics\\
 The University of Western Australia\\
 35 Stirling Highway\\
 Crawley WA 6009, Perth\\
 Australia\\
 \end{minipage}
}\\
{\mbox{}\\
\small \noindent \textbf{ Akihide Hanaki\\
    }  Email: \href{mailto://hanaki@shinshu-u.ac.jp} {\texttt{hanaki@shinshu-u.ac.jp}}\\
  Homepage: \href{http://math.shinshu-u.ac.jp/~hanaki/} {\texttt{http://math.shinshu-u.ac.jp/\texttt{\symbol{126}}hanaki/}}\\
  Address: \begin{minipage}[t]{8cm}\noindent
 Akihide Hanaki\\
 Department of Mathematics\\
 Faculty of Science, Shinshu University\\
 Matsumoto 390-8621, Japan\\
 \end{minipage}
}\\
{\mbox{}\\
\small \noindent \textbf{ Jesse Lansdown\\
    }  Email: \href{mailto://jesse.lansdown@research.uwa.edu.au} {\texttt{jesse.lansdown@research.uwa.edu.au}}\\
  Homepage: \href{http://www.jesselansdown.com} {\texttt{http://www.jesselansdown.com}}\\
  Address: \begin{minipage}[t]{8cm}\noindent
 Jesse Lansdown\\
 School of Mathematics and Statistics\\
 The University of Western Australia\\
 35 Stirling Highway\\
 Crawley WA 6009, Perth\\
 Australia\\
 \end{minipage}
}\\
\end{titlepage}

\newpage\setcounter{page}{2}
{\small 
\section*{Abstract}
\logpage{[ 0, 0, 1 ]}
 AssociationSchemes is a GAP package for working with association schemes and
homogeneous coherent configurations. \mbox{}}\\[1cm]
{\small 
\section*{Copyright}
\logpage{[ 0, 0, 2 ]}
 {\copyright} 2019 - 2022 John Bamberg, Akihide Hanaki, Jesse Lansdown

 This program is free software; you can redistribute it and/or modify it under
the terms of the GNU General Public License as published by the Free Software
Foundation; either version 2 of the License, or (at your option) any later
version.



This program is distributed in the hope that it will be useful, but WITHOUT
ANY WARRANTY; without even the implied warranty of MERCHANTABILITY or FITNESS
FOR A PARTICULAR PURPOSE. See the GNU General Public License for more details.



You should have received a copy of the GNU General Public License along with
this program; if not, write to the Free Software Foundation, Inc., 51 Franklin
Street, Fifth Floor, Boston, MA 02110-1301, USA. \mbox{}}\\[1cm]
{\small 
\section*{Acknowledgements}
\logpage{[ 0, 0, 3 ]}
 The third author would like to acknowledge the support of an Australian
Government Research Training Program (RTP) Scholarship while writing this
software. The first and third authors are also grateful for the 2019 CMSC
Retreat for providing an opportunity and environment for some of the founding
work on the package. \mbox{}}\\[1cm]
\newpage

\def\contentsname{Contents\logpage{[ 0, 0, 4 ]}}

\tableofcontents
\newpage

     
\chapter{\textcolor{Chapter }{Introduction}}\label{Chapter_Introduction}
\logpage{[ 1, 0, 0 ]}
\hyperdef{L}{X7DFB63A97E67C0A1}{}
{
  
\section{\textcolor{Chapter }{Welcome to AssociationSchemes}}\label{Chapter_Introduction_Section_Welcome_to_AssociationSchemes}
\logpage{[ 1, 1, 0 ]}
\hyperdef{L}{X84CBAD1987E00213}{}
{
  

 AssociationSchemes is a \textsf{GAP}\cite{GAP4} package for working with association schemes and homogeneous coherent
configurations. 

 For definitions and more information on the theory of association schemes and
homogeneous coherent configurations, we refer you to \cite{BannaiIto} and \cite{Godsil}. 

 It is important to note that the term "association scheme" is used differently
by different authors. We reserve the term "association scheme" to mean a
symmetric coherent configuration, and use "homogeneous coherent configuration"
to refer to the more general objects. 

 }

 
\section{\textcolor{Chapter }{Citing AssociationSchemes}}\label{Chapter_Introduction_Section_Citing_AssociationSchemes}
\logpage{[ 1, 2, 0 ]}
\hyperdef{L}{X8424F0C17AAE4D39}{}
{
  

 If you use AssociationSchemes in research leading to publication please cite
it as you would a paper. Example citations and a BibTeX entry are given below.
Please check that the version and DOI match the version of AssociationSchemes
used in your research. 

 Please also inform us by email of the paper, as we are very interested to hear
how AssociationSchemes is being used! 

 
\begin{Verbatim}[commandchars=!|C,fontsize=\small,frame=single,label=Example]
  @article{AssociationSchemes,
  Author = {Bamberg,  J.  and  Hanaki,  A.  and  Lansdown, J.},
  Doi = {10.5281/zenodo.2634955},
  Key = {AssociationSchemes},
  Title = {{AssociationSchemes -- AssociationSchemes: A GAP package for working
  with association schemes and homogeneous coherent configurations, Version 2.0.0}},
  Url = {http://doi.org/10.5281/zenodo.2634955},
  Year = 2022	,
  }
\end{Verbatim}
 

 }

 
\section{\textcolor{Chapter }{Dependencies}}\label{Chapter_Introduction_Section_Dependencies}
\logpage{[ 1, 3, 0 ]}
\hyperdef{L}{X7C0B91E37C8FC531}{}
{
  

 AssociationSchemes requires 

 
\begin{itemize}
\item  \textsf{GAP} 4.8 (or later) 
\end{itemize}
 

 as well as the following \textsf{GAP} packages: 

 
\begin{itemize}
\item  Digraphs 0.13.0 (or later) 
\end{itemize}
 

 You may of course use AssociationSchemes without the above packages, however
the corresponding functionality will be unavailable. 

 You may also want the following packages: 

 
\begin{itemize}
\item  NautyTracesInterface 0.2 (or later) 
\end{itemize}
 

 Note that NautyTracesInterface is not yet a deposited package. It can be
obtained from the link in the appendix. It is not necessary, but it might
improve speed of certain things such as finding the automorphism group, and
checking isomorphisms. 

 }

 
\section{\textcolor{Chapter }{Installation}}\label{Chapter_Introduction_Section_Installation}
\logpage{[ 1, 4, 0 ]}
\hyperdef{L}{X8360C04082558A12}{}
{
  

 To install AssociationSchemes, simply copy to the "pkg" directory of your GAP
installation and unzip. 

 Alternatively, you may load the package from a location other than the \textsf{GAP} "pkg" directory by using the -L flag when opening \textsf{GAP}. Note that this requires the parent directory of AssociationSchemes to be
called "pkg". See the \textsf{GAP} documentation for more details on how to do this. This is useful, for example,
when administrative priveledges are required to access the \textsf{GAP} root directory. 

 }

 }

   
\chapter{\textcolor{Chapter }{Getting Started}}\label{Chapter_Getting_Started}
\logpage{[ 2, 0, 0 ]}
\hyperdef{L}{X7B1863E17896BCE1}{}
{
  
\section{\textcolor{Chapter }{Tutorial - A first session with AssociationSchemes}}\label{Chapter_Getting_Started_Section_Tutorial_-_A_first_session_with_AssociationSchemes}
\logpage{[ 2, 1, 0 ]}
\hyperdef{L}{X7E1DAE3779F9F9E0}{}
{
  

 In this section we provide a "first session" introduction to the
AssociationSchemes package. It is intended to demonstrate the basic functions
of the package through a series of small examples. More detailed descriptions
of each of the methods are given in the chapter "Functionality". The
fundamental method of describing a scheme in the AssociationSchemes package is
via its relation matrix. Take for example the following relation matrix: 
\begin{Verbatim}[commandchars=!@|,fontsize=\small,frame=single,label=Example]
  !gapprompt@gap>| !gapinput@M:=|
  !gapprompt@>| !gapinput@[ [  0,  1,  2,  2,  3,  3,  3,  3,  3,  3,  3,  3 ],|
  !gapprompt@>| !gapinput@  [  1,  0,  2,  2,  3,  3,  3,  3,  3,  3,  3,  3 ],|
  !gapprompt@>| !gapinput@  [  2,  2,  0,  1,  3,  3,  3,  3,  3,  3,  3,  3 ],|
  !gapprompt@>| !gapinput@  [  2,  2,  1,  0,  3,  3,  3,  3,  3,  3,  3,  3 ],|
  !gapprompt@>| !gapinput@  [  3,  3,  3,  3,  0,  1,  2,  2,  3,  3,  3,  3 ],|
  !gapprompt@>| !gapinput@  [  3,  3,  3,  3,  1,  0,  2,  2,  3,  3,  3,  3 ],|
  !gapprompt@>| !gapinput@  [  3,  3,  3,  3,  2,  2,  0,  1,  3,  3,  3,  3 ],|
  !gapprompt@>| !gapinput@  [  3,  3,  3,  3,  2,  2,  1,  0,  3,  3,  3,  3 ],|
  !gapprompt@>| !gapinput@  [  3,  3,  3,  3,  3,  3,  3,  3,  0,  1,  2,  2 ],|
  !gapprompt@>| !gapinput@  [  3,  3,  3,  3,  3,  3,  3,  3,  1,  0,  2,  2 ],|
  !gapprompt@>| !gapinput@  [  3,  3,  3,  3,  3,  3,  3,  3,  2,  2,  0,  1 ],|
  !gapprompt@>| !gapinput@  [  3,  3,  3,  3,  3,  3,  3,  3,  2,  2,  1,  0 ] ];;|
\end{Verbatim}
 To construct a scheme from this matrix, we use the CoherentConfiguration
command. 
\begin{Verbatim}[commandchars=!@|,fontsize=\small,frame=single,label=Example]
  !gapprompt@gap>| !gapinput@CC := HomogeneousCoherentConfiguration(M);;|
\end{Verbatim}
 CoherentConfiguration performs a number of checks as it constructs the scheme
to make sure that it is in fact a homogeneous coherent configuration. However
if you are confident that M does in fact define a scheme, then you can skip
the checks by using CoherentConfigurationNC. Do not do this unless you are
sure! We can display the scheme and see that \textsf{GAP} already knows the class and order of CC, as well that CC is symmetric and
commutative. 
\begin{Verbatim}[commandchars=!@|,fontsize=\small,frame=single,label=Example]
  !gapprompt@gap>| !gapinput@Display(CC);|
  3-class association scheme of order 12.
    Symmetric: true
    Commutative: true
\end{Verbatim}
 We can directly ask if CC is commutative or symmetric. 
\begin{Verbatim}[commandchars=!@|,fontsize=\small,frame=single,label=Example]
  !gapprompt@gap>| !gapinput@IsCommutative(CC);|
  true
  !gapprompt@gap>| !gapinput@IsSymmetricCoherentConfiguration(CC);|
  true
\end{Verbatim}
 We can retrieve the relation matrix of a scheme 
\begin{Verbatim}[commandchars=!@|,fontsize=\small,frame=single,label=Example]
  !gapprompt@gap>| !gapinput@relmat := RelationMatrix(CC);;|
  !gapprompt@gap>| !gapinput@relmat = M;|
  true
\end{Verbatim}
 
\begin{Verbatim}[commandchars=!@|,fontsize=\small,frame=single,label=Example]
  !gapprompt@gap>| !gapinput@P := MatrixOfEigenvalues(CC);;|
  !gapprompt@gap>| !gapinput@Display(P);|
  [ [   1,   1,   2,   8 ],
    [   1,   1,   2,  -4 ],
    [   1,   1,  -2,   0 ],
    [   1,  -1,   0,   0 ] ]
\end{Verbatim}
 If we try displaying again, we will also obtain the matrix of eigenvalues and
the dual matrix of eigenvalues. 
\begin{Verbatim}[commandchars=!@|,fontsize=\small,frame=single,label=Example]
  !gapprompt@gap>| !gapinput@Display(CC);|
  3-class association scheme of order 12.
    Symmetric: true
    Commutative: true
    Metric: false
      Admits metric ordering: false
    Matrix of eigenvalues:
  [ [   1,   1,   2,   8 ],
    [   1,   1,   2,  -4 ],
    [   1,   1,  -2,   0 ],
    [   1,  -1,   0,   0 ] ]
    Matrix of dual eigenvalues:
  [ [   1,   2,   3,   6 ],
    [   1,   2,   3,  -6 ],
    [   1,   2,  -3,   0 ],
    [   1,  -1,   0,   0 ] ]
\end{Verbatim}
 If you want to print CC, it will return the relation matrix. This is useful if
you want to print to a file for exmaple. 
\begin{Verbatim}[commandchars=!@|,fontsize=\small,frame=single,label=Example]
  !gapprompt@gap>| !gapinput@Print(CC);|
  [ [ 0, 1, 2, 2, 3, 3, 3, 3, 3, 3, 3, 3 ], [ 1, 0, 2, 2, 3, 3, 3, 3, 3, 3, 3, 3 ], 
    [ 2, 2, 0, 1, 3, 3, 3, 3, 3, 3, 3, 3 ], [ 2, 2, 1, 0, 3, 3, 3, 3, 3, 3, 3, 3 ], 
    [ 3, 3, 3, 3, 0, 1, 2, 2, 3, 3, 3, 3 ], [ 3, 3, 3, 3, 1, 0, 2, 2, 3, 3, 3, 3 ], 
    [ 3, 3, 3, 3, 2, 2, 0, 1, 3, 3, 3, 3 ], [ 3, 3, 3, 3, 2, 2, 1, 0, 3, 3, 3, 3 ], 
    [ 3, 3, 3, 3, 3, 3, 3, 3, 0, 1, 2, 2 ], [ 3, 3, 3, 3, 3, 3, 3, 3, 1, 0, 2, 2 ], 
    [ 3, 3, 3, 3, 3, 3, 3, 3, 2, 2, 0, 1 ], [ 3, 3, 3, 3, 3, 3, 3, 3, 2, 2, 1, 0 ] ]
\end{Verbatim}
 You can obtain the adjacency matrices by doing: 
\begin{Verbatim}[commandchars=!@|,fontsize=\small,frame=single,label=Example]
  !gapprompt@gap>| !gapinput@AdjacencyMatrices(CC);;|
\end{Verbatim}
 If you were able to calculate the matrix of eigenvalues, then you can also
construct the minimal idempotents $E_i$ 
\begin{Verbatim}[commandchars=!@|,fontsize=\small,frame=single,label=Example]
  !gapprompt@gap>| !gapinput@MinimalIdempotents(CC);;|
\end{Verbatim}
 Note that if CC is Schurian (or has a transitive group associated with it)
then MinimalIdempotents will be much faster! You can check if a scheme is
schurian by doing 
\begin{Verbatim}[commandchars=!@|,fontsize=\small,frame=single,label=Example]
  !gapprompt@gap>| !gapinput@IsSchurian(CC);|
  true
\end{Verbatim}
 In doing this, a graph is constructed and the automorphism group for CC is
found. We can also find the automorphism group directly. 
\begin{Verbatim}[commandchars=!@|,fontsize=\small,frame=single,label=Example]
  !gapprompt@gap>| !gapinput@AutomorphismGroup(CC);|
  <permutation group with 11 generators>
\end{Verbatim}
 We can define homogeneous coherent figurations from transitive groups. This is
typically fast. 
\begin{Verbatim}[commandchars=!@|,fontsize=\small,frame=single,label=Example]
  !gapprompt@gap>| !gapinput@G := Group( [ ( 6,10)( 7,11)( 8,12)( 9,13)(15,28)(16,29)(17,30)(18,31)|
  !gapprompt@>| !gapinput@	(20,37)(21,38)(22,39)(23,40)(24,33)(25,34)(26,35)(27,36),|
  !gapprompt@>| !gapinput@    ( 1,15,22,31,18,26,16, 2, 5)( 3,24,21)( 4,20,40,29,11, 6,28,27,25)|
  !gapprompt@>| !gapinput@    ( 7,10,14)( 8,33,35,39,38,12,32,13,19)( 9,37,34)(23,36,30),|
  !gapprompt@>| !gapinput@    ( 3, 4)( 7,11)( 8, 9)(12,13)(15,28)(17,31)(18,30)(19,32)(20,33)|
  !gapprompt@>| !gapinput@    (21,25)(22,36)(23,35)(24,37)(26,40)(27,39)(34,38), () ] );;|
  !gapprompt@gap>| !gapinput@HomogeneousCoherentConfigurationByOrbitals(G);;|
\end{Verbatim}
 If G is transitive, we can construct a Schurian coherent configuration. If G
is generously transitive, then we can constuct a Schurian association scheme
(it will be symmetric) 
\begin{Verbatim}[commandchars=!@|,fontsize=\small,frame=single,label=Example]
  !gapprompt@gap>| !gapinput@IsTransitive(G);|
  true
  !gapprompt@gap>| !gapinput@SchurianCoherentConfiguration(G);;|
  !gapprompt@gap>| !gapinput@IsGenerouslyTransitive(G);|
  true
  !gapprompt@gap>| !gapinput@SchurianAssociationScheme(G);;|
\end{Verbatim}
 If we have a group G and subgroup H such that G acts transitively on G/H, then
we can also use the following construction. 
\begin{Verbatim}[commandchars=!@|,fontsize=\small,frame=single,label=Example]
  !gapprompt@gap>| !gapinput@G:=SymmetricGroup(5);;|
  !gapprompt@gap>| !gapinput@H:=Stabiliser(G, 1);;|
  !gapprompt@gap>| !gapinput@HomogeneousCoherentConfigurationByOrbitals(G, H);;|
\end{Verbatim}
 There are a number of special constructors, such as for Johnson schemes 
\begin{Verbatim}[commandchars=!@|,fontsize=\small,frame=single,label=Example]
  !gapprompt@gap>| !gapinput@JohnsonScheme(10,3);|
  3-class association scheme of order 120.
\end{Verbatim}
 AssociationSchemes also comes with a library of association schemes on small
numbers of vertices, according to \cite{Hanaki}. 
\begin{Verbatim}[commandchars=!@|,fontsize=\small,frame=single,label=Example]
  !gapprompt@gap>| !gapinput@m:=HomogeneousCoherentConfiguration(12, 7);;|
\end{Verbatim}
 We can test if two schemes are equal with "=". This will return true if the
schemes have the same relation matrix. The previous example from the library
is in fact the same as the example constructed from the matrix M at the start. 
\begin{Verbatim}[commandchars=!@|,fontsize=\small,frame=single,label=Example]
  !gapprompt@gap>| !gapinput@CC = m;|
  true
\end{Verbatim}
 There is also the option to create a fusion scheme. This takes a partition of
the relations, (where [0] must be cell of the partition. If the resulting
fusion is not a valid scheme this will return fail; 
\begin{Verbatim}[commandchars=!@|,fontsize=\small,frame=single,label=Example]
  !gapprompt@gap>| !gapinput@FusionOfHomogeneousCoherentConfiguration(m, [[0], [1,2],[3]]);|
  2-class association scheme of order 12.
\end{Verbatim}
 }

 }

   
\chapter{\textcolor{Chapter }{Homogeneous Coherent Configuration objects}}\label{Chapter_Homogeneous_Coherent_Configuration_objects}
\logpage{[ 3, 0, 0 ]}
\hyperdef{L}{X7CD35ADB7EC02992}{}
{
  
\section{\textcolor{Chapter }{Core functionality}}\label{Chapter_Homogeneous_Coherent_Configuration_objects_Section_Core_functionality}
\logpage{[ 3, 1, 0 ]}
\hyperdef{L}{X851AE2B98382B550}{}
{
  

\subsection{\textcolor{Chapter }{HomogeneousCoherentConfiguration (for IsMatrix)}}
\logpage{[ 3, 1, 1 ]}\nobreak
\hyperdef{L}{X862EA9667BCADC04}{}
{\noindent\textcolor{FuncColor}{$\triangleright$\enspace\texttt{HomogeneousCoherentConfiguration({\mdseries\slshape M})\index{HomogeneousCoherentConfiguration@\texttt{HomogeneousCoherentConfiguration}!for IsMatrix}
\label{HomogeneousCoherentConfiguration:for IsMatrix}
}\hfill{\scriptsize (operation)}}\\
\textbf{\indent Returns:\ }
homogeneous coherent configuration 



 Takes the relationship matrix, $M$, describing a homogeneous coherent configuration and returns a
HomogeneousCoherentConfiguration object. The matrix $M = \sum_{i=0}^d i A_i$, where $A_i$ are the adjacency matrices describing a coherent configuration. Checks that
the matrix satisfies the axioms of a homogeneous coherent configuration. }

 

\subsection{\textcolor{Chapter }{HomogeneousCoherentConfigurationNC (for IsMatrix)}}
\logpage{[ 3, 1, 2 ]}\nobreak
\hyperdef{L}{X7E6AEC267E78078C}{}
{\noindent\textcolor{FuncColor}{$\triangleright$\enspace\texttt{HomogeneousCoherentConfigurationNC({\mdseries\slshape M})\index{HomogeneousCoherentConfigurationNC@\texttt{HomogeneousCoherentConfigurationNC}!for IsMatrix}
\label{HomogeneousCoherentConfigurationNC:for IsMatrix}
}\hfill{\scriptsize (operation)}}\\
\textbf{\indent Returns:\ }
homogeneous coherent configuration 



 Same as HomogeneousCoherentConfiguration but without performing any checks.
Use this method only if you know with certainty that $M$ describes a coherent configuration. }

 

\subsection{\textcolor{Chapter }{AssociationScheme (for IsMatrix)}}
\logpage{[ 3, 1, 3 ]}\nobreak
\hyperdef{L}{X7C92838B87816988}{}
{\noindent\textcolor{FuncColor}{$\triangleright$\enspace\texttt{AssociationScheme({\mdseries\slshape M})\index{AssociationScheme@\texttt{AssociationScheme}!for IsMatrix}
\label{AssociationScheme:for IsMatrix}
}\hfill{\scriptsize (operation)}}\\
\textbf{\indent Returns:\ }
homogeneous coherent configuration 



 Takes the relationship matrix, $M$, describing an associatioin scheme and returns an association scheme
(symmetric coherent configuration). This is simply a
HomogeneousCoherentConfiguration object, but with the known property of being
symmetric. The matrix $M = \sum_{i=0}^d i A_i$, where $A_i$ are the adjacency matrices describing an association scheme. Checks that the
matrix satisfies the association scheme axioms. }

 

\subsection{\textcolor{Chapter }{AssociationSchemeNC (for IsMatrix)}}
\logpage{[ 3, 1, 4 ]}\nobreak
\hyperdef{L}{X7E3565C17884BF57}{}
{\noindent\textcolor{FuncColor}{$\triangleright$\enspace\texttt{AssociationSchemeNC({\mdseries\slshape M})\index{AssociationSchemeNC@\texttt{AssociationSchemeNC}!for IsMatrix}
\label{AssociationSchemeNC:for IsMatrix}
}\hfill{\scriptsize (operation)}}\\
\textbf{\indent Returns:\ }
homogeneous coherent configuration 



 Same as AssociationScheme but without performing any checks. Use this method
only if you know with certainty that $M$ describes an association scheme (symmetric coherent configuration). }

 

\subsection{\textcolor{Chapter }{ReorderRelations (for IsHomogeneousCoherentConfiguration, IsList)}}
\logpage{[ 3, 1, 5 ]}\nobreak
\hyperdef{L}{X7C7210787E5A87D0}{}
{\noindent\textcolor{FuncColor}{$\triangleright$\enspace\texttt{ReorderRelations({\mdseries\slshape CC, L})\index{ReorderRelations@\texttt{ReorderRelations}!for IsHomogeneousCoherentConfiguration, IsList}
\label{ReorderRelations:for IsHomogeneousCoherentConfiguration, IsList}
}\hfill{\scriptsize (operation)}}\\
\textbf{\indent Returns:\ }
coherent configuration 



 Takes a homogeneous coherent configuration CC and a list L, where L is a
reordering of the relations. Returns a homogeneous coherent configuration
where the $i$-th relation of the CC becomes the $j$-th relation in the new homogeneous coherent configuration, where $j = L_i$. Note that $L$ must be equal to $\{0, \ldots, d \}$ as a set, and additionally requires that $L_1 = 0$. }

 

\subsection{\textcolor{Chapter }{RelationMatrix (for IsHomogeneousCoherentConfiguration)}}
\logpage{[ 3, 1, 6 ]}\nobreak
\hyperdef{L}{X7E5F3FFA862C3FF1}{}
{\noindent\textcolor{FuncColor}{$\triangleright$\enspace\texttt{RelationMatrix({\mdseries\slshape CC})\index{RelationMatrix@\texttt{RelationMatrix}!for IsHomogeneousCoherentConfiguration}
\label{RelationMatrix:for IsHomogeneousCoherentConfiguration}
}\hfill{\scriptsize (operation)}}\\
\textbf{\indent Returns:\ }
$M$ 



 Takes a homogeneous coherent configuration and returns the underlying relation
matrix $M = \sum_{i=0}^d i A_i$, where $A_i$ are the adjacency matrices of the coherent configuration }

 

\subsection{\textcolor{Chapter }{Relation (for IsHomogeneousCoherentConfiguration, IsPosInt, IsPosInt)}}
\logpage{[ 3, 1, 7 ]}\nobreak
\hyperdef{L}{X81B6F58485D31123}{}
{\noindent\textcolor{FuncColor}{$\triangleright$\enspace\texttt{Relation({\mdseries\slshape CC, x, y})\index{Relation@\texttt{Relation}!for IsHomogeneousCoherentConfiguration, IsPosInt, IsPosInt}
\label{Relation:for IsHomogeneousCoherentConfiguration, IsPosInt, IsPosInt}
}\hfill{\scriptsize (operation)}}\\
\textbf{\indent Returns:\ }
i 



 Takes a CC and two points, x and y, and returns i such that $(x, y) \in R_i$. }

 

\subsection{\textcolor{Chapter }{Neighbours (for IsHomogeneousCoherentConfiguration, IsPosInt, IsInt)}}
\logpage{[ 3, 1, 8 ]}\nobreak
\hyperdef{L}{X81A33D6D7B581567}{}
{\noindent\textcolor{FuncColor}{$\triangleright$\enspace\texttt{Neighbours({\mdseries\slshape CC, p, k})\index{Neighbours@\texttt{Neighbours}!for IsHomogeneousCoherentConfiguration, IsPosInt, IsInt}
\label{Neighbours:for IsHomogeneousCoherentConfiguration, IsPosInt, IsInt}
}\hfill{\scriptsize (operation)}}\\
\textbf{\indent Returns:\ }
L 



 Returns a list $L$ of all the points $y$ of CC such that $(p,y) \in R_k$. }

 

\subsection{\textcolor{Chapter }{Neighbours (for IsHomogeneousCoherentConfiguration, IsInt, IsList)}}
\logpage{[ 3, 1, 9 ]}\nobreak
\hyperdef{L}{X86B7FB54839FB92A}{}
{\noindent\textcolor{FuncColor}{$\triangleright$\enspace\texttt{Neighbours({\mdseries\slshape CC, p, L})\index{Neighbours@\texttt{Neighbours}!for IsHomogeneousCoherentConfiguration, IsInt, IsList}
\label{Neighbours:for IsHomogeneousCoherentConfiguration, IsInt, IsList}
}\hfill{\scriptsize (operation)}}\\
\textbf{\indent Returns:\ }
L 



 Returns a list $L$ of all the points $y$ of CC such that $(p,y) \in R_k$ for some $k \in L$. }

 

\subsection{\textcolor{Chapter }{IsCommutative (for IsHomogeneousCoherentConfiguration)}}
\logpage{[ 3, 1, 10 ]}\nobreak
\hyperdef{L}{X7A08AD147E087933}{}
{\noindent\textcolor{FuncColor}{$\triangleright$\enspace\texttt{IsCommutative({\mdseries\slshape CC})\index{IsCommutative@\texttt{IsCommutative}!for IsHomogeneousCoherentConfiguration}
\label{IsCommutative:for IsHomogeneousCoherentConfiguration}
}\hfill{\scriptsize (property)}}\\
\textbf{\indent Returns:\ }
true or false 



 Checks if the input is a commutative coherent configuration. }

 

\subsection{\textcolor{Chapter }{IsSymmetricCoherentConfiguration (for IsHomogeneousCoherentConfiguration)}}
\logpage{[ 3, 1, 11 ]}\nobreak
\hyperdef{L}{X811BE49B82475787}{}
{\noindent\textcolor{FuncColor}{$\triangleright$\enspace\texttt{IsSymmetricCoherentConfiguration({\mdseries\slshape CC})\index{IsSymmetricCoherentConfiguration@\texttt{IsSymmetricCoherentConfiguration}!for IsHomogeneousCoherentConfiguration}
\label{IsSymmetricCoherentConfiguration:for IsHomogeneousCoherentConfiguration}
}\hfill{\scriptsize (property)}}\\
\textbf{\indent Returns:\ }
true or false 



 Checks if the input is a symmetric coherent configuration. }

 

\subsection{\textcolor{Chapter }{IsAssociationScheme (for IsHomogeneousCoherentConfiguration)}}
\logpage{[ 3, 1, 12 ]}\nobreak
\hyperdef{L}{X801063A083CF6BA0}{}
{\noindent\textcolor{FuncColor}{$\triangleright$\enspace\texttt{IsAssociationScheme({\mdseries\slshape CC})\index{IsAssociationScheme@\texttt{IsAssociationScheme}!for IsHomogeneousCoherentConfiguration}
\label{IsAssociationScheme:for IsHomogeneousCoherentConfiguration}
}\hfill{\scriptsize (operation)}}\\
\textbf{\indent Returns:\ }
true or false 



 Alias for IsSymmetricCoherentConfiguration }

 

\subsection{\textcolor{Chapter }{NumberOfClasses (for IsHomogeneousCoherentConfiguration)}}
\logpage{[ 3, 1, 13 ]}\nobreak
\hyperdef{L}{X87DFF5E98044D47C}{}
{\noindent\textcolor{FuncColor}{$\triangleright$\enspace\texttt{NumberOfClasses({\mdseries\slshape CC})\index{NumberOfClasses@\texttt{NumberOfClasses}!for IsHomogeneousCoherentConfiguration}
\label{NumberOfClasses:for IsHomogeneousCoherentConfiguration}
}\hfill{\scriptsize (attribute)}}\\
\textbf{\indent Returns:\ }
d 



 Returns $d$ for a $d$-class homogeneous coherent configuration. }

 

\subsection{\textcolor{Chapter }{Rank (for IsHomogeneousCoherentConfiguration)}}
\logpage{[ 3, 1, 14 ]}\nobreak
\hyperdef{L}{X7FF700697A625831}{}
{\noindent\textcolor{FuncColor}{$\triangleright$\enspace\texttt{Rank({\mdseries\slshape CC})\index{Rank@\texttt{Rank}!for IsHomogeneousCoherentConfiguration}
\label{Rank:for IsHomogeneousCoherentConfiguration}
}\hfill{\scriptsize (operation)}}\\
\textbf{\indent Returns:\ }
d 



 Returns $d$ for a $d$-class homogeneous coherent configuration. }

 

\subsection{\textcolor{Chapter }{AdjacencyMatrices (for IsHomogeneousCoherentConfiguration)}}
\logpage{[ 3, 1, 15 ]}\nobreak
\hyperdef{L}{X795BD63780A058D3}{}
{\noindent\textcolor{FuncColor}{$\triangleright$\enspace\texttt{AdjacencyMatrices({\mdseries\slshape CC})\index{AdjacencyMatrices@\texttt{AdjacencyMatrices}!for IsHomogeneousCoherentConfiguration}
\label{AdjacencyMatrices:for IsHomogeneousCoherentConfiguration}
}\hfill{\scriptsize (attribute)}}\\
\textbf{\indent Returns:\ }
L 



 Returns a list $L$, where the $i$-th entry of $L$ is the adjacency matrix $A_{i-1}$, where $(A_i)_{xy} =1$ if $(x,y) \in R_i$ and $(A_i)_{xy} =0$ otherwise. }

 

\subsection{\textcolor{Chapter }{Order (for IsHomogeneousCoherentConfiguration)}}
\logpage{[ 3, 1, 16 ]}\nobreak
\hyperdef{L}{X7F8EBBAB79CF3102}{}
{\noindent\textcolor{FuncColor}{$\triangleright$\enspace\texttt{Order({\mdseries\slshape CC})\index{Order@\texttt{Order}!for IsHomogeneousCoherentConfiguration}
\label{Order:for IsHomogeneousCoherentConfiguration}
}\hfill{\scriptsize (attribute)}}\\
\textbf{\indent Returns:\ }
n 



 Returns the order $n$ (number of vertices) of the coherent configuration. }

 

\subsection{\textcolor{Chapter }{IntersectionNumber (for IsHomogeneousCoherentConfiguration, IsInt, IsInt, IsInt)}}
\logpage{[ 3, 1, 17 ]}\nobreak
\hyperdef{L}{X81D573D978ED7818}{}
{\noindent\textcolor{FuncColor}{$\triangleright$\enspace\texttt{IntersectionNumber({\mdseries\slshape CC, i, j, k})\index{IntersectionNumber@\texttt{IntersectionNumber}!for IsHomogeneousCoherentConfiguration, IsInt, IsInt, IsInt}
\label{IntersectionNumber:for IsHomogeneousCoherentConfiguration, IsInt, IsInt, IsInt}
}\hfill{\scriptsize (operation)}}\\
\textbf{\indent Returns:\ }
$p_{ij}^k$ 



 Returns the intersection number $p_{ij}^k$ for a coherent configuration CC. }

 

\subsection{\textcolor{Chapter }{Valencies (for IsHomogeneousCoherentConfiguration)}}
\logpage{[ 3, 1, 18 ]}\nobreak
\hyperdef{L}{X86DA7C9780CC7184}{}
{\noindent\textcolor{FuncColor}{$\triangleright$\enspace\texttt{Valencies({\mdseries\slshape CC})\index{Valencies@\texttt{Valencies}!for IsHomogeneousCoherentConfiguration}
\label{Valencies:for IsHomogeneousCoherentConfiguration}
}\hfill{\scriptsize (attribute)}}\\
\textbf{\indent Returns:\ }
L 



 Returns a list L of valencies of a coherent configuration CC. The $i$-th entry of $L$ is $k_{i-1}$. }

 

\subsection{\textcolor{Chapter }{IntersectionAlgebraOfHomogeneousCoherentConfiguration (for IsHomogeneousCoherentConfiguration)}}
\logpage{[ 3, 1, 19 ]}\nobreak
\hyperdef{L}{X7B22C9BD7B67A5AF}{}
{\noindent\textcolor{FuncColor}{$\triangleright$\enspace\texttt{IntersectionAlgebraOfHomogeneousCoherentConfiguration({\mdseries\slshape CC})\index{IntersectionAlgebraOfHomogeneousCoherentConfiguration@\texttt{Intersection}\-\texttt{Algebra}\-\texttt{Of}\-\texttt{Homogeneous}\-\texttt{Coherent}\-\texttt{Configuration}!for IsHomogeneousCoherentConfiguration}
\label{IntersectionAlgebraOfHomogeneousCoherentConfiguration:for IsHomogeneousCoherentConfiguration}
}\hfill{\scriptsize (attribute)}}\\
\textbf{\indent Returns:\ }
L 



 Returns an IntersectionAlgebra object for CC }

 

\subsection{\textcolor{Chapter }{IntersectionMatrices (for IsHomogeneousCoherentConfiguration)}}
\logpage{[ 3, 1, 20 ]}\nobreak
\hyperdef{L}{X81A168787D0FA745}{}
{\noindent\textcolor{FuncColor}{$\triangleright$\enspace\texttt{IntersectionMatrices({\mdseries\slshape CC})\index{IntersectionMatrices@\texttt{IntersectionMatrices}!for IsHomogeneousCoherentConfiguration}
\label{IntersectionMatrices:for IsHomogeneousCoherentConfiguration}
}\hfill{\scriptsize (attribute)}}\\
\textbf{\indent Returns:\ }
L 



 Returns a list L of the intersection matrices of a homogeneous coherent
configuration $CC$, where the $i$-th entry of $L$ is $B_{i-1}$ and $(B_{i})_{jk} = p_{ij}^k$. }

 

\subsection{\textcolor{Chapter }{NumberOfCharacters (for IsHomogeneousCoherentConfiguration)}}
\logpage{[ 3, 1, 21 ]}\nobreak
\hyperdef{L}{X8137DD257F503890}{}
{\noindent\textcolor{FuncColor}{$\triangleright$\enspace\texttt{NumberOfCharacters({\mdseries\slshape CC})\index{NumberOfCharacters@\texttt{NumberOfCharacters}!for IsHomogeneousCoherentConfiguration}
\label{NumberOfCharacters:for IsHomogeneousCoherentConfiguration}
}\hfill{\scriptsize (attribute)}}\\
\textbf{\indent Returns:\ }
n 



 Returns the number $n$ of characters of CC. }

 

\subsection{\textcolor{Chapter }{SplittingField (for IsHomogeneousCoherentConfiguration)}}
\logpage{[ 3, 1, 22 ]}\nobreak
\hyperdef{L}{X829FF25680025270}{}
{\noindent\textcolor{FuncColor}{$\triangleright$\enspace\texttt{SplittingField({\mdseries\slshape CC})\index{SplittingField@\texttt{SplittingField}!for IsHomogeneousCoherentConfiguration}
\label{SplittingField:for IsHomogeneousCoherentConfiguration}
}\hfill{\scriptsize (attribute)}}\\
\textbf{\indent Returns:\ }
F 



 Returns the splitting field of the CC }

 

\subsection{\textcolor{Chapter }{HasRationalSplittingField (for IsHomogeneousCoherentConfiguration)}}
\logpage{[ 3, 1, 23 ]}\nobreak
\hyperdef{L}{X8628587081EA7495}{}
{\noindent\textcolor{FuncColor}{$\triangleright$\enspace\texttt{HasRationalSplittingField({\mdseries\slshape CC})\index{HasRationalSplittingField@\texttt{HasRationalSplittingField}!for IsHomogeneousCoherentConfiguration}
\label{HasRationalSplittingField:for IsHomogeneousCoherentConfiguration}
}\hfill{\scriptsize (property)}}\\
\textbf{\indent Returns:\ }
true or false 



 Returns true if the splitting field is the rationals, false otherwise. }

 

\subsection{\textcolor{Chapter }{KreinParameter (for IsHomogeneousCoherentConfiguration, IsInt, IsInt, IsInt)}}
\logpage{[ 3, 1, 24 ]}\nobreak
\hyperdef{L}{X783D72627DC6C094}{}
{\noindent\textcolor{FuncColor}{$\triangleright$\enspace\texttt{KreinParameter({\mdseries\slshape CC, i, j, k})\index{KreinParameter@\texttt{KreinParameter}!for IsHomogeneousCoherentConfiguration, IsInt, IsInt, IsInt}
\label{KreinParameter:for IsHomogeneousCoherentConfiguration, IsInt, IsInt, IsInt}
}\hfill{\scriptsize (operation)}}\\
\textbf{\indent Returns:\ }
$q_{i,j}^k$ 



 Compute the krein parameter $q_{i,j}^k$ of a commutative homogeneous coherent configuration. }

 

\subsection{\textcolor{Chapter }{KreinParameters (for IsHomogeneousCoherentConfiguration)}}
\logpage{[ 3, 1, 25 ]}\nobreak
\hyperdef{L}{X7F90E9B58624A11B}{}
{\noindent\textcolor{FuncColor}{$\triangleright$\enspace\texttt{KreinParameters({\mdseries\slshape CC})\index{KreinParameters@\texttt{KreinParameters}!for IsHomogeneousCoherentConfiguration}
\label{KreinParameters:for IsHomogeneousCoherentConfiguration}
}\hfill{\scriptsize (attribute)}}\\
\textbf{\indent Returns:\ }
L 



 Return a list $L$ of all Krein parameters of a commutative homogeneous coherent configuration,
where $L[i][j,k] = q_{i,j}^k$. }

 

\subsection{\textcolor{Chapter }{SaveHomogeneousCoherentConfigurationWithCertainAttributes (for IsString, IsHomogeneousCoherentConfiguration, IsList)}}
\logpage{[ 3, 1, 26 ]}\nobreak
\hyperdef{L}{X79D2FE8C7AC7C2A6}{}
{\noindent\textcolor{FuncColor}{$\triangleright$\enspace\texttt{SaveHomogeneousCoherentConfigurationWithCertainAttributes({\mdseries\slshape file, A, L})\index{SaveHomogeneousCoherentConfigurationWithCertainAttributes@\texttt{Save}\-\texttt{Homogeneous}\-\texttt{Coherent}\-\texttt{Configuration}\-\texttt{With}\-\texttt{Certain}\-\texttt{Attributes}!for IsString, IsHomogeneousCoherentConfiguration, IsList}
\label{SaveHomogeneousCoherentConfigurationWithCertainAttributes:for IsString, IsHomogeneousCoherentConfiguration, IsList}
}\hfill{\scriptsize (operation)}}\\
\textbf{\indent Returns:\ }
true 



 Saves homogeneous coherent configuration A to file F with the attributes
listed in L. Note that L must be a list of strings, where each entry is an
attribute known for A. Note that Print or PrintTo will only return the
relation matrix of a homogeneous coherent configuration, which contains all
necessary information about the homogeneous coherent configuration, but may
require a lot of computation to reobtain its attributes. Hence this method is
intended to alow saving of computationally difficult or time consuming
attributes directly. It also alows the user to choose which attributes to
save, since some attributes are very large, but easily recomputed. For
example, it is often desirable to save the matrix of eigenvalues, and perhaps
the automorphism group and intersection matrices, while it is not generally
desirable to also save the adjacency matrices or minimal idempotents. }

 

\subsection{\textcolor{Chapter }{ReadHomogeneousCoherentConfigurationWithCertainAttributes (for IsString)}}
\logpage{[ 3, 1, 27 ]}\nobreak
\hyperdef{L}{X7F1CC5F2874495B4}{}
{\noindent\textcolor{FuncColor}{$\triangleright$\enspace\texttt{ReadHomogeneousCoherentConfigurationWithCertainAttributes({\mdseries\slshape file, A, L})\index{ReadHomogeneousCoherentConfigurationWithCertainAttributes@\texttt{Read}\-\texttt{Homogeneous}\-\texttt{Coherent}\-\texttt{Configuration}\-\texttt{With}\-\texttt{Certain}\-\texttt{Attributes}!for IsString}
\label{ReadHomogeneousCoherentConfigurationWithCertainAttributes:for IsString}
}\hfill{\scriptsize (operation)}}\\
\textbf{\indent Returns:\ }
homogeneous coherent configuration 



 Reads in a homogenous coherent configuration from file and sets it to have the
attributes stored in the file. This reads files of the type formed by
SaveHomogeneousCoherentConfigurationWithCertainAttributes. }

 

\subsection{\textcolor{Chapter }{IsThin (for IsHomogeneousCoherentConfiguration)}}
\logpage{[ 3, 1, 28 ]}\nobreak
\hyperdef{L}{X7B3A1FAB86F88033}{}
{\noindent\textcolor{FuncColor}{$\triangleright$\enspace\texttt{IsThin({\mdseries\slshape CC})\index{IsThin@\texttt{IsThin}!for IsHomogeneousCoherentConfiguration}
\label{IsThin:for IsHomogeneousCoherentConfiguration}
}\hfill{\scriptsize (property)}}\\
\textbf{\indent Returns:\ }
true or false 



 Check if the homogeneous coherent configuration is thin. }

 

\subsection{\textcolor{Chapter }{IsQuasiThin (for IsHomogeneousCoherentConfiguration)}}
\logpage{[ 3, 1, 29 ]}\nobreak
\hyperdef{L}{X7EE5E0897CD88C21}{}
{\noindent\textcolor{FuncColor}{$\triangleright$\enspace\texttt{IsQuasiThin({\mdseries\slshape CC})\index{IsQuasiThin@\texttt{IsQuasiThin}!for IsHomogeneousCoherentConfiguration}
\label{IsQuasiThin:for IsHomogeneousCoherentConfiguration}
}\hfill{\scriptsize (property)}}\\
\textbf{\indent Returns:\ }
true or false 



 Check if the homogeneous coherent configuration is quasi thin. }

 

\subsection{\textcolor{Chapter }{IsPrimitive (for IsHomogeneousCoherentConfiguration)}}
\logpage{[ 3, 1, 30 ]}\nobreak
\hyperdef{L}{X87769E5C7BACEF1D}{}
{\noindent\textcolor{FuncColor}{$\triangleright$\enspace\texttt{IsPrimitive({\mdseries\slshape CC})\index{IsPrimitive@\texttt{IsPrimitive}!for IsHomogeneousCoherentConfiguration}
\label{IsPrimitive:for IsHomogeneousCoherentConfiguration}
}\hfill{\scriptsize (property)}}\\
\textbf{\indent Returns:\ }
true or false 



 Check if the homogeneous coherent configuration is primitve. }

 

\subsection{\textcolor{Chapter }{ReorderMinimalIdempotents (for IsHomogeneousCoherentConfiguration, IsList)}}
\logpage{[ 3, 1, 31 ]}\nobreak
\hyperdef{L}{X7F853FE983D32745}{}
{\noindent\textcolor{FuncColor}{$\triangleright$\enspace\texttt{ReorderMinimalIdempotents({\mdseries\slshape CC, L})\index{ReorderMinimalIdempotents@\texttt{ReorderMinimalIdempotents}!for IsHomogeneousCoherentConfiguration, IsList}
\label{ReorderMinimalIdempotents:for IsHomogeneousCoherentConfiguration, IsList}
}\hfill{\scriptsize (operation)}}\\
\textbf{\indent Returns:\ }
coherent configuration 



 Takes a homogeneous coherent configuration CC and a list L, where L is a
reordering of the minimal idempotents. Returns a homogeneous coherent
configuration where the $i$-th idempotent of the CC becomes the $j$-th idempotent in the new homogeneous coherent configuration, where $j = L_i$. Note that $L_i$ must be equal to $\{0, \ldots, d \}$ as a set, and additionally requires that $L_1 = 0$. }

 

\subsection{\textcolor{Chapter }{ViewRelationDistributionDiagram (for IsHomogeneousCoherentConfiguration)}}
\logpage{[ 3, 1, 32 ]}\nobreak
\hyperdef{L}{X874EB8097E4B31C9}{}
{\noindent\textcolor{FuncColor}{$\triangleright$\enspace\texttt{ViewRelationDistributionDiagram({\mdseries\slshape CC})\index{ViewRelationDistributionDiagram@\texttt{ViewRelationDistributionDiagram}!for IsHomogeneousCoherentConfiguration}
\label{ViewRelationDistributionDiagram:for IsHomogeneousCoherentConfiguration}
}\hfill{\scriptsize (operation)}}\\
\textbf{\indent Returns:\ }
true (Displays relation diistribution diagram) 



 Take a CC and display the relation-distribution diagram with respect to $R_1$. }

 

\subsection{\textcolor{Chapter }{Description (for IsHomogeneousCoherentConfiguration)}}
\logpage{[ 3, 1, 33 ]}\nobreak
\hyperdef{L}{X86D99EDC7B479A6E}{}
{\noindent\textcolor{FuncColor}{$\triangleright$\enspace\texttt{Description({\mdseries\slshape CC})\index{Description@\texttt{Description}!for IsHomogeneousCoherentConfiguration}
\label{Description:for IsHomogeneousCoherentConfiguration}
}\hfill{\scriptsize (attribute)}}\\
\textbf{\indent Returns:\ }
lis 



 Take a CC and returns a list containing various descriptions/names of the CC,
if available. Note that most homogeneous coherent configurations will not have
a description. Some famous homogeneous coherent configurations, association
schemes, or distance regular graphs in the library, as well as families that
have constructor methods, will have names. Some will have multiple
descriptions, hence they are given as a list. You can check if a homogeneous
coherent configuration has assigned descriptions with HasDescription, or set
one with SetDescription. }

 }

 
\section{\textcolor{Chapter }{Constructor methods}}\label{Chapter_Homogeneous_Coherent_Configuration_objects_Section_Constructor_methods}
\logpage{[ 3, 2, 0 ]}
\hyperdef{L}{X820CD05F85142F0A}{}
{
  

\subsection{\textcolor{Chapter }{DistanceRegularGraphScheme (for IsMatrix)}}
\logpage{[ 3, 2, 1 ]}\nobreak
\hyperdef{L}{X8644D8D47A767BAB}{}
{\noindent\textcolor{FuncColor}{$\triangleright$\enspace\texttt{DistanceRegularGraphScheme({\mdseries\slshape A})\index{DistanceRegularGraphScheme@\texttt{DistanceRegularGraphScheme}!for IsMatrix}
\label{DistanceRegularGraphScheme:for IsMatrix}
}\hfill{\scriptsize (operation)}}\\
\textbf{\indent Returns:\ }
homogeneous coherent configuration 



 Constructs an association scheme from the adjacency matrix A of a distance
regular graph. }

 

\subsection{\textcolor{Chapter }{DistanceRegularGraphSchemeNC (for IsMatrix)}}
\logpage{[ 3, 2, 2 ]}\nobreak
\hyperdef{L}{X7CB794B085EF524A}{}
{\noindent\textcolor{FuncColor}{$\triangleright$\enspace\texttt{DistanceRegularGraphSchemeNC({\mdseries\slshape A})\index{DistanceRegularGraphSchemeNC@\texttt{DistanceRegularGraphSchemeNC}!for IsMatrix}
\label{DistanceRegularGraphSchemeNC:for IsMatrix}
}\hfill{\scriptsize (operation)}}\\
\textbf{\indent Returns:\ }
homogeneous coherent configuration 



 Same as DistanceRegularGraphScheme but without checking that a valid
association scheme arises. }

 

\subsection{\textcolor{Chapter }{StronglyRegularGraphScheme (for IsMatrix)}}
\logpage{[ 3, 2, 3 ]}\nobreak
\hyperdef{L}{X7B53D816825017B5}{}
{\noindent\textcolor{FuncColor}{$\triangleright$\enspace\texttt{StronglyRegularGraphScheme({\mdseries\slshape A})\index{StronglyRegularGraphScheme@\texttt{StronglyRegularGraphScheme}!for IsMatrix}
\label{StronglyRegularGraphScheme:for IsMatrix}
}\hfill{\scriptsize (operation)}}\\
\textbf{\indent Returns:\ }
homogeneous coherent configuration 



 Constructs an association scheme from the adjacency matrix A of a strongly
regular graph. }

 

\subsection{\textcolor{Chapter }{StronglyRegularGraphSchemeNC (for IsMatrix)}}
\logpage{[ 3, 2, 4 ]}\nobreak
\hyperdef{L}{X8180A6437DC93E54}{}
{\noindent\textcolor{FuncColor}{$\triangleright$\enspace\texttt{StronglyRegularGraphSchemeNC({\mdseries\slshape A})\index{StronglyRegularGraphSchemeNC@\texttt{StronglyRegularGraphSchemeNC}!for IsMatrix}
\label{StronglyRegularGraphSchemeNC:for IsMatrix}
}\hfill{\scriptsize (operation)}}\\
\textbf{\indent Returns:\ }
homogeneous coherent configuration 



 Same as StronglyRegularGraphScheme but without checking that a valid
association scheme arises. }

 

\subsection{\textcolor{Chapter }{BilinearFormsScheme (for IsField, IsPosInt, IsPosInt)}}
\logpage{[ 3, 2, 5 ]}\nobreak
\hyperdef{L}{X83DA0DF97CE4EEBB}{}
{\noindent\textcolor{FuncColor}{$\triangleright$\enspace\texttt{BilinearFormsScheme({\mdseries\slshape F, n, k})\index{BilinearFormsScheme@\texttt{BilinearFormsScheme}!for IsField, IsPosInt, IsPosInt}
\label{BilinearFormsScheme:for IsField, IsPosInt, IsPosInt}
}\hfill{\scriptsize (operation)}}\\
\textbf{\indent Returns:\ }
homogeneous coherent configuration 



 Returns the bilinear forms scheme for the finite field $F$ with a bilinear form from $F^n \times F^n$ to $F^k$. }

 

\subsection{\textcolor{Chapter }{CyclotomicScheme (for IsPosInt, IsPosInt)}}
\logpage{[ 3, 2, 6 ]}\nobreak
\hyperdef{L}{X7BFC0DE884896A25}{}
{\noindent\textcolor{FuncColor}{$\triangleright$\enspace\texttt{CyclotomicScheme({\mdseries\slshape q, d})\index{CyclotomicScheme@\texttt{CyclotomicScheme}!for IsPosInt, IsPosInt}
\label{CyclotomicScheme:for IsPosInt, IsPosInt}
}\hfill{\scriptsize (operation)}}\\
\textbf{\indent Returns:\ }
homogeneous coherent configuration 



 Let $q$ be a prime power, and $d$ a divisor of $q-1$. Returns the cyclotomic scheme $Cyc(q,d)$. }

 

\subsection{\textcolor{Chapter }{GrassmannScheme (for IsPosInt, IsPosInt, IsPosInt)}}
\logpage{[ 3, 2, 7 ]}\nobreak
\hyperdef{L}{X7C6A577980AFC545}{}
{\noindent\textcolor{FuncColor}{$\triangleright$\enspace\texttt{GrassmannScheme({\mdseries\slshape n, k, q})\index{GrassmannScheme@\texttt{GrassmannScheme}!for IsPosInt, IsPosInt, IsPosInt}
\label{GrassmannScheme:for IsPosInt, IsPosInt, IsPosInt}
}\hfill{\scriptsize (operation)}}\\
\textbf{\indent Returns:\ }
homogeneous coherent configuration 



 Returns the Grassmann scheme, $J_q(n, k)$. }

 

\subsection{\textcolor{Chapter }{GroupCoherentConfiguration (for IsGroup)}}
\logpage{[ 3, 2, 8 ]}\nobreak
\hyperdef{L}{X86AE1E9C7B3A87A3}{}
{\noindent\textcolor{FuncColor}{$\triangleright$\enspace\texttt{GroupCoherentConfiguration({\mdseries\slshape G})\index{GroupCoherentConfiguration@\texttt{GroupCoherentConfiguration}!for IsGroup}
\label{GroupCoherentConfiguration:for IsGroup}
}\hfill{\scriptsize (operation)}}\\
\textbf{\indent Returns:\ }
homogeneous coherent configuration 



 Returns the coherent configuration on the conjugacy classes of a group $G$. }

 

\subsection{\textcolor{Chapter }{HammingScheme (for IsPosInt, IsPosInt)}}
\logpage{[ 3, 2, 9 ]}\nobreak
\hyperdef{L}{X84E428FD7E53D329}{}
{\noindent\textcolor{FuncColor}{$\triangleright$\enspace\texttt{HammingScheme({\mdseries\slshape n, q})\index{HammingScheme@\texttt{HammingScheme}!for IsPosInt, IsPosInt}
\label{HammingScheme:for IsPosInt, IsPosInt}
}\hfill{\scriptsize (operation)}}\\
\textbf{\indent Returns:\ }
homogeneous coherent configuration 



 Returns the Hamming scheme, $H(n, q)$. }

 

\subsection{\textcolor{Chapter }{JohnsonScheme (for IsPosInt, IsPosInt)}}
\logpage{[ 3, 2, 10 ]}\nobreak
\hyperdef{L}{X87F73F177F55A0A6}{}
{\noindent\textcolor{FuncColor}{$\triangleright$\enspace\texttt{JohnsonScheme({\mdseries\slshape n, k})\index{JohnsonScheme@\texttt{JohnsonScheme}!for IsPosInt, IsPosInt}
\label{JohnsonScheme:for IsPosInt, IsPosInt}
}\hfill{\scriptsize (operation)}}\\
\textbf{\indent Returns:\ }
homogeneous coherent configuration 



 Returns the Johnson scheme, $J(n, k)$. }

 

\subsection{\textcolor{Chapter }{DirectProductOfHomogeneousCoherentConfigurations (for IsHomogeneousCoherentConfiguration, IsHomogeneousCoherentConfiguration)}}
\logpage{[ 3, 2, 11 ]}\nobreak
\hyperdef{L}{X82C27C987D42E738}{}
{\noindent\textcolor{FuncColor}{$\triangleright$\enspace\texttt{DirectProductOfHomogeneousCoherentConfigurations({\mdseries\slshape CC1, CC2})\index{DirectProductOfHomogeneousCoherentConfigurations@\texttt{Direct}\-\texttt{Product}\-\texttt{Of}\-\texttt{Homogeneous}\-\texttt{Coherent}\-\texttt{Configurations}!for IsHomogeneousCoherentConfiguration, IsHomogeneousCoherentConfiguration}
\label{DirectProductOfHomogeneousCoherentConfigurations:for IsHomogeneousCoherentConfiguration, IsHomogeneousCoherentConfiguration}
}\hfill{\scriptsize (operation)}}\\
\textbf{\indent Returns:\ }
homogeneous coherent configuration 



 Takes two homogeneous coherent configurations CC1 and CC2 and returns their
direct product. }

 

\subsection{\textcolor{Chapter }{WreathProductOfHomogeneousCoherentConfigurations (for IsHomogeneousCoherentConfiguration, IsHomogeneousCoherentConfiguration)}}
\logpage{[ 3, 2, 12 ]}\nobreak
\hyperdef{L}{X78E3E09A7C979521}{}
{\noindent\textcolor{FuncColor}{$\triangleright$\enspace\texttt{WreathProductOfHomogeneousCoherentConfigurations({\mdseries\slshape CC1, CC2})\index{WreathProductOfHomogeneousCoherentConfigurations@\texttt{Wreath}\-\texttt{Product}\-\texttt{Of}\-\texttt{Homogeneous}\-\texttt{Coherent}\-\texttt{Configurations}!for IsHomogeneousCoherentConfiguration, IsHomogeneousCoherentConfiguration}
\label{WreathProductOfHomogeneousCoherentConfigurations:for IsHomogeneousCoherentConfiguration, IsHomogeneousCoherentConfiguration}
}\hfill{\scriptsize (operation)}}\\
\textbf{\indent Returns:\ }
homogeneous coherent configuration 



 Takes two homogeneous coherent configurations CC1 and CC2 and returns their
wreath product. }

 

\subsection{\textcolor{Chapter }{BipartiteDoubleOfAssociationScheme (for IsHomogeneousCoherentConfiguration)}}
\logpage{[ 3, 2, 13 ]}\nobreak
\hyperdef{L}{X854AA5FC7C353609}{}
{\noindent\textcolor{FuncColor}{$\triangleright$\enspace\texttt{BipartiteDoubleOfAssociationScheme({\mdseries\slshape A})\index{BipartiteDoubleOfAssociationScheme@\texttt{BipartiteDoubleOfAssociationScheme}!for IsHomogeneousCoherentConfiguration}
\label{BipartiteDoubleOfAssociationScheme:for IsHomogeneousCoherentConfiguration}
}\hfill{\scriptsize (operation)}}\\
\textbf{\indent Returns:\ }
Association scheme 



 Returns the bipartite double of an association scheme. }

 

\subsection{\textcolor{Chapter }{ExtendedQBipartiteDouble (for IsHomogeneousCoherentConfiguration)}}
\logpage{[ 3, 2, 14 ]}\nobreak
\hyperdef{L}{X8534326B82EE9891}{}
{\noindent\textcolor{FuncColor}{$\triangleright$\enspace\texttt{ExtendedQBipartiteDouble({\mdseries\slshape A})\index{ExtendedQBipartiteDouble@\texttt{ExtendedQBipartiteDouble}!for IsHomogeneousCoherentConfiguration}
\label{ExtendedQBipartiteDouble:for IsHomogeneousCoherentConfiguration}
}\hfill{\scriptsize (operation)}}\\
\textbf{\indent Returns:\ }
Association scheme 



 Given a cometric association scheme satisfying $b_j^{*} + c_{j+1}^{*} = m +1$ for $0 \leq j \leq d-1 $, returns the extended Q-bipartite double. }

 

\subsection{\textcolor{Chapter }{HomogeneousCoherentConfigurationByOrbitals (for IsPermGroup)}}
\logpage{[ 3, 2, 15 ]}\nobreak
\hyperdef{L}{X85D3B60C8693F466}{}
{\noindent\textcolor{FuncColor}{$\triangleright$\enspace\texttt{HomogeneousCoherentConfigurationByOrbitals({\mdseries\slshape G})\index{HomogeneousCoherentConfigurationByOrbitals@\texttt{Homogeneous}\-\texttt{Coherent}\-\texttt{Configuration}\-\texttt{By}\-\texttt{Orbitals}!for IsPermGroup}
\label{HomogeneousCoherentConfigurationByOrbitals:for IsPermGroup}
}\hfill{\scriptsize (operation)}}\\
\textbf{\indent Returns:\ }
homogeneous coherent configuration 



 Constructs a "group-case" coherent configuration, where the relations are
defined by the orbitals of $G$ on $\{1, \ldots, n\} \times \{1, \ldots, n\}$. $G$ must be a permutation group which is transitive on $\{1, \ldots, n\}$. }

 

\subsection{\textcolor{Chapter }{HomogeneousCoherentConfigurationByOrbitals (for IsGroup, IsGroup)}}
\logpage{[ 3, 2, 16 ]}\nobreak
\hyperdef{L}{X7B20E1AC7CAFCA4D}{}
{\noindent\textcolor{FuncColor}{$\triangleright$\enspace\texttt{HomogeneousCoherentConfigurationByOrbitals({\mdseries\slshape G, H})\index{HomogeneousCoherentConfigurationByOrbitals@\texttt{Homogeneous}\-\texttt{Coherent}\-\texttt{Configuration}\-\texttt{By}\-\texttt{Orbitals}!for IsGroup, IsGroup}
\label{HomogeneousCoherentConfigurationByOrbitals:for IsGroup, IsGroup}
}\hfill{\scriptsize (operation)}}\\
\textbf{\indent Returns:\ }
homogeneous coherent configuration 



 Constructs a "group-case" coherent configuration, where the relations are
defined by the orbitals of $G$ on $G/H$. $G$ is a group, $H$ is a subgroup of $G$, $G/H$ is the set of right cosets of $G$ on $H$, and $G$ must be transitive on $G/H$. }

 

\subsection{\textcolor{Chapter }{SchurianAssociationScheme (for IsPermGroup)}}
\logpage{[ 3, 2, 17 ]}\nobreak
\hyperdef{L}{X806EA85F83549F53}{}
{\noindent\textcolor{FuncColor}{$\triangleright$\enspace\texttt{SchurianAssociationScheme({\mdseries\slshape G})\index{SchurianAssociationScheme@\texttt{SchurianAssociationScheme}!for IsPermGroup}
\label{SchurianAssociationScheme:for IsPermGroup}
}\hfill{\scriptsize (operation)}}\\
\textbf{\indent Returns:\ }
homogeneous coherent configuration 



 Returns the Schurian scheme defined by $G$, where $G$ is a generously transitive permutation group. A Schurian scheme is a special
case of CoherentConfigurationByOrbitals and is symmetric. }

 

\subsection{\textcolor{Chapter }{SchurianCoherentConfiguration (for IsPermGroup)}}
\logpage{[ 3, 2, 18 ]}\nobreak
\hyperdef{L}{X8471FF757ADD27F0}{}
{\noindent\textcolor{FuncColor}{$\triangleright$\enspace\texttt{SchurianCoherentConfiguration({\mdseries\slshape G})\index{SchurianCoherentConfiguration@\texttt{SchurianCoherentConfiguration}!for IsPermGroup}
\label{SchurianCoherentConfiguration:for IsPermGroup}
}\hfill{\scriptsize (operation)}}\\
\textbf{\indent Returns:\ }
homogeneous coherent configuration 



 Alias for HomogeneousCoherentConfigurationByOrbitals }

 }

 
\section{\textcolor{Chapter }{Library}}\label{Chapter_Homogeneous_Coherent_Configuration_objects_Section_Library}
\logpage{[ 3, 3, 0 ]}
\hyperdef{L}{X7970DB3384AF719F}{}
{
  

\subsection{\textcolor{Chapter }{HomogeneousCoherentConfiguration (for IsPosInt, IsPosInt)}}
\logpage{[ 3, 3, 1 ]}\nobreak
\hyperdef{L}{X8610716786531117}{}
{\noindent\textcolor{FuncColor}{$\triangleright$\enspace\texttt{HomogeneousCoherentConfiguration({\mdseries\slshape n, k})\index{HomogeneousCoherentConfiguration@\texttt{HomogeneousCoherentConfiguration}!for IsPosInt, IsPosInt}
\label{HomogeneousCoherentConfiguration:for IsPosInt, IsPosInt}
}\hfill{\scriptsize (operation)}}\\
\textbf{\indent Returns:\ }
homogeneous coherent configuration 



 Returns the $k$-th homogeneous coherent configuration of order $n$. Library is complete for $5 \le n \le 38$ excluding $n \in \{31, 35, 36, 37\}$, corresponding to \cite{Hanaki}. }

 

\subsection{\textcolor{Chapter }{NumberOfHomogeneousCoherentConfigurations (for IsPosInt)}}
\logpage{[ 3, 3, 2 ]}\nobreak
\hyperdef{L}{X83905BC079F11C47}{}
{\noindent\textcolor{FuncColor}{$\triangleright$\enspace\texttt{NumberOfHomogeneousCoherentConfigurations({\mdseries\slshape n})\index{NumberOfHomogeneousCoherentConfigurations@\texttt{Number}\-\texttt{Of}\-\texttt{Homogeneous}\-\texttt{Coherent}\-\texttt{Configurations}!for IsPosInt}
\label{NumberOfHomogeneousCoherentConfigurations:for IsPosInt}
}\hfill{\scriptsize (operation)}}\\
\textbf{\indent Returns:\ }
m 



 Returns the number $m$ of homogeneous coherent configuration of order $n$ contained in the library. }

 

\subsection{\textcolor{Chapter }{AvailableHomogeneousCoherentConfigurations}}
\logpage{[ 3, 3, 3 ]}\nobreak
\hyperdef{L}{X7B72D0D081591C5E}{}
{\noindent\textcolor{FuncColor}{$\triangleright$\enspace\texttt{AvailableHomogeneousCoherentConfigurations({\mdseries\slshape })\index{AvailableHomogeneousCoherentConfigurations@\texttt{Available}\-\texttt{Homogeneous}\-\texttt{Coherent}\-\texttt{Configurations}}
\label{AvailableHomogeneousCoherentConfigurations}
}\hfill{\scriptsize (operation)}}\\
\textbf{\indent Returns:\ }
L 



 Returns a list $L$ of the orders for which the library contains homogeneous coherent
configurations. }

 

\subsection{\textcolor{Chapter }{AllHomogeneousCoherentConfigurations (for IsPosInt)}}
\logpage{[ 3, 3, 4 ]}\nobreak
\hyperdef{L}{X78064175875695D6}{}
{\noindent\textcolor{FuncColor}{$\triangleright$\enspace\texttt{AllHomogeneousCoherentConfigurations({\mdseries\slshape n})\index{AllHomogeneousCoherentConfigurations@\texttt{All}\-\texttt{Homogeneous}\-\texttt{Coherent}\-\texttt{Configurations}!for IsPosInt}
\label{AllHomogeneousCoherentConfigurations:for IsPosInt}
}\hfill{\scriptsize (operation)}}\\
\textbf{\indent Returns:\ }
L 



 Returns a list $L$ of all homogeneous coherent configuration of order $n$. }

 

\subsection{\textcolor{Chapter }{SmallSchemeIdentification (for IsHomogeneousCoherentConfiguration)}}
\logpage{[ 3, 3, 5 ]}\nobreak
\hyperdef{L}{X792738D57B9DE5EB}{}
{\noindent\textcolor{FuncColor}{$\triangleright$\enspace\texttt{SmallSchemeIdentification({\mdseries\slshape CC})\index{SmallSchemeIdentification@\texttt{SmallSchemeIdentification}!for IsHomogeneousCoherentConfiguration}
\label{SmallSchemeIdentification:for IsHomogeneousCoherentConfiguration}
}\hfill{\scriptsize (attribute)}}\\
\textbf{\indent Returns:\ }
id 



 Returns the identification, id, of the homogeneous coherent configuration in
the library which is is isomorphic to CC. Thus
HomogeneousCoherentConfiguration(n, id) will be isomorphic to CC, where n is
the order of CC. }

 }

 
\section{\textcolor{Chapter }{Graphs, automorphisms, and maps}}\label{Chapter_Homogeneous_Coherent_Configuration_objects_Section_Graphs_automorphisms_and_maps}
\logpage{[ 3, 4, 0 ]}
\hyperdef{L}{X7A99E8E67C672076}{}
{
  

\subsection{\textcolor{Chapter }{Digraph (for IsHomogeneousCoherentConfiguration, IsPosInt)}}
\logpage{[ 3, 4, 1 ]}\nobreak
\hyperdef{L}{X860EBD847DBEE0E4}{}
{\noindent\textcolor{FuncColor}{$\triangleright$\enspace\texttt{Digraph({\mdseries\slshape CC, k})\index{Digraph@\texttt{Digraph}!for IsHomogeneousCoherentConfiguration, IsPosInt}
\label{Digraph:for IsHomogeneousCoherentConfiguration, IsPosInt}
}\hfill{\scriptsize (operation)}}\\
\textbf{\indent Returns:\ }
homogeneous coherent configuration 



 Returns the digraph object associated with the k-th relation of a homogeneous
coherent configuration CC. Note that the identity relation is excluded. }

 

\subsection{\textcolor{Chapter }{Digraph (for IsHomogeneousCoherentConfiguration, IsList)}}
\logpage{[ 3, 4, 2 ]}\nobreak
\hyperdef{L}{X7F5760E97D92586F}{}
{\noindent\textcolor{FuncColor}{$\triangleright$\enspace\texttt{Digraph({\mdseries\slshape CC, S})\index{Digraph@\texttt{Digraph}!for IsHomogeneousCoherentConfiguration, IsList}
\label{Digraph:for IsHomogeneousCoherentConfiguration, IsList}
}\hfill{\scriptsize (operation)}}\\
\textbf{\indent Returns:\ }
homogeneous coherent configuration 



 Returns the digraph object which is a union of the relations of a homogeneous
coherent configuration CC given by the set S. Note that the identity relation
is excluded. }

 

\subsection{\textcolor{Chapter }{AutomorphismGroup (for IsHomogeneousCoherentConfiguration)}}
\logpage{[ 3, 4, 3 ]}\nobreak
\hyperdef{L}{X7DC6AA86846CB888}{}
{\noindent\textcolor{FuncColor}{$\triangleright$\enspace\texttt{AutomorphismGroup({\mdseries\slshape CC})\index{AutomorphismGroup@\texttt{AutomorphismGroup}!for IsHomogeneousCoherentConfiguration}
\label{AutomorphismGroup:for IsHomogeneousCoherentConfiguration}
}\hfill{\scriptsize (attribute)}}\\
\textbf{\indent Returns:\ }
G 



 Returns the automorphism group $G$ of the coherent configuration CC. $G$ is a permutation group acting on the index set of the vertices. }

 

\subsection{\textcolor{Chapter }{ImageOfHomogeneousCoherentConfiguration (for IsHomogeneousCoherentConfiguration, IsPerm, IsPerm)}}
\logpage{[ 3, 4, 4 ]}\nobreak
\hyperdef{L}{X8165808287B61ACD}{}
{\noindent\textcolor{FuncColor}{$\triangleright$\enspace\texttt{ImageOfHomogeneousCoherentConfiguration({\mdseries\slshape CC, p, \$\texttt{\symbol{92}}sigma\$})\index{ImageOfHomogeneousCoherentConfiguration@\texttt{Image}\-\texttt{Of}\-\texttt{Homogeneous}\-\texttt{Coherent}\-\texttt{Configuration}!for IsHomogeneousCoherentConfiguration, IsPerm, IsPerm}
\label{ImageOfHomogeneousCoherentConfiguration:for IsHomogeneousCoherentConfiguration, IsPerm, IsPerm}
}\hfill{\scriptsize (operation)}}\\
\textbf{\indent Returns:\ }
true or false 



 For a $d$-class homogeneous coherent configuration with relation matrix $M$ and order $n$, takes a permutation $p$ on the set $\{1 .. n\}$ and permutation $\sigma$ on the set $\{1 .. d\}$ and returns the $d$-class homogenous coherent configuration with relation matrix $P^{-1} M^\sigma P$, where $P$ is the permutation matrix defined by $\sigma$. }

 

\subsection{\textcolor{Chapter }{IsomorphismHomogeneousCoherentConfigurations (for IsHomogeneousCoherentConfiguration, IsHomogeneousCoherentConfiguration)}}
\logpage{[ 3, 4, 5 ]}\nobreak
\hyperdef{L}{X7CC1855982DF9DE4}{}
{\noindent\textcolor{FuncColor}{$\triangleright$\enspace\texttt{IsomorphismHomogeneousCoherentConfigurations({\mdseries\slshape A, B})\index{IsomorphismHomogeneousCoherentConfigurations@\texttt{Isomorphism}\-\texttt{Homogeneous}\-\texttt{Coherent}\-\texttt{Configurations}!for IsHomogeneousCoherentConfiguration, IsHomogeneousCoherentConfiguration}
\label{IsomorphismHomogeneousCoherentConfigurations:for IsHomogeneousCoherentConfiguration, IsHomogeneousCoherentConfiguration}
}\hfill{\scriptsize (operation)}}\\
\textbf{\indent Returns:\ }
true or false 



 If there exists a permutation matrix $P$ and permutation $\sigma$ such that if $M$ is the relation matrix of $A$, then $P^{-1} M^\sigma P$ is the relation matrix of $B$, then the operation will return $[p, \sigma]$, where $p$ is the permutation defining $P$. If no such $P$ and $\sigma$ exist, then the operation will return fail. }

 

\subsection{\textcolor{Chapter }{AreIsomorphicHomogeneousCoherentConfigurations (for IsHomogeneousCoherentConfiguration, IsHomogeneousCoherentConfiguration)}}
\logpage{[ 3, 4, 6 ]}\nobreak
\hyperdef{L}{X8156304D817F8C22}{}
{\noindent\textcolor{FuncColor}{$\triangleright$\enspace\texttt{AreIsomorphicHomogeneousCoherentConfigurations({\mdseries\slshape A, B})\index{AreIsomorphicHomogeneousCoherentConfigurations@\texttt{Are}\-\texttt{Isomorphic}\-\texttt{Homogeneous}\-\texttt{Coherent}\-\texttt{Configurations}!for IsHomogeneousCoherentConfiguration, IsHomogeneousCoherentConfiguration}
\label{AreIsomorphicHomogeneousCoherentConfigurations:for IsHomogeneousCoherentConfiguration, IsHomogeneousCoherentConfiguration}
}\hfill{\scriptsize (operation)}}\\
\textbf{\indent Returns:\ }
true or false 



 If there exists a permutation matrix $P$ and permutation $\sigma$ such that if $M$ is the relation matrix of $A$, then $P^{-1} M^\sigma P$ is the relation matrix of $B$, then the operation will return true. Returns false otherwise. }

 

\subsection{\textcolor{Chapter }{CanonisingMap (for IsHomogeneousCoherentConfiguration)}}
\logpage{[ 3, 4, 7 ]}\nobreak
\hyperdef{L}{X7A6C4BCF85A7027A}{}
{\noindent\textcolor{FuncColor}{$\triangleright$\enspace\texttt{CanonisingMap({\mdseries\slshape CC})\index{CanonisingMap@\texttt{CanonisingMap}!for IsHomogeneousCoherentConfiguration}
\label{CanonisingMap:for IsHomogeneousCoherentConfiguration}
}\hfill{\scriptsize (attribute)}}\\
\textbf{\indent Returns:\ }
[perm1, perm2] 



 Returns two permutations which will produce the canonical form of the
homogeneous coherent configuration CC. The canonical form can be obtained by
ImageOfHomogeneousCoherentConfiguration(CC, perm1, perm2) Any homogenouse
coherent configuration which is isomorphic to CC will the same canonical form. }

 

\subsection{\textcolor{Chapter }{CanonicalFormOfHomogeneousCoherentConfiguration (for IsHomogeneousCoherentConfiguration)}}
\logpage{[ 3, 4, 8 ]}\nobreak
\hyperdef{L}{X85BE27AA7ABB96CD}{}
{\noindent\textcolor{FuncColor}{$\triangleright$\enspace\texttt{CanonicalFormOfHomogeneousCoherentConfiguration({\mdseries\slshape CC})\index{CanonicalFormOfHomogeneousCoherentConfiguration@\texttt{Canonical}\-\texttt{Form}\-\texttt{Of}\-\texttt{Homogeneous}\-\texttt{Coherent}\-\texttt{Configuration}!for IsHomogeneousCoherentConfiguration}
\label{CanonicalFormOfHomogeneousCoherentConfiguration:for IsHomogeneousCoherentConfiguration}
}\hfill{\scriptsize (operation)}}\\
\textbf{\indent Returns:\ }
CC2 



 Returns the canonical form, CC2, of the homogeneous coherent configuration CC.
Any homogenouse coherent configuration which is isomorphic to CC will have CC2
as the canonical form. }

 

\subsection{\textcolor{Chapter }{ConstructorGroup (for IsHomogeneousCoherentConfiguration)}}
\logpage{[ 3, 4, 9 ]}\nobreak
\hyperdef{L}{X792D264C80BA1A3B}{}
{\noindent\textcolor{FuncColor}{$\triangleright$\enspace\texttt{ConstructorGroup({\mdseries\slshape CC})\index{ConstructorGroup@\texttt{ConstructorGroup}!for IsHomogeneousCoherentConfiguration}
\label{ConstructorGroup:for IsHomogeneousCoherentConfiguration}
}\hfill{\scriptsize (attribute)}}\\
\textbf{\indent Returns:\ }
group or false 



 Checks if the coherent configuration was constructed by a group and returns it
if it was, or returns false otherwise. }

 

\subsection{\textcolor{Chapter }{IsGenerouslyTransitive (for IsPermGroup)}}
\logpage{[ 3, 4, 10 ]}\nobreak
\hyperdef{L}{X7E9D8FA485C36A2B}{}
{\noindent\textcolor{FuncColor}{$\triangleright$\enspace\texttt{IsGenerouslyTransitive({\mdseries\slshape G})\index{IsGenerouslyTransitive@\texttt{IsGenerouslyTransitive}!for IsPermGroup}
\label{IsGenerouslyTransitive:for IsPermGroup}
}\hfill{\scriptsize (property)}}\\
\textbf{\indent Returns:\ }
true or false 



 Checks if the permutation group $G$ is generously transitive. }

 

\subsection{\textcolor{Chapter }{IsGenerouslyTransitive (for IsPermGroup, IsList)}}
\logpage{[ 3, 4, 11 ]}\nobreak
\hyperdef{L}{X7DB4FFBC7C0A2173}{}
{\noindent\textcolor{FuncColor}{$\triangleright$\enspace\texttt{IsGenerouslyTransitive({\mdseries\slshape G, L})\index{IsGenerouslyTransitive@\texttt{IsGenerouslyTransitive}!for IsPermGroup, IsList}
\label{IsGenerouslyTransitive:for IsPermGroup, IsList}
}\hfill{\scriptsize (operation)}}\\
\textbf{\indent Returns:\ }
true or false 



 Checks that the permutation group $G$ acts generously transitive on the set $L$. }

 

\subsection{\textcolor{Chapter }{IsSchurian (for IsHomogeneousCoherentConfiguration)}}
\logpage{[ 3, 4, 12 ]}\nobreak
\hyperdef{L}{X8305508584992579}{}
{\noindent\textcolor{FuncColor}{$\triangleright$\enspace\texttt{IsSchurian({\mdseries\slshape CC})\index{IsSchurian@\texttt{IsSchurian}!for IsHomogeneousCoherentConfiguration}
\label{IsSchurian:for IsHomogeneousCoherentConfiguration}
}\hfill{\scriptsize (property)}}\\
\textbf{\indent Returns:\ }
true or false 



 Checks if the input is a Schurian scheme, that is, if the automorphism group
is transitive. }

 }

 
\section{\textcolor{Chapter }{Fusions}}\label{Chapter_Homogeneous_Coherent_Configuration_objects_Section_Fusions}
\logpage{[ 3, 5, 0 ]}
\hyperdef{L}{X791CDD977B9FD97A}{}
{
  

\subsection{\textcolor{Chapter }{IsFusionOfHomogeneousCoherentConfiguration (for IsHomogeneousCoherentConfiguration, IsList)}}
\logpage{[ 3, 5, 1 ]}\nobreak
\hyperdef{L}{X81A5B33780DFAE1D}{}
{\noindent\textcolor{FuncColor}{$\triangleright$\enspace\texttt{IsFusionOfHomogeneousCoherentConfiguration({\mdseries\slshape CC, L})\index{IsFusionOfHomogeneousCoherentConfiguration@\texttt{IsFusion}\-\texttt{Of}\-\texttt{Homogeneous}\-\texttt{Coherent}\-\texttt{Configuration}!for IsHomogeneousCoherentConfiguration, IsList}
\label{IsFusionOfHomogeneousCoherentConfiguration:for IsHomogeneousCoherentConfiguration, IsList}
}\hfill{\scriptsize (operation)}}\\
\textbf{\indent Returns:\ }
true or false 



 Takes a $d$-class homogeneous coherent configuration CC, and checks if the partion L of $\{0, \ldots, d\}$ corresponds to a valid fusion. }

 

\subsection{\textcolor{Chapter }{FusionOfHomogeneousCoherentConfiguration (for IsHomogeneousCoherentConfiguration, IsList)}}
\logpage{[ 3, 5, 2 ]}\nobreak
\hyperdef{L}{X81E2D8678593F6B7}{}
{\noindent\textcolor{FuncColor}{$\triangleright$\enspace\texttt{FusionOfHomogeneousCoherentConfiguration({\mdseries\slshape CC, L})\index{FusionOfHomogeneousCoherentConfiguration@\texttt{Fusion}\-\texttt{Of}\-\texttt{Homogeneous}\-\texttt{Coherent}\-\texttt{Configuration}!for IsHomogeneousCoherentConfiguration, IsList}
\label{FusionOfHomogeneousCoherentConfiguration:for IsHomogeneousCoherentConfiguration, IsList}
}\hfill{\scriptsize (operation)}}\\
\textbf{\indent Returns:\ }
homogeneous coherent configuration 



 Takes a $d$-class homogeneous coherent configuration CC and returns a fusion scheme
corresponding to L, where L is a partion of $\{0, \ldots, d\}$. Note that the ordering of the cells of L is irrelevant. The method will sort
the fused relations according to the smallest value in each cell. }

 

\subsection{\textcolor{Chapter }{ConverseRelationPairs (for IsHomogeneousCoherentConfiguration)}}
\logpage{[ 3, 5, 3 ]}\nobreak
\hyperdef{L}{X843860417966869C}{}
{\noindent\textcolor{FuncColor}{$\triangleright$\enspace\texttt{ConverseRelationPairs({\mdseries\slshape CC})\index{ConverseRelationPairs@\texttt{ConverseRelationPairs}!for IsHomogeneousCoherentConfiguration}
\label{ConverseRelationPairs:for IsHomogeneousCoherentConfiguration}
}\hfill{\scriptsize (attribute)}}\\
\textbf{\indent Returns:\ }
L 



 Returns a list L of either tuples or singletons, corresponding to relations
and their converses or symmetric relations. }

 

\subsection{\textcolor{Chapter }{ConverseRelation (for IsHomogeneousCoherentConfiguration, IsInt)}}
\logpage{[ 3, 5, 4 ]}\nobreak
\hyperdef{L}{X7FA127E67B7399C8}{}
{\noindent\textcolor{FuncColor}{$\triangleright$\enspace\texttt{ConverseRelation({\mdseries\slshape CC, i})\index{ConverseRelation@\texttt{ConverseRelation}!for IsHomogeneousCoherentConfiguration, IsInt}
\label{ConverseRelation:for IsHomogeneousCoherentConfiguration, IsInt}
}\hfill{\scriptsize (operation)}}\\
\textbf{\indent Returns:\ }
j 



 Returns j such that $R_j = R_i^\top$, the converse relation of $i$ . }

 

\subsection{\textcolor{Chapter }{IsStratifiable (for IsHomogeneousCoherentConfiguration)}}
\logpage{[ 3, 5, 5 ]}\nobreak
\hyperdef{L}{X7E2240F0806D0D25}{}
{\noindent\textcolor{FuncColor}{$\triangleright$\enspace\texttt{IsStratifiable({\mdseries\slshape CC})\index{IsStratifiable@\texttt{IsStratifiable}!for IsHomogeneousCoherentConfiguration}
\label{IsStratifiable:for IsHomogeneousCoherentConfiguration}
}\hfill{\scriptsize (attribute)}}\\
\textbf{\indent Returns:\ }
true or false 



 If the fusion of transposed relations produces a valid association scheme,
then CC is stratifiable. }

 

\subsection{\textcolor{Chapter }{SymmetrisationOfHomogeneousCoherentConfiguration (for IsHomogeneousCoherentConfiguration)}}
\logpage{[ 3, 5, 6 ]}\nobreak
\hyperdef{L}{X7BB4E2B27F9E2390}{}
{\noindent\textcolor{FuncColor}{$\triangleright$\enspace\texttt{SymmetrisationOfHomogeneousCoherentConfiguration({\mdseries\slshape CC})\index{SymmetrisationOfHomogeneousCoherentConfiguration@\texttt{Symmetrisation}\-\texttt{Of}\-\texttt{Homogeneous}\-\texttt{Coherent}\-\texttt{Configuration}!for IsHomogeneousCoherentConfiguration}
\label{SymmetrisationOfHomogeneousCoherentConfiguration:for IsHomogeneousCoherentConfiguration}
}\hfill{\scriptsize (operation)}}\\
\textbf{\indent Returns:\ }
Association scheme 



 Given a homogeneous coherent configuration, CC, the symmetrisation is computed
if possible, otherwise fail is returned. The symmetrisation of a homogeneous
coherent configuration takes any non-symmetric relations and fuses them
together. The result may or may not be a valid homogeneous coherent
configuration. If it is valid, then it is an association scheme (Symmetric
coherent configuration). If CC is commutative, then it can be symmetrised. }

 

\subsection{\textcolor{Chapter }{FusingPartitionOfHomogeneousCoherentConfigurations (for IsHomogeneousCoherentConfiguration, IsHomogeneousCoherentConfiguration)}}
\logpage{[ 3, 5, 7 ]}\nobreak
\hyperdef{L}{X79E4F7917BF736B7}{}
{\noindent\textcolor{FuncColor}{$\triangleright$\enspace\texttt{FusingPartitionOfHomogeneousCoherentConfigurations({\mdseries\slshape CC1, CC2})\index{FusingPartitionOfHomogeneousCoherentConfigurations@\texttt{Fusing}\-\texttt{Partition}\-\texttt{Of}\-\texttt{Homogeneous}\-\texttt{Coherent}\-\texttt{Configurations}!for IsHomogeneousCoherentConfiguration, IsHomogeneousCoherentConfiguration}
\label{FusingPartitionOfHomogeneousCoherentConfigurations:for IsHomogeneousCoherentConfiguration, IsHomogeneousCoherentConfiguration}
}\hfill{\scriptsize (operation)}}\\
\textbf{\indent Returns:\ }
partition 



 Takes two homogeneous coherent configurations, CC1 and CC2, where CC1 is $d$-class. If CC2 is equal to a homogeneous coherent configuration formed by
fusing the relations of CC2, this will return the partition of ${0, \ldots, d}$ corresponding to this fusion. If CC2 cannot be produced as a fusion, then
"fail" is returned. This operation does not consider isomorphic homogeneous
coherent configurations - CC2 must be exactly equal to a fusion. }

 

\subsection{\textcolor{Chapter }{FeasibleNonTrivialFusionsOfHomgeneousCoherentConfiguration (for IsHomogeneousCoherentConfiguration)}}
\logpage{[ 3, 5, 8 ]}\nobreak
\hyperdef{L}{X79453D1A81CE4047}{}
{\noindent\textcolor{FuncColor}{$\triangleright$\enspace\texttt{FeasibleNonTrivialFusionsOfHomgeneousCoherentConfiguration({\mdseries\slshape CC[, k[, flag]]})\index{FeasibleNonTrivialFusionsOfHomgeneousCoherentConfiguration@\texttt{Feasible}\-\texttt{Non}\-\texttt{Trivial}\-\texttt{Fusions}\-\texttt{Of}\-\texttt{Homgeneous}\-\texttt{Coherent}\-\texttt{Configuration}!for IsHomogeneousCoherentConfiguration}
\label{FeasibleNonTrivialFusionsOfHomgeneousCoherentConfiguration:for IsHomogeneousCoherentConfiguration}
}\hfill{\scriptsize (attribute)}}\\
\textbf{\indent Returns:\ }
list of feasibly fusionable relations 



 Returns a list where each entry is a collection of relations which may be
fused to form a feasible homogeneous coherent configuration Trivial means
either no relations are fused, or all non-identity relations are fused. If the
additional argument k is given, only fusions with k-classes are returned. If
flag is also given and is equal to true, then all fusions with at most
k-classes are returned. }

 

\subsection{\textcolor{Chapter }{AllNonTrivialFusionsOfHomgeneousCoherentConfiguration (for IsHomogeneousCoherentConfiguration)}}
\logpage{[ 3, 5, 9 ]}\nobreak
\hyperdef{L}{X804E8749792F2CB7}{}
{\noindent\textcolor{FuncColor}{$\triangleright$\enspace\texttt{AllNonTrivialFusionsOfHomgeneousCoherentConfiguration({\mdseries\slshape CC})\index{AllNonTrivialFusionsOfHomgeneousCoherentConfiguration@\texttt{All}\-\texttt{Non}\-\texttt{Trivial}\-\texttt{Fusions}\-\texttt{Of}\-\texttt{Homgeneous}\-\texttt{Coherent}\-\texttt{Configuration}!for IsHomogeneousCoherentConfiguration}
\label{AllNonTrivialFusionsOfHomgeneousCoherentConfiguration:for IsHomogeneousCoherentConfiguration}
}\hfill{\scriptsize (operation)}}\\
\textbf{\indent Returns:\ }
List of all non-trivial fusions of CC 



 Returns a list of all homogeneous coherent configurations such that each
element of the list is a non-trivial fusion of CC. Trivial means either no
relations are fused, or all non-identity relations are fused. If the
additional argument k is given, only fusions with k-classes are returned. If
flag is also given and is equal to true, then all fusions with at most
k-classes are returned. }

 

\subsection{\textcolor{Chapter }{AllFusionsOfHomgeneousCoherentConfiguration (for IsHomogeneousCoherentConfiguration)}}
\logpage{[ 3, 5, 10 ]}\nobreak
\hyperdef{L}{X83C9C5DB7C87500F}{}
{\noindent\textcolor{FuncColor}{$\triangleright$\enspace\texttt{AllFusionsOfHomgeneousCoherentConfiguration({\mdseries\slshape CC})\index{AllFusionsOfHomgeneousCoherentConfiguration@\texttt{All}\-\texttt{Fusions}\-\texttt{Of}\-\texttt{Homgeneous}\-\texttt{Coherent}\-\texttt{Configuration}!for IsHomogeneousCoherentConfiguration}
\label{AllFusionsOfHomgeneousCoherentConfiguration:for IsHomogeneousCoherentConfiguration}
}\hfill{\scriptsize (operation)}}\\
\textbf{\indent Returns:\ }
List of all fusions of CC 



 Returns a list of all homogeneous coherent configurations such that each
element of the list is a fusion of CC. Includes trivial fusions, i.e the
original homogeneous coherent configuration, and the coherent configuration
resulting from the fusion of all non-identity relations }

 

\subsection{\textcolor{Chapter }{FeasibleNonTrivialSymmetricFusionsOfHomgeneousCoherentConfiguration (for IsHomogeneousCoherentConfiguration)}}
\logpage{[ 3, 5, 11 ]}\nobreak
\hyperdef{L}{X8432F3617F33B1BE}{}
{\noindent\textcolor{FuncColor}{$\triangleright$\enspace\texttt{FeasibleNonTrivialSymmetricFusionsOfHomgeneousCoherentConfiguration({\mdseries\slshape CC})\index{FeasibleNonTrivialSymmetricFusionsOfHomgeneousCoherentConfiguration@\texttt{Feasible}\-\texttt{Non}\-\texttt{Trivial}\-\texttt{Symmetric}\-\texttt{Fusions}\-\texttt{Of}\-\texttt{Homgeneous}\-\texttt{Coherent}\-\texttt{Configuration}!for IsHomogeneousCoherentConfiguration}
\label{FeasibleNonTrivialSymmetricFusionsOfHomgeneousCoherentConfiguration:for IsHomogeneousCoherentConfiguration}
}\hfill{\scriptsize (attribute)}}\\
\textbf{\indent Returns:\ }
list of feasibly fusionable relations 



 Returns a list where each entry is a collection of relations which may be
fused to form a feasible association scheme (i.e. relations are symmetric)
Trivial means either no relations are fused, or all non-identity relations are
fused. If the additional argument k is given, only fusions with k-classes are
returned. If flag is also given and is equal to true, then all fusions with at
most k-classes are returned. }

 

\subsection{\textcolor{Chapter }{AllNonTrivialSymmetricFusionsOfHomgeneousCoherentConfiguration (for IsHomogeneousCoherentConfiguration)}}
\logpage{[ 3, 5, 12 ]}\nobreak
\hyperdef{L}{X841E8F567F9D3C18}{}
{\noindent\textcolor{FuncColor}{$\triangleright$\enspace\texttt{AllNonTrivialSymmetricFusionsOfHomgeneousCoherentConfiguration({\mdseries\slshape CC})\index{AllNonTrivialSymmetricFusionsOfHomgeneousCoherentConfiguration@\texttt{All}\-\texttt{Non}\-\texttt{Trivial}\-\texttt{Symmetric}\-\texttt{Fusions}\-\texttt{Of}\-\texttt{Homgeneous}\-\texttt{Coherent}\-\texttt{Configuration}!for IsHomogeneousCoherentConfiguration}
\label{AllNonTrivialSymmetricFusionsOfHomgeneousCoherentConfiguration:for IsHomogeneousCoherentConfiguration}
}\hfill{\scriptsize (operation)}}\\
\textbf{\indent Returns:\ }
List of all non-trivial fusions of CC 



 Returns a list of all association schemes (i.e symmetric relations) such that
each element of the list is a non-trivial fusion of CC. Trivial means either
no relations are fused, or all non-identity relations are fused. If the
additional argument k is given, only fusions with k-classes are returned. If
flag is also given and is equal to true, then all fusions with at most
k-classes are returned. }

 

\subsection{\textcolor{Chapter }{AllSymmetricFusionsOfHomgeneousCoherentConfiguration (for IsHomogeneousCoherentConfiguration)}}
\logpage{[ 3, 5, 13 ]}\nobreak
\hyperdef{L}{X7A63406C84E706E5}{}
{\noindent\textcolor{FuncColor}{$\triangleright$\enspace\texttt{AllSymmetricFusionsOfHomgeneousCoherentConfiguration({\mdseries\slshape CC})\index{AllSymmetricFusionsOfHomgeneousCoherentConfiguration@\texttt{All}\-\texttt{Symmetric}\-\texttt{Fusions}\-\texttt{Of}\-\texttt{Homgeneous}\-\texttt{Coherent}\-\texttt{Configuration}!for IsHomogeneousCoherentConfiguration}
\label{AllSymmetricFusionsOfHomgeneousCoherentConfiguration:for IsHomogeneousCoherentConfiguration}
}\hfill{\scriptsize (operation)}}\\
\textbf{\indent Returns:\ }
List of all fusions of CC 



 Returns a list of all association schemes (i.e symmetric relations) such that
each element of the list is a fusion of CC. Includes trivial fusions, i.e the
original homogeneous coherent configuration (if it is an association scheme),
and the coherent configuration resulting from the fusion of all non-identity
relations }

 

\subsection{\textcolor{Chapter }{IsomorphismToFusionScheme (for IsHomogeneousCoherentConfiguration, IsHomogeneousCoherentConfiguration)}}
\logpage{[ 3, 5, 14 ]}\nobreak
\hyperdef{L}{X87D08BC5870911CF}{}
{\noindent\textcolor{FuncColor}{$\triangleright$\enspace\texttt{IsomorphismToFusionScheme({\mdseries\slshape A, B})\index{IsomorphismToFusionScheme@\texttt{IsomorphismToFusionScheme}!for IsHomogeneousCoherentConfiguration, IsHomogeneousCoherentConfiguration}
\label{IsomorphismToFusionScheme:for IsHomogeneousCoherentConfiguration, IsHomogeneousCoherentConfiguration}
}\hfill{\scriptsize (operation)}}\\
\textbf{\indent Returns:\ }
p1, p2, f 



 If A is isomorphic to a fusion of B, then f is the fusion of $B$, and [p1, p2] is the map which carries $B$ to this fusion. }

 

\subsection{\textcolor{Chapter }{IsIsomorphicToFusionScheme (for IsHomogeneousCoherentConfiguration, IsHomogeneousCoherentConfiguration)}}
\logpage{[ 3, 5, 15 ]}\nobreak
\hyperdef{L}{X819EE3A478CD95F2}{}
{\noindent\textcolor{FuncColor}{$\triangleright$\enspace\texttt{IsIsomorphicToFusionScheme({\mdseries\slshape A, B})\index{IsIsomorphicToFusionScheme@\texttt{IsIsomorphicToFusionScheme}!for IsHomogeneousCoherentConfiguration, IsHomogeneousCoherentConfiguration}
\label{IsIsomorphicToFusionScheme:for IsHomogeneousCoherentConfiguration, IsHomogeneousCoherentConfiguration}
}\hfill{\scriptsize (operation)}}\\
\textbf{\indent Returns:\ }
p1, p2, f 



 Returns true if $A$ is isomorphic to a fusion of $B$. Returns false otherwise. }

 }

 
\section{\textcolor{Chapter }{Bose-Mesner algebra}}\label{Chapter_Homogeneous_Coherent_Configuration_objects_Section_Bose-Mesner_algebra}
\logpage{[ 3, 6, 0 ]}
\hyperdef{L}{X7BEA38D2788CD9FC}{}
{
  

\subsection{\textcolor{Chapter }{MapFromAdjacencyMatricesToMinimalIdempotents (for IsHomogeneousCoherentConfiguration)}}
\logpage{[ 3, 6, 1 ]}\nobreak
\hyperdef{L}{X7AD11E4E83F31019}{}
{\noindent\textcolor{FuncColor}{$\triangleright$\enspace\texttt{MapFromAdjacencyMatricesToMinimalIdempotents({\mdseries\slshape CC})\index{MapFromAdjacencyMatricesToMinimalIdempotents@\texttt{Map}\-\texttt{From}\-\texttt{Adjacency}\-\texttt{Matrices}\-\texttt{To}\-\texttt{Minimal}\-\texttt{Idempotents}!for IsHomogeneousCoherentConfiguration}
\label{MapFromAdjacencyMatricesToMinimalIdempotents:for IsHomogeneousCoherentConfiguration}
}\hfill{\scriptsize (attribute)}}\\
\textbf{\indent Returns:\ }
M 



 Takes a homogeneous coherent configuration object, CC, and returns a matrix,
M, which maps the adjacency matrices to the minimal idempotents of the
adjacency algebra. The central idempotent $E_i = \sum_{i=0}^d M_{ij} A_j$. }

 

\subsection{\textcolor{Chapter }{MapFromAdjacencyMatricesToMinimalIdempotentsOverRationals (for IsHomogeneousCoherentConfiguration)}}
\logpage{[ 3, 6, 2 ]}\nobreak
\hyperdef{L}{X7951042A7DC233F1}{}
{\noindent\textcolor{FuncColor}{$\triangleright$\enspace\texttt{MapFromAdjacencyMatricesToMinimalIdempotentsOverRationals({\mdseries\slshape CC})\index{MapFromAdjacencyMatricesToMinimalIdempotentsOverRationals@\texttt{Map}\-\texttt{From}\-\texttt{Adjacency}\-\texttt{Matrices}\-\texttt{To}\-\texttt{Minimal}\-\texttt{Idempotents}\-\texttt{Over}\-\texttt{Rationals}!for IsHomogeneousCoherentConfiguration}
\label{MapFromAdjacencyMatricesToMinimalIdempotentsOverRationals:for IsHomogeneousCoherentConfiguration}
}\hfill{\scriptsize (attribute)}}\\
\textbf{\indent Returns:\ }
M 



 Takes a homogeneous coherent configuration, CC, and returns a matrix, M, which
maps the adjacency matrices to the minimal idempotents of the adjacency
algebra over the rationals. The minimal idempotent $E_i = \sum_{i=0}^d M_{ij} A_j$. }

 

\subsection{\textcolor{Chapter }{MinimalIdempotents (for IsHomogeneousCoherentConfiguration)}}
\logpage{[ 3, 6, 3 ]}\nobreak
\hyperdef{L}{X86E4696378C91275}{}
{\noindent\textcolor{FuncColor}{$\triangleright$\enspace\texttt{MinimalIdempotents({\mdseries\slshape CC})\index{MinimalIdempotents@\texttt{MinimalIdempotents}!for IsHomogeneousCoherentConfiguration}
\label{MinimalIdempotents:for IsHomogeneousCoherentConfiguration}
}\hfill{\scriptsize (attribute)}}\\
\textbf{\indent Returns:\ }
minimal idempotents 



 Returns the minimal idempotents of the homogeneous coherent configuration. }

 

\subsection{\textcolor{Chapter }{MinimalIdempotentsOverRationals (for IsHomogeneousCoherentConfiguration)}}
\logpage{[ 3, 6, 4 ]}\nobreak
\hyperdef{L}{X84BFF34F7F5B4EB9}{}
{\noindent\textcolor{FuncColor}{$\triangleright$\enspace\texttt{MinimalIdempotentsOverRationals({\mdseries\slshape CC})\index{MinimalIdempotentsOverRationals@\texttt{MinimalIdempotentsOverRationals}!for IsHomogeneousCoherentConfiguration}
\label{MinimalIdempotentsOverRationals:for IsHomogeneousCoherentConfiguration}
}\hfill{\scriptsize (attribute)}}\\
\textbf{\indent Returns:\ }
minimal idempotents 



 Returns the minimal idempotents of the homogeneous coherent configuration over
the rationals. }

 

\subsection{\textcolor{Chapter }{MatrixOfEigenvalues (for IsHomogeneousCoherentConfiguration)}}
\logpage{[ 3, 6, 5 ]}\nobreak
\hyperdef{L}{X7D66223978A4AC1C}{}
{\noindent\textcolor{FuncColor}{$\triangleright$\enspace\texttt{MatrixOfEigenvalues({\mdseries\slshape CC})\index{MatrixOfEigenvalues@\texttt{MatrixOfEigenvalues}!for IsHomogeneousCoherentConfiguration}
\label{MatrixOfEigenvalues:for IsHomogeneousCoherentConfiguration}
}\hfill{\scriptsize (attribute)}}\\
\textbf{\indent Returns:\ }
P 



 Returns a the matrix of eigenvalues (or character table), $P$, for a homogeneous coherent configuration CC. }

 

\subsection{\textcolor{Chapter }{MatrixOfDualEigenvalues (for IsHomogeneousCoherentConfiguration)}}
\logpage{[ 3, 6, 6 ]}\nobreak
\hyperdef{L}{X7B5FBF42843130AF}{}
{\noindent\textcolor{FuncColor}{$\triangleright$\enspace\texttt{MatrixOfDualEigenvalues({\mdseries\slshape CC})\index{MatrixOfDualEigenvalues@\texttt{MatrixOfDualEigenvalues}!for IsHomogeneousCoherentConfiguration}
\label{MatrixOfDualEigenvalues:for IsHomogeneousCoherentConfiguration}
}\hfill{\scriptsize (attribute)}}\\
\textbf{\indent Returns:\ }
Q 



 Returns a the matrix of dual eigenvalues, $Q$, for a homogeneous coherent configuration CC. }

 

\subsection{\textcolor{Chapter }{Multiplicities (for IsHomogeneousCoherentConfiguration)}}
\logpage{[ 3, 6, 7 ]}\nobreak
\hyperdef{L}{X844CA82583657047}{}
{\noindent\textcolor{FuncColor}{$\triangleright$\enspace\texttt{Multiplicities({\mdseries\slshape CC})\index{Multiplicities@\texttt{Multiplicities}!for IsHomogeneousCoherentConfiguration}
\label{Multiplicities:for IsHomogeneousCoherentConfiguration}
}\hfill{\scriptsize (attribute)}}\\
\textbf{\indent Returns:\ }
n 



 Returns the multiplicities of characters of CC. }

 

\subsection{\textcolor{Chapter }{CharacterTableOfHomogeneousCoherentConfiguration (for IsHomogeneousCoherentConfiguration)}}
\logpage{[ 3, 6, 8 ]}\nobreak
\hyperdef{L}{X7B3AD5F67E3FDC50}{}
{\noindent\textcolor{FuncColor}{$\triangleright$\enspace\texttt{CharacterTableOfHomogeneousCoherentConfiguration({\mdseries\slshape CC})\index{CharacterTableOfHomogeneousCoherentConfiguration@\texttt{Character}\-\texttt{Table}\-\texttt{Of}\-\texttt{Homogeneous}\-\texttt{Coherent}\-\texttt{Configuration}!for IsHomogeneousCoherentConfiguration}
\label{CharacterTableOfHomogeneousCoherentConfiguration:for IsHomogeneousCoherentConfiguration}
}\hfill{\scriptsize (attribute)}}\\
\textbf{\indent Returns:\ }
P 



 TODO. }

 

\subsection{\textcolor{Chapter }{FitMatrixOfEigenvalues (for IsHomogeneousCoherentConfiguration, IsMatrix)}}
\logpage{[ 3, 6, 9 ]}\nobreak
\hyperdef{L}{X821425FA7923B4C8}{}
{\noindent\textcolor{FuncColor}{$\triangleright$\enspace\texttt{FitMatrixOfEigenvalues({\mdseries\slshape CC, P})\index{FitMatrixOfEigenvalues@\texttt{FitMatrixOfEigenvalues}!for IsHomogeneousCoherentConfiguration, IsMatrix}
\label{FitMatrixOfEigenvalues:for IsHomogeneousCoherentConfiguration, IsMatrix}
}\hfill{\scriptsize (operation)}}\\
\textbf{\indent Returns:\ }
P2 



 Checks if P is the matrix of eigenvalues of homogeneous coherent configuration
CC, upto some reordering of the columns. In such a case, P2, the reordered
matrix is returned. If not, returns fail. }

 

\subsection{\textcolor{Chapter }{CharacterTableOfSchurianHomogeneousCoherentConfiguration (for IsHomogeneousCoherentConfiguration)}}
\logpage{[ 3, 6, 10 ]}\nobreak
\hyperdef{L}{X7BF28D4185AE6B89}{}
{\noindent\textcolor{FuncColor}{$\triangleright$\enspace\texttt{CharacterTableOfSchurianHomogeneousCoherentConfiguration({\mdseries\slshape CC, P})\index{CharacterTableOfSchurianHomogeneousCoherentConfiguration@\texttt{Character}\-\texttt{Table}\-\texttt{Of}\-\texttt{Schurian}\-\texttt{Homogeneous}\-\texttt{Coherent}\-\texttt{Configuration}!for IsHomogeneousCoherentConfiguration}
\label{CharacterTableOfSchurianHomogeneousCoherentConfiguration:for IsHomogeneousCoherentConfiguration}
}\hfill{\scriptsize (operation)}}\\
\textbf{\indent Returns:\ }
P2 



 Computes the character table of a Schurian coherent configuration by using the
group. The ordering of the columns does not respect the ordering of the
coherent configuration, so "FitMatrixOfEigenvalues" must be used. Sometimes
the group theoretic method is much faster and sometimes it is much slower than
the other methods. }

 

\subsection{\textcolor{Chapter }{MatrixOfEigenvaluesOfCyclotomicScheme (for IsPosInt, IsPosInt)}}
\logpage{[ 3, 6, 11 ]}\nobreak
\hyperdef{L}{X823719AA854623FB}{}
{\noindent\textcolor{FuncColor}{$\triangleright$\enspace\texttt{MatrixOfEigenvaluesOfCyclotomicScheme({\mdseries\slshape n, k, q})\index{MatrixOfEigenvaluesOfCyclotomicScheme@\texttt{Matrix}\-\texttt{Of}\-\texttt{Eigenvalues}\-\texttt{Of}\-\texttt{Cyclotomic}\-\texttt{Scheme}!for IsPosInt, IsPosInt}
\label{MatrixOfEigenvaluesOfCyclotomicScheme:for IsPosInt, IsPosInt}
}\hfill{\scriptsize (operation)}}\\
\textbf{\indent Returns:\ }
P 



 Returns the matrix of eigenvalues $P$ of the scheme $Cyc(q,d)$. }

 

\subsection{\textcolor{Chapter }{MatrixOfEigenvaluesOfGrassmannScheme (for IsPosInt, IsPosInt, IsPosInt)}}
\logpage{[ 3, 6, 12 ]}\nobreak
\hyperdef{L}{X81D585B285091A4F}{}
{\noindent\textcolor{FuncColor}{$\triangleright$\enspace\texttt{MatrixOfEigenvaluesOfGrassmannScheme({\mdseries\slshape n, k, q})\index{MatrixOfEigenvaluesOfGrassmannScheme@\texttt{Matrix}\-\texttt{Of}\-\texttt{Eigenvalues}\-\texttt{Of}\-\texttt{Grassmann}\-\texttt{Scheme}!for IsPosInt, IsPosInt, IsPosInt}
\label{MatrixOfEigenvaluesOfGrassmannScheme:for IsPosInt, IsPosInt, IsPosInt}
}\hfill{\scriptsize (operation)}}\\
\textbf{\indent Returns:\ }
P 



 Returns the matrix of eigenvalues $P$ of the Grassmann scheme $J_q(n, k)$. }

 

\subsection{\textcolor{Chapter }{MatrixOfEigenvaluesOfHammingScheme (for IsPosInt, IsPosInt)}}
\logpage{[ 3, 6, 13 ]}\nobreak
\hyperdef{L}{X861E3C8080FE3A0D}{}
{\noindent\textcolor{FuncColor}{$\triangleright$\enspace\texttt{MatrixOfEigenvaluesOfHammingScheme({\mdseries\slshape n, q})\index{MatrixOfEigenvaluesOfHammingScheme@\texttt{MatrixOfEigenvaluesOfHammingScheme}!for IsPosInt, IsPosInt}
\label{MatrixOfEigenvaluesOfHammingScheme:for IsPosInt, IsPosInt}
}\hfill{\scriptsize (operation)}}\\
\textbf{\indent Returns:\ }
P 



 Returns matrix of eigenvalue $P$ for the Hamming scheme, $H(n, q)$. }

 

\subsection{\textcolor{Chapter }{MatrixOfEigenvaluesOfJohnsonScheme (for IsPosInt, IsPosInt)}}
\logpage{[ 3, 6, 14 ]}\nobreak
\hyperdef{L}{X850D2B6A799A9263}{}
{\noindent\textcolor{FuncColor}{$\triangleright$\enspace\texttt{MatrixOfEigenvaluesOfJohnsonScheme({\mdseries\slshape n, k})\index{MatrixOfEigenvaluesOfJohnsonScheme@\texttt{MatrixOfEigenvaluesOfJohnsonScheme}!for IsPosInt, IsPosInt}
\label{MatrixOfEigenvaluesOfJohnsonScheme:for IsPosInt, IsPosInt}
}\hfill{\scriptsize (operation)}}\\
\textbf{\indent Returns:\ }
P 



 Returns the matrix of eigenvalues $P$ of the Johnson scheme $J(n, k)$. }

 }

 
\section{\textcolor{Chapter }{Metric schemes}}\label{Chapter_Homogeneous_Coherent_Configuration_objects_Section_Metric_schemes}
\logpage{[ 3, 7, 0 ]}
\hyperdef{L}{X7EEAC69486AF989B}{}
{
  

\subsection{\textcolor{Chapter }{IsPPolynomial (for IsHomogeneousCoherentConfiguration)}}
\logpage{[ 3, 7, 1 ]}\nobreak
\hyperdef{L}{X789D746E781B671A}{}
{\noindent\textcolor{FuncColor}{$\triangleright$\enspace\texttt{IsPPolynomial({\mdseries\slshape CC})\index{IsPPolynomial@\texttt{IsPPolynomial}!for IsHomogeneousCoherentConfiguration}
\label{IsPPolynomial:for IsHomogeneousCoherentConfiguration}
}\hfill{\scriptsize (property)}}\\
\textbf{\indent Returns:\ }
true or false 



 Returns if the homogeneous coherent configuration CC is P-polynomial. }

 

\subsection{\textcolor{Chapter }{FirstPPolynomialOrdering (for IsHomogeneousCoherentConfiguration)}}
\logpage{[ 3, 7, 2 ]}\nobreak
\hyperdef{L}{X81E16D677C3957C9}{}
{\noindent\textcolor{FuncColor}{$\triangleright$\enspace\texttt{FirstPPolynomialOrdering({\mdseries\slshape CC})\index{FirstPPolynomialOrdering@\texttt{FirstPPolynomialOrdering}!for IsHomogeneousCoherentConfiguration}
\label{FirstPPolynomialOrdering:for IsHomogeneousCoherentConfiguration}
}\hfill{\scriptsize (attribute)}}\\
\textbf{\indent Returns:\ }
P-polynomial ordering or fail 



 Returns the first P-polynomial ordering admitted by the homogeneous coherent
configuration CC, and fail otherwise. }

 

\subsection{\textcolor{Chapter }{AdmitsPPolynomialOrdering (for IsHomogeneousCoherentConfiguration)}}
\logpage{[ 3, 7, 3 ]}\nobreak
\hyperdef{L}{X86B225878491F343}{}
{\noindent\textcolor{FuncColor}{$\triangleright$\enspace\texttt{AdmitsPPolynomialOrdering({\mdseries\slshape CC})\index{AdmitsPPolynomialOrdering@\texttt{AdmitsPPolynomialOrdering}!for IsHomogeneousCoherentConfiguration}
\label{AdmitsPPolynomialOrdering:for IsHomogeneousCoherentConfiguration}
}\hfill{\scriptsize (property)}}\\
\textbf{\indent Returns:\ }
true or false 



 Returns if the homogeneous coherent configuration CC admits a P-polynomial
ordering. }

 

\subsection{\textcolor{Chapter }{IsMetric (for IsHomogeneousCoherentConfiguration)}}
\logpage{[ 3, 7, 4 ]}\nobreak
\hyperdef{L}{X85DD4C3A86B5AC83}{}
{\noindent\textcolor{FuncColor}{$\triangleright$\enspace\texttt{IsMetric({\mdseries\slshape CC})\index{IsMetric@\texttt{IsMetric}!for IsHomogeneousCoherentConfiguration}
\label{IsMetric:for IsHomogeneousCoherentConfiguration}
}\hfill{\scriptsize (operation)}}\\
\textbf{\indent Returns:\ }
true or false 



 Alias for IsPPolynomial. }

 

\subsection{\textcolor{Chapter }{FirstMetricOrdering (for IsHomogeneousCoherentConfiguration)}}
\logpage{[ 3, 7, 5 ]}\nobreak
\hyperdef{L}{X7A5DF9F07CAB66AD}{}
{\noindent\textcolor{FuncColor}{$\triangleright$\enspace\texttt{FirstMetricOrdering({\mdseries\slshape CC})\index{FirstMetricOrdering@\texttt{FirstMetricOrdering}!for IsHomogeneousCoherentConfiguration}
\label{FirstMetricOrdering:for IsHomogeneousCoherentConfiguration}
}\hfill{\scriptsize (operation)}}\\
\textbf{\indent Returns:\ }
metric ordering or fail 



 Alias for FirstPPolynomialOrdering. }

 

\subsection{\textcolor{Chapter }{AdmitsMetricOrdering (for IsHomogeneousCoherentConfiguration)}}
\logpage{[ 3, 7, 6 ]}\nobreak
\hyperdef{L}{X8399CD2783FCA56B}{}
{\noindent\textcolor{FuncColor}{$\triangleright$\enspace\texttt{AdmitsMetricOrdering({\mdseries\slshape CC})\index{AdmitsMetricOrdering@\texttt{AdmitsMetricOrdering}!for IsHomogeneousCoherentConfiguration}
\label{AdmitsMetricOrdering:for IsHomogeneousCoherentConfiguration}
}\hfill{\scriptsize (operation)}}\\
\textbf{\indent Returns:\ }
true or false 



 Alias for AdmitsPPolynomialOrdering. }

 

\subsection{\textcolor{Chapter }{AllPPolynomialOrderings (for IsHomogeneousCoherentConfiguration)}}
\logpage{[ 3, 7, 7 ]}\nobreak
\hyperdef{L}{X7F80E2E3851FEEF8}{}
{\noindent\textcolor{FuncColor}{$\triangleright$\enspace\texttt{AllPPolynomialOrderings({\mdseries\slshape CC})\index{AllPPolynomialOrderings@\texttt{AllPPolynomialOrderings}!for IsHomogeneousCoherentConfiguration}
\label{AllPPolynomialOrderings:for IsHomogeneousCoherentConfiguration}
}\hfill{\scriptsize (attribute)}}\\
\textbf{\indent Returns:\ }
L 



 Calculate the list $L$ of all P-polynomial orderings of a homogeneous coherent configuration. }

 

\subsection{\textcolor{Chapter }{AllMetricOrderings (for IsHomogeneousCoherentConfiguration)}}
\logpage{[ 3, 7, 8 ]}\nobreak
\hyperdef{L}{X810B62B37A2CCBE7}{}
{\noindent\textcolor{FuncColor}{$\triangleright$\enspace\texttt{AllMetricOrderings({\mdseries\slshape CC})\index{AllMetricOrderings@\texttt{AllMetricOrderings}!for IsHomogeneousCoherentConfiguration}
\label{AllMetricOrderings:for IsHomogeneousCoherentConfiguration}
}\hfill{\scriptsize (operation)}}\\
\textbf{\indent Returns:\ }
L 



 Alias for AllPPolynomialOrderings. }

 

\subsection{\textcolor{Chapter }{IsStronglyRegularGraph (for IsHomogeneousCoherentConfiguration)}}
\logpage{[ 3, 7, 9 ]}\nobreak
\hyperdef{L}{X7816543D7994BC51}{}
{\noindent\textcolor{FuncColor}{$\triangleright$\enspace\texttt{IsStronglyRegularGraph({\mdseries\slshape CC})\index{IsStronglyRegularGraph@\texttt{IsStronglyRegularGraph}!for IsHomogeneousCoherentConfiguration}
\label{IsStronglyRegularGraph:for IsHomogeneousCoherentConfiguration}
}\hfill{\scriptsize (property)}}\\
\textbf{\indent Returns:\ }
true or false 



 Check if a coherent configuration is a strongly regular graph (a $2$-class primitive association scheme). }

 

\subsection{\textcolor{Chapter }{IntersectionArray (for IsHomogeneousCoherentConfiguration)}}
\logpage{[ 3, 7, 10 ]}\nobreak
\hyperdef{L}{X79B6D31979ADD8CB}{}
{\noindent\textcolor{FuncColor}{$\triangleright$\enspace\texttt{IntersectionArray({\mdseries\slshape CC})\index{IntersectionArray@\texttt{IntersectionArray}!for IsHomogeneousCoherentConfiguration}
\label{IntersectionArray:for IsHomogeneousCoherentConfiguration}
}\hfill{\scriptsize (attribute)}}\\
\textbf{\indent Returns:\ }
List 



 Returns the intersection array if CC is P-polynomial. }

 

\subsection{\textcolor{Chapter }{ClassicalParameters (for IsHomogeneousCoherentConfiguration)}}
\logpage{[ 3, 7, 11 ]}\nobreak
\hyperdef{L}{X7B51DF2587BF1A24}{}
{\noindent\textcolor{FuncColor}{$\triangleright$\enspace\texttt{ClassicalParameters({\mdseries\slshape CC})\index{ClassicalParameters@\texttt{ClassicalParameters}!for IsHomogeneousCoherentConfiguration}
\label{ClassicalParameters:for IsHomogeneousCoherentConfiguration}
}\hfill{\scriptsize (attribute)}}\\
\textbf{\indent Returns:\ }
[d, b, $\alpha$, $\beta$] 



 Returns the classical parameters if the CC is metric with classical
parameters. }

 

\subsection{\textcolor{Chapter }{StronglyRegularGraphParameters (for IsHomogeneousCoherentConfiguration)}}
\logpage{[ 3, 7, 12 ]}\nobreak
\hyperdef{L}{X79DDD11F7F64FDF3}{}
{\noindent\textcolor{FuncColor}{$\triangleright$\enspace\texttt{StronglyRegularGraphParameters({\mdseries\slshape CC})\index{StronglyRegularGraphParameters@\texttt{StronglyRegularGraphParameters}!for IsHomogeneousCoherentConfiguration}
\label{StronglyRegularGraphParameters:for IsHomogeneousCoherentConfiguration}
}\hfill{\scriptsize (attribute)}}\\
\textbf{\indent Returns:\ }
[d, b, $\alpha$, $\beta$] 



 Returns the parameters $\{n, k; \lambda, \mu \}$ if the CC is an association scheme with 2 classes. }

 

\subsection{\textcolor{Chapter }{IsPBipartite (for IsHomogeneousCoherentConfiguration)}}
\logpage{[ 3, 7, 13 ]}\nobreak
\hyperdef{L}{X8703BDD579030384}{}
{\noindent\textcolor{FuncColor}{$\triangleright$\enspace\texttt{IsPBipartite({\mdseries\slshape CC})\index{IsPBipartite@\texttt{IsPBipartite}!for IsHomogeneousCoherentConfiguration}
\label{IsPBipartite:for IsHomogeneousCoherentConfiguration}
}\hfill{\scriptsize (property)}}\\
\textbf{\indent Returns:\ }
true or false 



 Returns if the homogeneous coherent configuration CC is bipartite. }

 

\subsection{\textcolor{Chapter }{IsPAntipodal (for IsHomogeneousCoherentConfiguration)}}
\logpage{[ 3, 7, 14 ]}\nobreak
\hyperdef{L}{X7998D1728439D45B}{}
{\noindent\textcolor{FuncColor}{$\triangleright$\enspace\texttt{IsPAntipodal({\mdseries\slshape CC})\index{IsPAntipodal@\texttt{IsPAntipodal}!for IsHomogeneousCoherentConfiguration}
\label{IsPAntipodal:for IsHomogeneousCoherentConfiguration}
}\hfill{\scriptsize (property)}}\\
\textbf{\indent Returns:\ }
true or false 



 Returns if the homogeneous coherent configuration CC is antipodal. }

 }

 
\section{\textcolor{Chapter }{Cometric schemes}}\label{Chapter_Homogeneous_Coherent_Configuration_objects_Section_Cometric_schemes}
\logpage{[ 3, 8, 0 ]}
\hyperdef{L}{X81F482747C1E7A48}{}
{
  

\subsection{\textcolor{Chapter }{AdmitsQPolynomialOrdering (for IsHomogeneousCoherentConfiguration)}}
\logpage{[ 3, 8, 1 ]}\nobreak
\hyperdef{L}{X821E557E7EF683F8}{}
{\noindent\textcolor{FuncColor}{$\triangleright$\enspace\texttt{AdmitsQPolynomialOrdering({\mdseries\slshape CC})\index{AdmitsQPolynomialOrdering@\texttt{AdmitsQPolynomialOrdering}!for IsHomogeneousCoherentConfiguration}
\label{AdmitsQPolynomialOrdering:for IsHomogeneousCoherentConfiguration}
}\hfill{\scriptsize (property)}}\\
\textbf{\indent Returns:\ }
true or false 



 Returns if the homogeneous coherent configuration CC admits a Q-polynomial
ordering. }

 

\subsection{\textcolor{Chapter }{AdmitsCometricOrdering (for IsHomogeneousCoherentConfiguration)}}
\logpage{[ 3, 8, 2 ]}\nobreak
\hyperdef{L}{X872EC130803BD308}{}
{\noindent\textcolor{FuncColor}{$\triangleright$\enspace\texttt{AdmitsCometricOrdering({\mdseries\slshape CC})\index{AdmitsCometricOrdering@\texttt{AdmitsCometricOrdering}!for IsHomogeneousCoherentConfiguration}
\label{AdmitsCometricOrdering:for IsHomogeneousCoherentConfiguration}
}\hfill{\scriptsize (operation)}}\\
\textbf{\indent Returns:\ }
true or false 



 Alias for AdmitsQPolynomialOrdering. }

 

\subsection{\textcolor{Chapter }{IsQPolynomial (for IsHomogeneousCoherentConfiguration)}}
\logpage{[ 3, 8, 3 ]}\nobreak
\hyperdef{L}{X80DE65F1780741B9}{}
{\noindent\textcolor{FuncColor}{$\triangleright$\enspace\texttt{IsQPolynomial({\mdseries\slshape CC})\index{IsQPolynomial@\texttt{IsQPolynomial}!for IsHomogeneousCoherentConfiguration}
\label{IsQPolynomial:for IsHomogeneousCoherentConfiguration}
}\hfill{\scriptsize (property)}}\\
\textbf{\indent Returns:\ }
true or false 



 Returns if the commutative coherent configuration CC is Q-polynomial. }

 

\subsection{\textcolor{Chapter }{IsCometric (for IsHomogeneousCoherentConfiguration)}}
\logpage{[ 3, 8, 4 ]}\nobreak
\hyperdef{L}{X7B1853C97D873F75}{}
{\noindent\textcolor{FuncColor}{$\triangleright$\enspace\texttt{IsCometric({\mdseries\slshape CC})\index{IsCometric@\texttt{IsCometric}!for IsHomogeneousCoherentConfiguration}
\label{IsCometric:for IsHomogeneousCoherentConfiguration}
}\hfill{\scriptsize (operation)}}\\
\textbf{\indent Returns:\ }
true or false 



 Alias for is Q-polynomial. }

 

\subsection{\textcolor{Chapter }{AllQPolynomialOrderings (for IsHomogeneousCoherentConfiguration)}}
\logpage{[ 3, 8, 5 ]}\nobreak
\hyperdef{L}{X791422BC7E94288E}{}
{\noindent\textcolor{FuncColor}{$\triangleright$\enspace\texttt{AllQPolynomialOrderings({\mdseries\slshape CC})\index{AllQPolynomialOrderings@\texttt{AllQPolynomialOrderings}!for IsHomogeneousCoherentConfiguration}
\label{AllQPolynomialOrderings:for IsHomogeneousCoherentConfiguration}
}\hfill{\scriptsize (attribute)}}\\
\textbf{\indent Returns:\ }
L 



 Calculate a list $L$ of all Q-polynomial orderings of a homogeneous coherent configuration. }

 

\subsection{\textcolor{Chapter }{AllCometricOrderings (for IsHomogeneousCoherentConfiguration)}}
\logpage{[ 3, 8, 6 ]}\nobreak
\hyperdef{L}{X7F70C9E27836C0C0}{}
{\noindent\textcolor{FuncColor}{$\triangleright$\enspace\texttt{AllCometricOrderings({\mdseries\slshape CC})\index{AllCometricOrderings@\texttt{AllCometricOrderings}!for IsHomogeneousCoherentConfiguration}
\label{AllCometricOrderings:for IsHomogeneousCoherentConfiguration}
}\hfill{\scriptsize (operation)}}\\
\textbf{\indent Returns:\ }
L 



 Alias for AllQPolynomialOrderings. }

 

\subsection{\textcolor{Chapter }{KreinArray (for IsHomogeneousCoherentConfiguration)}}
\logpage{[ 3, 8, 7 ]}\nobreak
\hyperdef{L}{X7C63079983A353A9}{}
{\noindent\textcolor{FuncColor}{$\triangleright$\enspace\texttt{KreinArray({\mdseries\slshape CC})\index{KreinArray@\texttt{KreinArray}!for IsHomogeneousCoherentConfiguration}
\label{KreinArray:for IsHomogeneousCoherentConfiguration}
}\hfill{\scriptsize (attribute)}}\\
\textbf{\indent Returns:\ }
List 



 Returns the Krein (or dual intersection) array if CC is Q-polynomial. }

 

\subsection{\textcolor{Chapter }{DualIntersectionArray (for IsHomogeneousCoherentConfiguration)}}
\logpage{[ 3, 8, 8 ]}\nobreak
\hyperdef{L}{X7EA51F43826622B2}{}
{\noindent\textcolor{FuncColor}{$\triangleright$\enspace\texttt{DualIntersectionArray({\mdseries\slshape CC})\index{DualIntersectionArray@\texttt{DualIntersectionArray}!for IsHomogeneousCoherentConfiguration}
\label{DualIntersectionArray:for IsHomogeneousCoherentConfiguration}
}\hfill{\scriptsize (operation)}}\\
\textbf{\indent Returns:\ }
List 



 Alias for KreinArray. }

 

\subsection{\textcolor{Chapter }{IsQBipartite (for IsHomogeneousCoherentConfiguration)}}
\logpage{[ 3, 8, 9 ]}\nobreak
\hyperdef{L}{X7FF63F24791F2527}{}
{\noindent\textcolor{FuncColor}{$\triangleright$\enspace\texttt{IsQBipartite({\mdseries\slshape CC})\index{IsQBipartite@\texttt{IsQBipartite}!for IsHomogeneousCoherentConfiguration}
\label{IsQBipartite:for IsHomogeneousCoherentConfiguration}
}\hfill{\scriptsize (property)}}\\
\textbf{\indent Returns:\ }
true or false 



 Returns if the homogeneous coherent configuration CC is Q-bipartite. }

 

\subsection{\textcolor{Chapter }{IsQAntipodal (for IsHomogeneousCoherentConfiguration)}}
\logpage{[ 3, 8, 10 ]}\nobreak
\hyperdef{L}{X816D53838425F2F8}{}
{\noindent\textcolor{FuncColor}{$\triangleright$\enspace\texttt{IsQAntipodal({\mdseries\slshape CC})\index{IsQAntipodal@\texttt{IsQAntipodal}!for IsHomogeneousCoherentConfiguration}
\label{IsQAntipodal:for IsHomogeneousCoherentConfiguration}
}\hfill{\scriptsize (property)}}\\
\textbf{\indent Returns:\ }
true or false 



 Returns if the homogeneous coherent configuration CC is Q-antipodal. }

 }

 
\section{\textcolor{Chapter }{Subsets}}\label{Chapter_Homogeneous_Coherent_Configuration_objects_Section_Subsets}
\logpage{[ 3, 9, 0 ]}
\hyperdef{L}{X87BB51CD81E9D34A}{}
{
  

\subsection{\textcolor{Chapter }{CharacteristicVector (for IsList, IsList)}}
\logpage{[ 3, 9, 1 ]}\nobreak
\hyperdef{L}{X807E4FBF7B15CB1D}{}
{\noindent\textcolor{FuncColor}{$\triangleright$\enspace\texttt{CharacteristicVector({\mdseries\slshape Omega, X})\index{CharacteristicVector@\texttt{CharacteristicVector}!for IsList, IsList}
\label{CharacteristicVector:for IsList, IsList}
}\hfill{\scriptsize (operation)}}\\
\textbf{\indent Returns:\ }
$\chi_X$ 



 Takes a subset X of Omega and returns the characteristic vector. The
characteristic vector is a 0,1-vector indexed by the entries of Omega, with a
1 at position x if x is in X, and 0 otherwise. }

 

\subsection{\textcolor{Chapter }{CharacteristicVector (for IsList, IsPosInt)}}
\logpage{[ 3, 9, 2 ]}\nobreak
\hyperdef{L}{X7FAF72567E7AFEA0}{}
{\noindent\textcolor{FuncColor}{$\triangleright$\enspace\texttt{CharacteristicVector({\mdseries\slshape X, n})\index{CharacteristicVector@\texttt{CharacteristicVector}!for IsList, IsPosInt}
\label{CharacteristicVector:for IsList, IsPosInt}
}\hfill{\scriptsize (operation)}}\\
\textbf{\indent Returns:\ }
$\chi_X$ 



 Takes a subset X of of [1 .. n] and returns the characteristic vector $chi_X$. }

 

\subsection{\textcolor{Chapter }{InnerDistribution (for IsHomogeneousCoherentConfiguration, IsList)}}
\logpage{[ 3, 9, 3 ]}\nobreak
\hyperdef{L}{X7B8EAB02816156F1}{}
{\noindent\textcolor{FuncColor}{$\triangleright$\enspace\texttt{InnerDistribution({\mdseries\slshape v, CC})\index{InnerDistribution@\texttt{InnerDistribution}!for IsHomogeneousCoherentConfiguration, IsList}
\label{InnerDistribution:for IsHomogeneousCoherentConfiguration, IsList}
}\hfill{\scriptsize (operation)}}\\
\textbf{\indent Returns:\ }
a 



 Returns the inner distribution $a$ of a vector v with respect to the adjacency matrices of the coherent
configuration CC. Note that v must be a vector over R\texttt{\symbol{94}}n
where n is the order of CC. CC must be commutative. }

 

\subsection{\textcolor{Chapter }{MacWilliamsTransform (for IsHomogeneousCoherentConfiguration, IsList)}}
\logpage{[ 3, 9, 4 ]}\nobreak
\hyperdef{L}{X823DE0DD84783724}{}
{\noindent\textcolor{FuncColor}{$\triangleright$\enspace\texttt{MacWilliamsTransform({\mdseries\slshape v, CC})\index{MacWilliamsTransform@\texttt{MacWilliamsTransform}!for IsHomogeneousCoherentConfiguration, IsList}
\label{MacWilliamsTransform:for IsHomogeneousCoherentConfiguration, IsList}
}\hfill{\scriptsize (operation)}}\\
\textbf{\indent Returns:\ }
aQ 



 Returns the MacWilliams transform $aQ$ of a vector $v$ with respect to a coherent confiiguration CC. Takes either a vector $v$ in $R^n$ and converts it to its inner distribution vector first, or takes the inner
distribution directly. }

 

\subsection{\textcolor{Chapter }{DualBoseMesnerBasis (for IsHomogeneousCoherentConfiguration, IsPosInt)}}
\logpage{[ 3, 9, 5 ]}\nobreak
\hyperdef{L}{X867926D07A427660}{}
{\noindent\textcolor{FuncColor}{$\triangleright$\enspace\texttt{DualBoseMesnerBasis({\mdseries\slshape CC, p})\index{DualBoseMesnerBasis@\texttt{DualBoseMesnerBasis}!for IsHomogeneousCoherentConfiguration, IsPosInt}
\label{DualBoseMesnerBasis:for IsHomogeneousCoherentConfiguration, IsPosInt}
}\hfill{\scriptsize (operation)}}\\
\textbf{\indent Returns:\ }
L 



 Returns a list $L$ with the dual Bose-Mesner basis of a homogeneous coherent configuration with
respect to the point p, such that $L_i = \tilde E_{i-1}$. }

 

\subsection{\textcolor{Chapter }{DualBoseMesnerBasis (for IsHomogeneousCoherentConfiguration)}}
\logpage{[ 3, 9, 6 ]}\nobreak
\hyperdef{L}{X7ADD9C5E7A3279A2}{}
{\noindent\textcolor{FuncColor}{$\triangleright$\enspace\texttt{DualBoseMesnerBasis({\mdseries\slshape CC})\index{DualBoseMesnerBasis@\texttt{DualBoseMesnerBasis}!for IsHomogeneousCoherentConfiguration}
\label{DualBoseMesnerBasis:for IsHomogeneousCoherentConfiguration}
}\hfill{\scriptsize (operation)}}\\
\textbf{\indent Returns:\ }
L 



 Returns a list $L$ with the dual Bose-Mesner basis of a homogeneous coherent configuration with
respect to the first point, such that $L_i = \tilde E_{i-1}$. }

 

\subsection{\textcolor{Chapter }{OuterDistribution (for IsHomogeneousCoherentConfiguration, IsList)}}
\logpage{[ 3, 9, 7 ]}\nobreak
\hyperdef{L}{X79986ED0847F1271}{}
{\noindent\textcolor{FuncColor}{$\triangleright$\enspace\texttt{OuterDistribution({\mdseries\slshape v, CC})\index{OuterDistribution@\texttt{OuterDistribution}!for IsHomogeneousCoherentConfiguration, IsList}
\label{OuterDistribution:for IsHomogeneousCoherentConfiguration, IsList}
}\hfill{\scriptsize (operation)}}\\
\textbf{\indent Returns:\ }
B 



 Returns the outer distribution $B$ of a vector $v$ with respect to the adjacency matrices of the coherent configuration CC. Note
that $v$ must be a vector over $R^n$ where n is the order of CC. CC must be commutative. }

 

\subsection{\textcolor{Chapter }{DelsarteDesignType (for IsHomogeneousCoherentConfiguration, IsList)}}
\logpage{[ 3, 9, 8 ]}\nobreak
\hyperdef{L}{X7ECE798F847B0FF4}{}
{\noindent\textcolor{FuncColor}{$\triangleright$\enspace\texttt{DelsarteDesignType({\mdseries\slshape CC, S})\index{DelsarteDesignType@\texttt{DelsarteDesignType}!for IsHomogeneousCoherentConfiguration, IsList}
\label{DelsarteDesignType:for IsHomogeneousCoherentConfiguration, IsList}
}\hfill{\scriptsize (operation)}}\\
\textbf{\indent Returns:\ }
T 



 Returns T such that S is a Delsarte T-design, that is, $\chi_S E_i =0$ for all $i \in T$. S must be a subset of the vertices. }

 

\subsection{\textcolor{Chapter }{WeightedDelsarteDesignType (for IsHomogeneousCoherentConfiguration, IsList)}}
\logpage{[ 3, 9, 9 ]}\nobreak
\hyperdef{L}{X8633CF8481B95FE7}{}
{\noindent\textcolor{FuncColor}{$\triangleright$\enspace\texttt{WeightedDelsarteDesignType({\mdseries\slshape CC, S})\index{WeightedDelsarteDesignType@\texttt{WeightedDelsarteDesignType}!for IsHomogeneousCoherentConfiguration, IsList}
\label{WeightedDelsarteDesignType:for IsHomogeneousCoherentConfiguration, IsList}
}\hfill{\scriptsize (operation)}}\\
\textbf{\indent Returns:\ }
T 



 Returns T such that v is a (weighted) Delsarte T-design, that is, $v E_i =0$ for all $i \in T$. v must be a weighted characteristic vector with respect to the vertices. }

 

\subsection{\textcolor{Chapter }{IsDelsarteTDesign (for IsHomogeneousCoherentConfiguration, IsList, IsList)}}
\logpage{[ 3, 9, 10 ]}\nobreak
\hyperdef{L}{X7A277D597C52EAA9}{}
{\noindent\textcolor{FuncColor}{$\triangleright$\enspace\texttt{IsDelsarteTDesign({\mdseries\slshape CC, S})\index{IsDelsarteTDesign@\texttt{IsDelsarteTDesign}!for IsHomogeneousCoherentConfiguration, IsList, IsList}
\label{IsDelsarteTDesign:for IsHomogeneousCoherentConfiguration, IsList, IsList}
}\hfill{\scriptsize (operation)}}\\
\textbf{\indent Returns:\ }
true or false 



 Checks that S is a (weighted) Delsarte T-design, that is, $\chi_S E_i =0$ for all $i \in T$. S must be either a subset of the vertices, or a weighted characteristic
vector with respect to the vertices. }

 

\subsection{\textcolor{Chapter }{DualDegreeSet (for IsHomogeneousCoherentConfiguration, IsList)}}
\logpage{[ 3, 9, 11 ]}\nobreak
\hyperdef{L}{X8270DDAF7F79F0E5}{}
{\noindent\textcolor{FuncColor}{$\triangleright$\enspace\texttt{DualDegreeSet({\mdseries\slshape CC, S})\index{DualDegreeSet@\texttt{DualDegreeSet}!for IsHomogeneousCoherentConfiguration, IsList}
\label{DualDegreeSet:for IsHomogeneousCoherentConfiguration, IsList}
}\hfill{\scriptsize (operation)}}\\
\textbf{\indent Returns:\ }
K 



 Returns the dual degree set K for a set S, that is, $\chi_S E_i \neq 0$ for all $i \in K$. S must be a subset of the vertices. }

 

\subsection{\textcolor{Chapter }{WeightedDualDegreeSet (for IsHomogeneousCoherentConfiguration, IsList)}}
\logpage{[ 3, 9, 12 ]}\nobreak
\hyperdef{L}{X7FEC532685C555AE}{}
{\noindent\textcolor{FuncColor}{$\triangleright$\enspace\texttt{WeightedDualDegreeSet({\mdseries\slshape CC, S})\index{WeightedDualDegreeSet@\texttt{WeightedDualDegreeSet}!for IsHomogeneousCoherentConfiguration, IsList}
\label{WeightedDualDegreeSet:for IsHomogeneousCoherentConfiguration, IsList}
}\hfill{\scriptsize (operation)}}\\
\textbf{\indent Returns:\ }
K 



 Returns the dual degree set K for a set S, that is, $v E_i \neq 0$ for all $i \in K$. v must be a (weighted) characteristic vector with restpect to the vertices. }

 

\subsection{\textcolor{Chapter }{AreDesignOrthogonal (for IsHomogeneousCoherentConfiguration, IsList, IsList)}}
\logpage{[ 3, 9, 13 ]}\nobreak
\hyperdef{L}{X7E263036846B1111}{}
{\noindent\textcolor{FuncColor}{$\triangleright$\enspace\texttt{AreDesignOrthogonal({\mdseries\slshape CC, S1, S2})\index{AreDesignOrthogonal@\texttt{AreDesignOrthogonal}!for IsHomogeneousCoherentConfiguration, IsList, IsList}
\label{AreDesignOrthogonal:for IsHomogeneousCoherentConfiguration, IsList, IsList}
}\hfill{\scriptsize (operation)}}\\
\textbf{\indent Returns:\ }
true or false 



 If S1 and S2 are either subsets of vertices, or (weighted) characteristic
vectors, this checks that they are design orthogonal, that is, their dual
degree sets are disjoint. }

 }

 
\section{\textcolor{Chapter }{Approximations}}\label{Chapter_Homogeneous_Coherent_Configuration_objects_Section_Approximations}
\logpage{[ 3, 10, 0 ]}
\hyperdef{L}{X80EB12DF81F8C650}{}
{
  

\subsection{\textcolor{Chapter }{ApproximateRealMinimalIdempotent (for IsHomogeneousCoherentConfiguration, IsInt)}}
\logpage{[ 3, 10, 1 ]}\nobreak
\hyperdef{L}{X7A0BD21D7E275ACD}{}
{\noindent\textcolor{FuncColor}{$\triangleright$\enspace\texttt{ApproximateRealMinimalIdempotent({\mdseries\slshape CC, i})\index{ApproximateRealMinimalIdempotent@\texttt{ApproximateRealMinimalIdempotent}!for IsHomogeneousCoherentConfiguration, IsInt}
\label{ApproximateRealMinimalIdempotent:for IsHomogeneousCoherentConfiguration, IsInt}
}\hfill{\scriptsize (operation)}}\\
\textbf{\indent Returns:\ }
approximation of $E_i$ 



 Returns the $i$-th idempotent with entries approximated by floats. All entries must be real
values. }

 

\subsection{\textcolor{Chapter }{ApproximateRealMinimalIdempotents (for IsHomogeneousCoherentConfiguration)}}
\logpage{[ 3, 10, 2 ]}\nobreak
\hyperdef{L}{X7CC529B279C1AB60}{}
{\noindent\textcolor{FuncColor}{$\triangleright$\enspace\texttt{ApproximateRealMinimalIdempotents({\mdseries\slshape CC})\index{ApproximateRealMinimalIdempotents@\texttt{ApproximateRealMinimalIdempotents}!for IsHomogeneousCoherentConfiguration}
\label{ApproximateRealMinimalIdempotents:for IsHomogeneousCoherentConfiguration}
}\hfill{\scriptsize (operation)}}\\
\textbf{\indent Returns:\ }
approximation of minimal idempotents 



 Returns a list of idempotents with entries approximated by floats. All entries
must be real values. }

 

\subsection{\textcolor{Chapter }{ApproximateRealMatrixOfEigenvalues (for IsHomogeneousCoherentConfiguration)}}
\logpage{[ 3, 10, 3 ]}\nobreak
\hyperdef{L}{X8221482C82DF4E2D}{}
{\noindent\textcolor{FuncColor}{$\triangleright$\enspace\texttt{ApproximateRealMatrixOfEigenvalues({\mdseries\slshape CC})\index{ApproximateRealMatrixOfEigenvalues@\texttt{ApproximateRealMatrixOfEigenvalues}!for IsHomogeneousCoherentConfiguration}
\label{ApproximateRealMatrixOfEigenvalues:for IsHomogeneousCoherentConfiguration}
}\hfill{\scriptsize (operation)}}\\
\textbf{\indent Returns:\ }
approximation of matrix of eigenvalues 



 Returns the matrix of eigenvalues with entries approximated by floats. All
entries must be real values. }

 }

 
\section{\textcolor{Chapter }{Algebras}}\label{Chapter_Homogeneous_Coherent_Configuration_objects_Section_Algebras}
\logpage{[ 3, 11, 0 ]}
\hyperdef{L}{X7DDBF6F47A2E021C}{}
{
  

 

\subsection{\textcolor{Chapter }{BoseMesnerAlgebra (for IsHomogeneousCoherentConfiguration)}}
\logpage{[ 3, 11, 1 ]}\nobreak
\hyperdef{L}{X81D2BD717BA813C0}{}
{\noindent\textcolor{FuncColor}{$\triangleright$\enspace\texttt{BoseMesnerAlgebra({\mdseries\slshape CC})\index{BoseMesnerAlgebra@\texttt{BoseMesnerAlgebra}!for IsHomogeneousCoherentConfiguration}
\label{BoseMesnerAlgebra:for IsHomogeneousCoherentConfiguration}
}\hfill{\scriptsize (operation)}}\\
\textbf{\indent Returns:\ }
A 



 Returns the Bose-Mesner algebra $A$ of a homogeneous coherent configuration. }

 

\subsection{\textcolor{Chapter }{AdjacencyAlgebra (for IsHomogeneousCoherentConfiguration)}}
\logpage{[ 3, 11, 2 ]}\nobreak
\hyperdef{L}{X811233BD7EC4ED5F}{}
{\noindent\textcolor{FuncColor}{$\triangleright$\enspace\texttt{AdjacencyAlgebra({\mdseries\slshape CC})\index{AdjacencyAlgebra@\texttt{AdjacencyAlgebra}!for IsHomogeneousCoherentConfiguration}
\label{AdjacencyAlgebra:for IsHomogeneousCoherentConfiguration}
}\hfill{\scriptsize (operation)}}\\
\textbf{\indent Returns:\ }
A 



 Returns the adjacency algebra $A$ of a homogeneous coherent configuration. This is an alias for
BoseMesnerAlgebra. }

 

\subsection{\textcolor{Chapter }{TerwilligerAlgebra (for IsHomogeneousCoherentConfiguration, IsInt)}}
\logpage{[ 3, 11, 3 ]}\nobreak
\hyperdef{L}{X820C1A5A7FF8F6B0}{}
{\noindent\textcolor{FuncColor}{$\triangleright$\enspace\texttt{TerwilligerAlgebra({\mdseries\slshape CC, p})\index{TerwilligerAlgebra@\texttt{TerwilligerAlgebra}!for IsHomogeneousCoherentConfiguration, IsInt}
\label{TerwilligerAlgebra:for IsHomogeneousCoherentConfiguration, IsInt}
}\hfill{\scriptsize (operation)}}\\
\textbf{\indent Returns:\ }
T 



 Returns the Terwilliger algebra $T$ of a homogeneous coherent configuration with respect to the point p. }

 

\subsection{\textcolor{Chapter }{TerwilligerAlgebra (for IsHomogeneousCoherentConfiguration)}}
\logpage{[ 3, 11, 4 ]}\nobreak
\hyperdef{L}{X8164E1657B7ED844}{}
{\noindent\textcolor{FuncColor}{$\triangleright$\enspace\texttt{TerwilligerAlgebra({\mdseries\slshape CC})\index{TerwilligerAlgebra@\texttt{TerwilligerAlgebra}!for IsHomogeneousCoherentConfiguration}
\label{TerwilligerAlgebra:for IsHomogeneousCoherentConfiguration}
}\hfill{\scriptsize (operation)}}\\
\textbf{\indent Returns:\ }
T 



 Returns the Terwilliger algebra $T$ of a homogeneous coherent configuration with respect to the first point. }

 }

 }

   
\chapter{\textcolor{Chapter }{Intersection Algebra objects}}\label{Chapter_Intersection_Algebra_objects}
\logpage{[ 4, 0, 0 ]}
\hyperdef{L}{X785CC0977ACB7807}{}
{
  
\section{\textcolor{Chapter }{Core functionality}}\label{Chapter_Intersection_Algebra_objects_Section_Core_functionality}
\logpage{[ 4, 1, 0 ]}
\hyperdef{L}{X851AE2B98382B550}{}
{
  

\subsection{\textcolor{Chapter }{IntersectionAlgebra (for IsList)}}
\logpage{[ 4, 1, 1 ]}\nobreak
\hyperdef{L}{X80FF69E585F8E179}{}
{\noindent\textcolor{FuncColor}{$\triangleright$\enspace\texttt{IntersectionAlgebra({\mdseries\slshape B})\index{IntersectionAlgebra@\texttt{IntersectionAlgebra}!for IsList}
\label{IntersectionAlgebra:for IsList}
}\hfill{\scriptsize (operation)}}\\
\textbf{\indent Returns:\ }
homogeneous coherent configuration 



 Takes a list of intersection matrices, B, and returns the Intersection
Algebra. The intersection matrix $(B_i)_{jk} = p_{ij}^k$ must be for valid intersection numbers $p_{ij}^k$ for some homogeneous coherent configuration. }

 

\subsection{\textcolor{Chapter }{IntersectionMatrices (for IsIntersectionAlgebraObject)}}
\logpage{[ 4, 1, 2 ]}\nobreak
\hyperdef{L}{X790BECF9797E075D}{}
{\noindent\textcolor{FuncColor}{$\triangleright$\enspace\texttt{IntersectionMatrices({\mdseries\slshape CC})\index{IntersectionMatrices@\texttt{IntersectionMatrices}!for IsIntersectionAlgebraObject}
\label{IntersectionMatrices:for IsIntersectionAlgebraObject}
}\hfill{\scriptsize (attribute)}}\\
\textbf{\indent Returns:\ }
L 



 Returns a list L of the intersection matrices of an intersecgion algebra
object, $CC$, where the $i$-th entry of $L$ is $B_{i-1}$ and $(B_{i})_{jk} = p_{ij}^k$. }

 

\subsection{\textcolor{Chapter }{IntersectionNumber (for IsIntersectionAlgebraObject, IsInt, IsInt, IsInt)}}
\logpage{[ 4, 1, 3 ]}\nobreak
\hyperdef{L}{X84FCFC117F92A772}{}
{\noindent\textcolor{FuncColor}{$\triangleright$\enspace\texttt{IntersectionNumber({\mdseries\slshape CC, i, j, k})\index{IntersectionNumber@\texttt{IntersectionNumber}!for IsIntersectionAlgebraObject, IsInt, IsInt, IsInt}
\label{IntersectionNumber:for IsIntersectionAlgebraObject, IsInt, IsInt, IsInt}
}\hfill{\scriptsize (operation)}}\\
\textbf{\indent Returns:\ }
$p_{ij}^k$ 



 Returns the intersection number $p_{ij}^k$ for a intersection algebra. }

 

\subsection{\textcolor{Chapter }{NumberOfClasses (for IsIntersectionAlgebraObject)}}
\logpage{[ 4, 1, 4 ]}\nobreak
\hyperdef{L}{X7C4F5A1485AA676B}{}
{\noindent\textcolor{FuncColor}{$\triangleright$\enspace\texttt{NumberOfClasses({\mdseries\slshape CC})\index{NumberOfClasses@\texttt{NumberOfClasses}!for IsIntersectionAlgebraObject}
\label{NumberOfClasses:for IsIntersectionAlgebraObject}
}\hfill{\scriptsize (attribute)}}\\
\textbf{\indent Returns:\ }
d 



 Returns $d$ for a $d$-class intersection algebra. }

 

\subsection{\textcolor{Chapter }{Valencies (for IsIntersectionAlgebraObject)}}
\logpage{[ 4, 1, 5 ]}\nobreak
\hyperdef{L}{X7B14A4317DEE2AC5}{}
{\noindent\textcolor{FuncColor}{$\triangleright$\enspace\texttt{Valencies({\mdseries\slshape CC})\index{Valencies@\texttt{Valencies}!for IsIntersectionAlgebraObject}
\label{Valencies:for IsIntersectionAlgebraObject}
}\hfill{\scriptsize (attribute)}}\\
\textbf{\indent Returns:\ }
L 



 Returns a list L of valencies of a coherent configuration CC. The $i$-th entry of $L$ is $k_{i-1}$. }

 

\subsection{\textcolor{Chapter }{Order (for IsIntersectionAlgebraObject)}}
\logpage{[ 4, 1, 6 ]}\nobreak
\hyperdef{L}{X7B8F4F6C78A44799}{}
{\noindent\textcolor{FuncColor}{$\triangleright$\enspace\texttt{Order({\mdseries\slshape CC})\index{Order@\texttt{Order}!for IsIntersectionAlgebraObject}
\label{Order:for IsIntersectionAlgebraObject}
}\hfill{\scriptsize (attribute)}}\\
\textbf{\indent Returns:\ }
n 



 Returns the order $n$ (number of vertices) of the intersection algebra. }

 

\subsection{\textcolor{Chapter }{SplittingField (for IsIntersectionAlgebraObject)}}
\logpage{[ 4, 1, 7 ]}\nobreak
\hyperdef{L}{X791D09F6835ED527}{}
{\noindent\textcolor{FuncColor}{$\triangleright$\enspace\texttt{SplittingField({\mdseries\slshape CC})\index{SplittingField@\texttt{SplittingField}!for IsIntersectionAlgebraObject}
\label{SplittingField:for IsIntersectionAlgebraObject}
}\hfill{\scriptsize (attribute)}}\\
\textbf{\indent Returns:\ }
F 



 Returns the splitting field of the CC }

 

\subsection{\textcolor{Chapter }{HasRationalSplittingField (for IsIntersectionAlgebraObject)}}
\logpage{[ 4, 1, 8 ]}\nobreak
\hyperdef{L}{X87E3C51878414487}{}
{\noindent\textcolor{FuncColor}{$\triangleright$\enspace\texttt{HasRationalSplittingField({\mdseries\slshape CC})\index{HasRationalSplittingField@\texttt{HasRationalSplittingField}!for IsIntersectionAlgebraObject}
\label{HasRationalSplittingField:for IsIntersectionAlgebraObject}
}\hfill{\scriptsize (property)}}\\
\textbf{\indent Returns:\ }
true or false 



 Returns true if the splitting field is the rationals, false otherwise. }

 

\subsection{\textcolor{Chapter }{KreinParameter (for IsIntersectionAlgebraObject, IsInt, IsInt, IsInt)}}
\logpage{[ 4, 1, 9 ]}\nobreak
\hyperdef{L}{X7DD2B3A679EA3698}{}
{\noindent\textcolor{FuncColor}{$\triangleright$\enspace\texttt{KreinParameter({\mdseries\slshape CC, i, j, k})\index{KreinParameter@\texttt{KreinParameter}!for IsIntersectionAlgebraObject, IsInt, IsInt, IsInt}
\label{KreinParameter:for IsIntersectionAlgebraObject, IsInt, IsInt, IsInt}
}\hfill{\scriptsize (operation)}}\\
\textbf{\indent Returns:\ }
$q_{i,j}^k$ 



 Compute the krein parameter $q_{i,j}^k$ of a commutative intersection algebra. }

 

\subsection{\textcolor{Chapter }{KreinParameters (for IsIntersectionAlgebraObject)}}
\logpage{[ 4, 1, 10 ]}\nobreak
\hyperdef{L}{X7F5118AC83CA120C}{}
{\noindent\textcolor{FuncColor}{$\triangleright$\enspace\texttt{KreinParameters({\mdseries\slshape CC})\index{KreinParameters@\texttt{KreinParameters}!for IsIntersectionAlgebraObject}
\label{KreinParameters:for IsIntersectionAlgebraObject}
}\hfill{\scriptsize (attribute)}}\\
\textbf{\indent Returns:\ }
L 



 Return a list $L$ of all Krein parameters of a commutative intersection algebra, where $L[i][j,k] = q_{i,j}^k$. }

 

\subsection{\textcolor{Chapter }{IsQBipartite (for IsIntersectionAlgebraObject)}}
\logpage{[ 4, 1, 11 ]}\nobreak
\hyperdef{L}{X7DD1A98E84BEECBF}{}
{\noindent\textcolor{FuncColor}{$\triangleright$\enspace\texttt{IsQBipartite({\mdseries\slshape CC})\index{IsQBipartite@\texttt{IsQBipartite}!for IsIntersectionAlgebraObject}
\label{IsQBipartite:for IsIntersectionAlgebraObject}
}\hfill{\scriptsize (property)}}\\
\textbf{\indent Returns:\ }
true or false 



 Returns if the intersection algebra CC is Q-bipartite. }

 

\subsection{\textcolor{Chapter }{IsPBipartite (for IsIntersectionAlgebraObject)}}
\logpage{[ 4, 1, 12 ]}\nobreak
\hyperdef{L}{X87C231C084A2CA1C}{}
{\noindent\textcolor{FuncColor}{$\triangleright$\enspace\texttt{IsPBipartite({\mdseries\slshape CC})\index{IsPBipartite@\texttt{IsPBipartite}!for IsIntersectionAlgebraObject}
\label{IsPBipartite:for IsIntersectionAlgebraObject}
}\hfill{\scriptsize (property)}}\\
\textbf{\indent Returns:\ }
true or false 



 Returns if the intersection algebra CC is bipartite. }

 

\subsection{\textcolor{Chapter }{IsQAntipodal (for IsIntersectionAlgebraObject)}}
\logpage{[ 4, 1, 13 ]}\nobreak
\hyperdef{L}{X7F9A176979843B60}{}
{\noindent\textcolor{FuncColor}{$\triangleright$\enspace\texttt{IsQAntipodal({\mdseries\slshape CC})\index{IsQAntipodal@\texttt{IsQAntipodal}!for IsIntersectionAlgebraObject}
\label{IsQAntipodal:for IsIntersectionAlgebraObject}
}\hfill{\scriptsize (property)}}\\
\textbf{\indent Returns:\ }
true or false 



 Returns if the intersection algebra CC is Q-antipodal. }

 

\subsection{\textcolor{Chapter }{IsPAntipodal (for IsIntersectionAlgebraObject)}}
\logpage{[ 4, 1, 14 ]}\nobreak
\hyperdef{L}{X85898F2779981DC3}{}
{\noindent\textcolor{FuncColor}{$\triangleright$\enspace\texttt{IsPAntipodal({\mdseries\slshape CC})\index{IsPAntipodal@\texttt{IsPAntipodal}!for IsIntersectionAlgebraObject}
\label{IsPAntipodal:for IsIntersectionAlgebraObject}
}\hfill{\scriptsize (property)}}\\
\textbf{\indent Returns:\ }
true or false 



 Returns if the intersection algebra CC is antipodal. }

 

\subsection{\textcolor{Chapter }{ReorderRelations (for IsIntersectionAlgebraObject, IsList)}}
\logpage{[ 4, 1, 15 ]}\nobreak
\hyperdef{L}{X7E1425D9780A8CA6}{}
{\noindent\textcolor{FuncColor}{$\triangleright$\enspace\texttt{ReorderRelations({\mdseries\slshape CC, L})\index{ReorderRelations@\texttt{ReorderRelations}!for IsIntersectionAlgebraObject, IsList}
\label{ReorderRelations:for IsIntersectionAlgebraObject, IsList}
}\hfill{\scriptsize (operation)}}\\
\textbf{\indent Returns:\ }
coherent configuration 



 Takes a intersection algebra CC and a list L, where L is a reordering of the
relations. Returns an intersection algebra where the $i$-th relation of the CC becomes the $j$-th relation in the intersection algebra, where $j = L_i$. Note that $L_i$ must be equal to $\{0, \ldots, d \}$ as a set, and additionally requires that $L_1 = 0$. }

 

\subsection{\textcolor{Chapter }{ReorderMinimalIdempotents (for IsIntersectionAlgebraObject, IsList)}}
\logpage{[ 4, 1, 16 ]}\nobreak
\hyperdef{L}{X86F72366870753E3}{}
{\noindent\textcolor{FuncColor}{$\triangleright$\enspace\texttt{ReorderMinimalIdempotents({\mdseries\slshape CC, L})\index{ReorderMinimalIdempotents@\texttt{ReorderMinimalIdempotents}!for IsIntersectionAlgebraObject, IsList}
\label{ReorderMinimalIdempotents:for IsIntersectionAlgebraObject, IsList}
}\hfill{\scriptsize (operation)}}\\
\textbf{\indent Returns:\ }
coherent configuration 



 Takes an intersection algebra CC and a list L, where L is a reordering of the
minimal idempotents. Returns an intersection algebra where the $i$-th idempotent of the CC becomes the $j$-th idempotent in the new intersection algebra, where $j = L_i$. Note that $L_i$ must be equal to $\{0, \ldots, d \}$ as a set, and additionally requires that $L_1 = 0$. }

 

\subsection{\textcolor{Chapter }{ViewRelationDistributionDiagram (for IsIntersectionAlgebraObject)}}
\logpage{[ 4, 1, 17 ]}\nobreak
\hyperdef{L}{X7EE0A7017A7A89EE}{}
{\noindent\textcolor{FuncColor}{$\triangleright$\enspace\texttt{ViewRelationDistributionDiagram({\mdseries\slshape CC})\index{ViewRelationDistributionDiagram@\texttt{ViewRelationDistributionDiagram}!for IsIntersectionAlgebraObject}
\label{ViewRelationDistributionDiagram:for IsIntersectionAlgebraObject}
}\hfill{\scriptsize (operation)}}\\
\textbf{\indent Returns:\ }
true (Displays relation distribution diagram) 



 Take a CC and display the relation-distribution diagram with respect to $R_1$. }

 

\subsection{\textcolor{Chapter }{IsCommutative (for IsIntersectionAlgebraObject)}}
\logpage{[ 4, 1, 18 ]}\nobreak
\hyperdef{L}{X7D34641A79D5A111}{}
{\noindent\textcolor{FuncColor}{$\triangleright$\enspace\texttt{IsCommutative({\mdseries\slshape CC})\index{IsCommutative@\texttt{IsCommutative}!for IsIntersectionAlgebraObject}
\label{IsCommutative:for IsIntersectionAlgebraObject}
}\hfill{\scriptsize (property)}}\\
\textbf{\indent Returns:\ }
true or false 



 Checks if the input is a commutative intersection algebra. }

 

\subsection{\textcolor{Chapter }{IsSymmetricIntersectionAlgebra (for IsIntersectionAlgebraObject)}}
\logpage{[ 4, 1, 19 ]}\nobreak
\hyperdef{L}{X787265D5870A8B9A}{}
{\noindent\textcolor{FuncColor}{$\triangleright$\enspace\texttt{IsSymmetricIntersectionAlgebra({\mdseries\slshape CC})\index{IsSymmetricIntersectionAlgebra@\texttt{IsSymmetricIntersectionAlgebra}!for IsIntersectionAlgebraObject}
\label{IsSymmetricIntersectionAlgebra:for IsIntersectionAlgebraObject}
}\hfill{\scriptsize (property)}}\\
\textbf{\indent Returns:\ }
true or false 



 Checks if the input is a symmetric intersection algebra. }

 

\subsection{\textcolor{Chapter }{NumberOfCharacters (for IsIntersectionAlgebraObject)}}
\logpage{[ 4, 1, 20 ]}\nobreak
\hyperdef{L}{X8034C62985C2232A}{}
{\noindent\textcolor{FuncColor}{$\triangleright$\enspace\texttt{NumberOfCharacters({\mdseries\slshape CC})\index{NumberOfCharacters@\texttt{NumberOfCharacters}!for IsIntersectionAlgebraObject}
\label{NumberOfCharacters:for IsIntersectionAlgebraObject}
}\hfill{\scriptsize (attribute)}}\\
\textbf{\indent Returns:\ }
n 



 Returns the number $n$ of characters of CC. }

 }

 
\section{\textcolor{Chapter }{Constructor methods}}\label{Chapter_Intersection_Algebra_objects_Section_Constructor_methods}
\logpage{[ 4, 2, 0 ]}
\hyperdef{L}{X820CD05F85142F0A}{}
{
  

\subsection{\textcolor{Chapter }{IntersectionAlgebraFromMatrixOfEigenvalues (for IsMatrix)}}
\logpage{[ 4, 2, 1 ]}\nobreak
\hyperdef{L}{X7E05148D80305715}{}
{\noindent\textcolor{FuncColor}{$\triangleright$\enspace\texttt{IntersectionAlgebraFromMatrixOfEigenvalues({\mdseries\slshape P})\index{IntersectionAlgebraFromMatrixOfEigenvalues@\texttt{Intersection}\-\texttt{Algebra}\-\texttt{From}\-\texttt{Matrix}\-\texttt{Of}\-\texttt{Eigenvalues}!for IsMatrix}
\label{IntersectionAlgebraFromMatrixOfEigenvalues:for IsMatrix}
}\hfill{\scriptsize (operation)}}\\
\textbf{\indent Returns:\ }
intersection algebra object 



 Returns the intersection algebra determined by a matrix of eigenvalues. }

 

\subsection{\textcolor{Chapter }{HammingSchemeIntersectionAlgebra (for IsPosInt, IsPosInt)}}
\logpage{[ 4, 2, 2 ]}\nobreak
\hyperdef{L}{X86DFDA3F86B4390D}{}
{\noindent\textcolor{FuncColor}{$\triangleright$\enspace\texttt{HammingSchemeIntersectionAlgebra({\mdseries\slshape n, q})\index{HammingSchemeIntersectionAlgebra@\texttt{HammingSchemeIntersectionAlgebra}!for IsPosInt, IsPosInt}
\label{HammingSchemeIntersectionAlgebra:for IsPosInt, IsPosInt}
}\hfill{\scriptsize (operation)}}\\
\textbf{\indent Returns:\ }
intersection algebra objectn 



 Returns the intersection algebra of the Hamming scheme, $H(n, q)$. }

 

\subsection{\textcolor{Chapter }{GrassmanSchemeIntersectionAlgebra (for IsPosInt, IsPosInt, IsPosInt)}}
\logpage{[ 4, 2, 3 ]}\nobreak
\hyperdef{L}{X7D09346C8602DBAA}{}
{\noindent\textcolor{FuncColor}{$\triangleright$\enspace\texttt{GrassmanSchemeIntersectionAlgebra({\mdseries\slshape n, k, q})\index{GrassmanSchemeIntersectionAlgebra@\texttt{GrassmanSchemeIntersectionAlgebra}!for IsPosInt, IsPosInt, IsPosInt}
\label{GrassmanSchemeIntersectionAlgebra:for IsPosInt, IsPosInt, IsPosInt}
}\hfill{\scriptsize (operation)}}\\
\textbf{\indent Returns:\ }
intersection algebra object 



 Returns the intersection algebra of the Grassmann scheme, $J_q(n, k)$. }

 

\subsection{\textcolor{Chapter }{CyclotomicSchemeIntersectionAlgebra (for IsPosInt, IsPosInt)}}
\logpage{[ 4, 2, 4 ]}\nobreak
\hyperdef{L}{X7E2E077F81242C77}{}
{\noindent\textcolor{FuncColor}{$\triangleright$\enspace\texttt{CyclotomicSchemeIntersectionAlgebra({\mdseries\slshape n, d})\index{CyclotomicSchemeIntersectionAlgebra@\texttt{CyclotomicSchemeIntersectionAlgebra}!for IsPosInt, IsPosInt}
\label{CyclotomicSchemeIntersectionAlgebra:for IsPosInt, IsPosInt}
}\hfill{\scriptsize (operation)}}\\
\textbf{\indent Returns:\ }
intersection algebra object 



 Returns the intersection algebra of the Cyclotomic scheme, $Cyc(n, d)$. }

 

\subsection{\textcolor{Chapter }{IntersectionAlgebraFromIntersectionArray (for IsList)}}
\logpage{[ 4, 2, 5 ]}\nobreak
\hyperdef{L}{X7813B9CB85C361CB}{}
{\noindent\textcolor{FuncColor}{$\triangleright$\enspace\texttt{IntersectionAlgebraFromIntersectionArray({\mdseries\slshape n, k, q})\index{IntersectionAlgebraFromIntersectionArray@\texttt{Intersection}\-\texttt{Algebra}\-\texttt{From}\-\texttt{Intersection}\-\texttt{Array}!for IsList}
\label{IntersectionAlgebraFromIntersectionArray:for IsList}
}\hfill{\scriptsize (operation)}}\\
\textbf{\indent Returns:\ }
intersection algebra object 



 Returns the intersection algebra of a DRG given by its intersection array. }

 

\subsection{\textcolor{Chapter }{IntersectionAlgebraFromClassicalParameters (for IsList)}}
\logpage{[ 4, 2, 6 ]}\nobreak
\hyperdef{L}{X7AA6FADB83444453}{}
{\noindent\textcolor{FuncColor}{$\triangleright$\enspace\texttt{IntersectionAlgebraFromClassicalParameters({\mdseries\slshape n, k, q})\index{IntersectionAlgebraFromClassicalParameters@\texttt{Intersection}\-\texttt{Algebra}\-\texttt{From}\-\texttt{Classical}\-\texttt{Parameters}!for IsList}
\label{IntersectionAlgebraFromClassicalParameters:for IsList}
}\hfill{\scriptsize (operation)}}\\
\textbf{\indent Returns:\ }
intersection algebra object 



 Returns the intersection algebra of a DRG given by its classical parameters. }

 

\subsection{\textcolor{Chapter }{IntersectionAlgebraFromStronglyRegularGraphParameters (for IsPosInt, IsPosInt, IsInt, IsInt)}}
\logpage{[ 4, 2, 7 ]}\nobreak
\hyperdef{L}{X82E28A4E83434396}{}
{\noindent\textcolor{FuncColor}{$\triangleright$\enspace\texttt{IntersectionAlgebraFromStronglyRegularGraphParameters({\mdseries\slshape n, k, q})\index{IntersectionAlgebraFromStronglyRegularGraphParameters@\texttt{Intersection}\-\texttt{Algebra}\-\texttt{From}\-\texttt{Strongly}\-\texttt{Regular}\-\texttt{Graph}\-\texttt{Parameters}!for IsPosInt, IsPosInt, IsInt, IsInt}
\label{IntersectionAlgebraFromStronglyRegularGraphParameters:for IsPosInt, IsPosInt, IsInt, IsInt}
}\hfill{\scriptsize (operation)}}\\
\textbf{\indent Returns:\ }
intersection algebra object 



 Returns the intersection algebra of a SRG given by its parameters $[n, k, \lambda, \mu]$. }

 

\subsection{\textcolor{Chapter }{IsFusionOfHomogeneousCoherentConfiguration (for IsIntersectionAlgebraObject, IsList)}}
\logpage{[ 4, 2, 8 ]}\nobreak
\hyperdef{L}{X7887CDF2811D5AAD}{}
{\noindent\textcolor{FuncColor}{$\triangleright$\enspace\texttt{IsFusionOfHomogeneousCoherentConfiguration({\mdseries\slshape CC, L})\index{IsFusionOfHomogeneousCoherentConfiguration@\texttt{IsFusion}\-\texttt{Of}\-\texttt{Homogeneous}\-\texttt{Coherent}\-\texttt{Configuration}!for IsIntersectionAlgebraObject, IsList}
\label{IsFusionOfHomogeneousCoherentConfiguration:for IsIntersectionAlgebraObject, IsList}
}\hfill{\scriptsize (operation)}}\\
\textbf{\indent Returns:\ }
true or false 



 Takes the intersection algebra object of a $d$-class homogeneous coherent configuration CC, and checks if the partion L of $\{0, \ldots, d\}$ corresponds to a valid fusion. }

 

\subsection{\textcolor{Chapter }{SchurianSchemeIntersectionAlgebra (for IsPermGroup)}}
\logpage{[ 4, 2, 9 ]}\nobreak
\hyperdef{L}{X7CF0D6B878269365}{}
{\noindent\textcolor{FuncColor}{$\triangleright$\enspace\texttt{SchurianSchemeIntersectionAlgebra({\mdseries\slshape G})\index{SchurianSchemeIntersectionAlgebra@\texttt{SchurianSchemeIntersectionAlgebra}!for IsPermGroup}
\label{SchurianSchemeIntersectionAlgebra:for IsPermGroup}
}\hfill{\scriptsize (operation)}}\\
\textbf{\indent Returns:\ }
intersection algebra 



 Returns the Schurian scheme defined by $G$, where $G$ is a transitive permutation group. A Schurian scheme is a special case of
CoherentConfigurationByOrbitals and is symmetric. }

 }

 
\section{\textcolor{Chapter }{Automorphisms and maps}}\label{Chapter_Intersection_Algebra_objects_Section_Automorphisms_and_maps}
\logpage{[ 4, 3, 0 ]}
\hyperdef{L}{X7FC3546D85D4FF29}{}
{
  

\subsection{\textcolor{Chapter }{MapFromIntersectionMatricesToCentralIdempotents (for IsIntersectionAlgebraObject)}}
\logpage{[ 4, 3, 1 ]}\nobreak
\hyperdef{L}{X7F58CA6B87D5D9B9}{}
{\noindent\textcolor{FuncColor}{$\triangleright$\enspace\texttt{MapFromIntersectionMatricesToCentralIdempotents({\mdseries\slshape I})\index{MapFromIntersectionMatricesToCentralIdempotents@\texttt{Map}\-\texttt{From}\-\texttt{Intersection}\-\texttt{Matrices}\-\texttt{To}\-\texttt{Central}\-\texttt{Idempotents}!for IsIntersectionAlgebraObject}
\label{MapFromIntersectionMatricesToCentralIdempotents:for IsIntersectionAlgebraObject}
}\hfill{\scriptsize (attribute)}}\\
\textbf{\indent Returns:\ }
M 



 Takes an intersection algebra object, I, and returns a matrix, M, which maps
the intersection matrices to the central idempotents of the intersection
algebra. The central idempotent $F_i = \sum_{i=1}^{d+1} M_{ij} B_j$. }

 

\subsection{\textcolor{Chapter }{MapFromIntersectionMatricesToCentralIdempotentsOverRationals (for IsIntersectionAlgebraObject)}}
\logpage{[ 4, 3, 2 ]}\nobreak
\hyperdef{L}{X8524F6998417238B}{}
{\noindent\textcolor{FuncColor}{$\triangleright$\enspace\texttt{MapFromIntersectionMatricesToCentralIdempotentsOverRationals({\mdseries\slshape I})\index{MapFromIntersectionMatricesToCentralIdempotentsOverRationals@\texttt{Map}\-\texttt{From}\-\texttt{Intersection}\-\texttt{Matrices}\-\texttt{To}\-\texttt{Central}\-\texttt{Idempotents}\-\texttt{Over}\-\texttt{Rationals}!for IsIntersectionAlgebraObject}
\label{MapFromIntersectionMatricesToCentralIdempotentsOverRationals:for IsIntersectionAlgebraObject}
}\hfill{\scriptsize (attribute)}}\\
\textbf{\indent Returns:\ }
M 



 Takes an intersection algebra object, I, and returns a matrix, M, which maps
the intersection matrices to the central idempotents of the intersection
algebra over the rationals. The central idempotent $F_i = \sum_{i=1}^{d+1} M_{ij} B_j$. }

 

\subsection{\textcolor{Chapter }{ImageOfIntersectionAlgebra (for IsIntersectionAlgebraObject, IsPerm)}}
\logpage{[ 4, 3, 3 ]}\nobreak
\hyperdef{L}{X7D7B6E69830C785E}{}
{\noindent\textcolor{FuncColor}{$\triangleright$\enspace\texttt{ImageOfIntersectionAlgebra({\mdseries\slshape \$A\$, \$\texttt{\symbol{92}}sigma\$})\index{ImageOfIntersectionAlgebra@\texttt{ImageOfIntersectionAlgebra}!for IsIntersectionAlgebraObject, IsPerm}
\label{ImageOfIntersectionAlgebra:for IsIntersectionAlgebraObject, IsPerm}
}\hfill{\scriptsize (operation)}}\\
\textbf{\indent Returns:\ }
true $A^\sigma$ 



 Take a $d$-class intersection algebra $A$ and return its image under the permutation $\sigma \in Sym([1 .. d])$. If $p_{ij}^k$ is an intersection number of $A$, then in the image the intersection number is $p_{i^\sigma j^\sigma}^{k^\sigma}$ }

 

\subsection{\textcolor{Chapter }{IsomorphismIntersectionAlgebras (for IsIntersectionAlgebraObject, IsIntersectionAlgebraObject)}}
\logpage{[ 4, 3, 4 ]}\nobreak
\hyperdef{L}{X80E1857E7CF06B41}{}
{\noindent\textcolor{FuncColor}{$\triangleright$\enspace\texttt{IsomorphismIntersectionAlgebras({\mdseries\slshape \$A\$, \$B\$})\index{IsomorphismIntersectionAlgebras@\texttt{IsomorphismIntersectionAlgebras}!for IsIntersectionAlgebraObject, IsIntersectionAlgebraObject}
\label{IsomorphismIntersectionAlgebras:for IsIntersectionAlgebraObject, IsIntersectionAlgebraObject}
}\hfill{\scriptsize (operation)}}\\
\textbf{\indent Returns:\ }
$\sigma$ 



 Take two $d$-class intersection algebras $A$ and $B$ and return $\sigma \in Sym([1 .. d])$ such that $A^\sigma = B$. }

 

\subsection{\textcolor{Chapter }{AreIsomorphicIntersectionAlgebras (for IsIntersectionAlgebraObject, IsIntersectionAlgebraObject)}}
\logpage{[ 4, 3, 5 ]}\nobreak
\hyperdef{L}{X7AEFAE9F7C8FD7FC}{}
{\noindent\textcolor{FuncColor}{$\triangleright$\enspace\texttt{AreIsomorphicIntersectionAlgebras({\mdseries\slshape \$A\$, \$B\$})\index{AreIsomorphicIntersectionAlgebras@\texttt{AreIsomorphicIntersectionAlgebras}!for IsIntersectionAlgebraObject, IsIntersectionAlgebraObject}
\label{AreIsomorphicIntersectionAlgebras:for IsIntersectionAlgebraObject, IsIntersectionAlgebraObject}
}\hfill{\scriptsize (operation)}}\\
\textbf{\indent Returns:\ }
$\sigma$ 



 Take two $d$-class intersection algebras $A$ and $B$ and return true if they are isomorphic. Return false otherwise. }

 

\subsection{\textcolor{Chapter }{CanonisingMap (for IsIntersectionAlgebraObject)}}
\logpage{[ 4, 3, 6 ]}\nobreak
\hyperdef{L}{X78D9D1E2827ADA58}{}
{\noindent\textcolor{FuncColor}{$\triangleright$\enspace\texttt{CanonisingMap({\mdseries\slshape A})\index{CanonisingMap@\texttt{CanonisingMap}!for IsIntersectionAlgebraObject}
\label{CanonisingMap:for IsIntersectionAlgebraObject}
}\hfill{\scriptsize (attribute)}}\\
\textbf{\indent Returns:\ }
perm 



 Returns two permutations which will produce the canonical form of the
intersection algebra A. The canonical form can be obtained by
ImageOfIntersectionAlgebra(A, perm) Any intersection algebra which is
isomorphic to A will the same canonical form. }

 

\subsection{\textcolor{Chapter }{CanonicalFormOfIntersectionAlgebra (for IsIntersectionAlgebraObject)}}
\logpage{[ 4, 3, 7 ]}\nobreak
\hyperdef{L}{X7FD9045C7A8D2F75}{}
{\noindent\textcolor{FuncColor}{$\triangleright$\enspace\texttt{CanonicalFormOfIntersectionAlgebra({\mdseries\slshape A})\index{CanonicalFormOfIntersectionAlgebra@\texttt{CanonicalFormOfIntersectionAlgebra}!for IsIntersectionAlgebraObject}
\label{CanonicalFormOfIntersectionAlgebra:for IsIntersectionAlgebraObject}
}\hfill{\scriptsize (operation)}}\\
\textbf{\indent Returns:\ }
B 



 Returns the canonical form, B, of the intersection algebra A. Any intersection
algebra which is isomorphic to A will have B as the canonical form. }

 

\subsection{\textcolor{Chapter }{AutomorphismGroup (for IsIntersectionAlgebraObject)}}
\logpage{[ 4, 3, 8 ]}\nobreak
\hyperdef{L}{X8521C6DD7DBDE766}{}
{\noindent\textcolor{FuncColor}{$\triangleright$\enspace\texttt{AutomorphismGroup({\mdseries\slshape A})\index{AutomorphismGroup@\texttt{AutomorphismGroup}!for IsIntersectionAlgebraObject}
\label{AutomorphismGroup:for IsIntersectionAlgebraObject}
}\hfill{\scriptsize (attribute)}}\\
\textbf{\indent Returns:\ }
G 



 Returns the automorphism group $G$ of the intersection algebra object A. $G$ is a permutation group acting on the relations, such that for all $g \in G$, $p_{i^g j^g}^{k^g} = p_{ij}^k$. }

 }

 
\section{\textcolor{Chapter }{Fusions}}\label{Chapter_Intersection_Algebra_objects_Section_Fusions}
\logpage{[ 4, 4, 0 ]}
\hyperdef{L}{X791CDD977B9FD97A}{}
{
  

\subsection{\textcolor{Chapter }{IsFusionOfIntersectionAlgebra (for IsIntersectionAlgebraObject, IsList)}}
\logpage{[ 4, 4, 1 ]}\nobreak
\hyperdef{L}{X7A2AE752848B5FEB}{}
{\noindent\textcolor{FuncColor}{$\triangleright$\enspace\texttt{IsFusionOfIntersectionAlgebra({\mdseries\slshape CC, L})\index{IsFusionOfIntersectionAlgebra@\texttt{IsFusionOfIntersectionAlgebra}!for IsIntersectionAlgebraObject, IsList}
\label{IsFusionOfIntersectionAlgebra:for IsIntersectionAlgebraObject, IsList}
}\hfill{\scriptsize (operation)}}\\
\textbf{\indent Returns:\ }
true or false 



 Takes a $d$-class homogeneous coherent configuration CC, and checks if the partion L of $\{0, \ldots, d\}$ corresponds to a valid fusion. }

 

\subsection{\textcolor{Chapter }{FusionOfIntersectionAlgebra (for IsIntersectionAlgebraObject, IsList)}}
\logpage{[ 4, 4, 2 ]}\nobreak
\hyperdef{L}{X7F43C9937A538D77}{}
{\noindent\textcolor{FuncColor}{$\triangleright$\enspace\texttt{FusionOfIntersectionAlgebra({\mdseries\slshape CC, L})\index{FusionOfIntersectionAlgebra@\texttt{FusionOfIntersectionAlgebra}!for IsIntersectionAlgebraObject, IsList}
\label{FusionOfIntersectionAlgebra:for IsIntersectionAlgebraObject, IsList}
}\hfill{\scriptsize (operation)}}\\
\textbf{\indent Returns:\ }
homogeneous coherent configuration 



 Takes a $d$-class homogeneous coherent configuration CC and returns a fusion scheme
corresponding to L, where L is a partion of $\{0, \ldots, d\}$. Note that the ordering of the cells of L is irrelevant. The method will sort
the fused relations according to the smallest value in each cell. }

 

\subsection{\textcolor{Chapter }{ConverseRelationPairs (for IsIntersectionAlgebraObject)}}
\logpage{[ 4, 4, 3 ]}\nobreak
\hyperdef{L}{X8435162780C90C07}{}
{\noindent\textcolor{FuncColor}{$\triangleright$\enspace\texttt{ConverseRelationPairs({\mdseries\slshape CC})\index{ConverseRelationPairs@\texttt{ConverseRelationPairs}!for IsIntersectionAlgebraObject}
\label{ConverseRelationPairs:for IsIntersectionAlgebraObject}
}\hfill{\scriptsize (attribute)}}\\
\textbf{\indent Returns:\ }
L 



 Returns a list L of either tuples or singletons, corresponding to relations
and their converses or symmetric relations. }

 

\subsection{\textcolor{Chapter }{ConverseRelation (for IsIntersectionAlgebraObject, IsInt)}}
\logpage{[ 4, 4, 4 ]}\nobreak
\hyperdef{L}{X7D8A365087D85C0D}{}
{\noindent\textcolor{FuncColor}{$\triangleright$\enspace\texttt{ConverseRelation({\mdseries\slshape CC, i})\index{ConverseRelation@\texttt{ConverseRelation}!for IsIntersectionAlgebraObject, IsInt}
\label{ConverseRelation:for IsIntersectionAlgebraObject, IsInt}
}\hfill{\scriptsize (operation)}}\\
\textbf{\indent Returns:\ }
j 



 Returns j such that $R_j = R_i^\top$, the converse relation of $i$ . }

 

\subsection{\textcolor{Chapter }{IsStratifiable (for IsIntersectionAlgebraObject)}}
\logpage{[ 4, 4, 5 ]}\nobreak
\hyperdef{L}{X8530E48E83318A72}{}
{\noindent\textcolor{FuncColor}{$\triangleright$\enspace\texttt{IsStratifiable({\mdseries\slshape CC})\index{IsStratifiable@\texttt{IsStratifiable}!for IsIntersectionAlgebraObject}
\label{IsStratifiable:for IsIntersectionAlgebraObject}
}\hfill{\scriptsize (attribute)}}\\
\textbf{\indent Returns:\ }
true or false 



 If the fusion of transposed relations produces a valid association scheme,
then CC is stratifiable. }

 

\subsection{\textcolor{Chapter }{SymmetrisationOfIntersectionAlgebra (for IsIntersectionAlgebraObject)}}
\logpage{[ 4, 4, 6 ]}\nobreak
\hyperdef{L}{X81CC46958133415A}{}
{\noindent\textcolor{FuncColor}{$\triangleright$\enspace\texttt{SymmetrisationOfIntersectionAlgebra({\mdseries\slshape CC})\index{SymmetrisationOfIntersectionAlgebra@\texttt{SymmetrisationOfIntersectionAlgebra}!for IsIntersectionAlgebraObject}
\label{SymmetrisationOfIntersectionAlgebra:for IsIntersectionAlgebraObject}
}\hfill{\scriptsize (operation)}}\\
\textbf{\indent Returns:\ }
Association scheme 



 Given a homogeneous coherent configuration, CC, the symmetrisation is computed
if possible, otherwise fail is returned. The symmetrisation of a homogeneous
coherent configuration takes any non-symmetric relations and fuses them
together. The result may or may not be a valid homogeneous coherent
configuration. If it is valid, then it is an association scheme (Symmetric
coherent configuration). If CC is commutative, then it can be symmetrised. }

 

\subsection{\textcolor{Chapter }{FeasibleNonTrivialFusionsOfIntersectionAlgebra (for IsIntersectionAlgebraObject)}}
\logpage{[ 4, 4, 7 ]}\nobreak
\hyperdef{L}{X8212A528859E9045}{}
{\noindent\textcolor{FuncColor}{$\triangleright$\enspace\texttt{FeasibleNonTrivialFusionsOfIntersectionAlgebra({\mdseries\slshape CC[, k[, flag]]})\index{FeasibleNonTrivialFusionsOfIntersectionAlgebra@\texttt{Feasible}\-\texttt{Non}\-\texttt{Trivial}\-\texttt{Fusions}\-\texttt{Of}\-\texttt{Intersection}\-\texttt{Algebra}!for IsIntersectionAlgebraObject}
\label{FeasibleNonTrivialFusionsOfIntersectionAlgebra:for IsIntersectionAlgebraObject}
}\hfill{\scriptsize (attribute)}}\\
\textbf{\indent Returns:\ }
list of feasibly fusionable relations 



 Returns a list where each entry is a collection of relations which may be
fused to form a feasible homogeneous coherent configuration Trivial means
either no relations are fused, or all non-identity relations are fused. If the
additional argument k is given, only fusions with k-classes are returned. If
flag is also given and is equal to true, then all fusions with at most
k-classes are returned. }

 

\subsection{\textcolor{Chapter }{AllNonTrivialFusionsOfIntersectionAlgebra (for IsIntersectionAlgebraObject)}}
\logpage{[ 4, 4, 8 ]}\nobreak
\hyperdef{L}{X7A42D1AF80E1C9A8}{}
{\noindent\textcolor{FuncColor}{$\triangleright$\enspace\texttt{AllNonTrivialFusionsOfIntersectionAlgebra({\mdseries\slshape CC})\index{AllNonTrivialFusionsOfIntersectionAlgebra@\texttt{All}\-\texttt{Non}\-\texttt{Trivial}\-\texttt{Fusions}\-\texttt{Of}\-\texttt{Intersection}\-\texttt{Algebra}!for IsIntersectionAlgebraObject}
\label{AllNonTrivialFusionsOfIntersectionAlgebra:for IsIntersectionAlgebraObject}
}\hfill{\scriptsize (operation)}}\\
\textbf{\indent Returns:\ }
List of all non-trivial fusions of CC 



 Returns a list of all homogeneous coherent configurations such that each
element of the list is a non-trivial fusion of CC. Trivial means either no
relations are fused, or all non-identity relations are fused. If the
additional argument k is given, only fusions with k-classes are returned. If
flag is also given and is equal to true, then all fusions with at most
k-classes are returned. }

 

\subsection{\textcolor{Chapter }{AllFusionsOfIntersectionAlgebra (for IsIntersectionAlgebraObject)}}
\logpage{[ 4, 4, 9 ]}\nobreak
\hyperdef{L}{X7CE39DC97A0F1393}{}
{\noindent\textcolor{FuncColor}{$\triangleright$\enspace\texttt{AllFusionsOfIntersectionAlgebra({\mdseries\slshape CC})\index{AllFusionsOfIntersectionAlgebra@\texttt{AllFusionsOfIntersectionAlgebra}!for IsIntersectionAlgebraObject}
\label{AllFusionsOfIntersectionAlgebra:for IsIntersectionAlgebraObject}
}\hfill{\scriptsize (operation)}}\\
\textbf{\indent Returns:\ }
List of all fusions of CC 



 Returns a list of all homogeneous coherent configurations such that each
element of the list is a fusion of CC. Includes trivial fusions, i.e the
original homogeneous coherent configuration, and the coherent configuration
resulting from the fusion of all non-identity relations }

 

\subsection{\textcolor{Chapter }{FeasibleNonTrivialSymmetricFusionsOfIntersectionAlgebra (for IsIntersectionAlgebraObject)}}
\logpage{[ 4, 4, 10 ]}\nobreak
\hyperdef{L}{X82A5979583C921AD}{}
{\noindent\textcolor{FuncColor}{$\triangleright$\enspace\texttt{FeasibleNonTrivialSymmetricFusionsOfIntersectionAlgebra({\mdseries\slshape CC})\index{FeasibleNonTrivialSymmetricFusionsOfIntersectionAlgebra@\texttt{Feasible}\-\texttt{Non}\-\texttt{Trivial}\-\texttt{Symmetric}\-\texttt{Fusions}\-\texttt{Of}\-\texttt{Intersection}\-\texttt{Algebra}!for IsIntersectionAlgebraObject}
\label{FeasibleNonTrivialSymmetricFusionsOfIntersectionAlgebra:for IsIntersectionAlgebraObject}
}\hfill{\scriptsize (attribute)}}\\
\textbf{\indent Returns:\ }
list of feasibly fusionable relations 



 Returns a list where each entry is a collection of relations which may be
fused to form a feasible association scheme (i.e. relations are symmetric)
Trivial means either no relations are fused, or all non-identity relations are
fused. If the additional argument k is given, only fusions with k-classes are
returned. If flag is also given and is equal to true, then all fusions with at
most k-classes are returned. }

 

\subsection{\textcolor{Chapter }{AllNonTrivialSymmetricFusionsOfIntersectionAlgebra (for IsIntersectionAlgebraObject)}}
\logpage{[ 4, 4, 11 ]}\nobreak
\hyperdef{L}{X874E117E7CF92FA3}{}
{\noindent\textcolor{FuncColor}{$\triangleright$\enspace\texttt{AllNonTrivialSymmetricFusionsOfIntersectionAlgebra({\mdseries\slshape CC})\index{AllNonTrivialSymmetricFusionsOfIntersectionAlgebra@\texttt{All}\-\texttt{Non}\-\texttt{Trivial}\-\texttt{Symmetric}\-\texttt{Fusions}\-\texttt{Of}\-\texttt{Intersection}\-\texttt{Algebra}!for IsIntersectionAlgebraObject}
\label{AllNonTrivialSymmetricFusionsOfIntersectionAlgebra:for IsIntersectionAlgebraObject}
}\hfill{\scriptsize (operation)}}\\
\textbf{\indent Returns:\ }
List of all non-trivial fusions of CC 



 Returns a list of all association schemes (i.e symmetric relations) such that
each element of the list is a non-trivial fusion of CC. Trivial means either
no relations are fused, or all non-identity relations are fused. If the
additional argument k is given, only fusions with k-classes are returned. If
flag is also given and is equal to true, then all fusions with at most
k-classes are returned. }

 

\subsection{\textcolor{Chapter }{AllSymmetricFusionsOfIntersectionAlgebra (for IsIntersectionAlgebraObject)}}
\logpage{[ 4, 4, 12 ]}\nobreak
\hyperdef{L}{X84881BF07A927379}{}
{\noindent\textcolor{FuncColor}{$\triangleright$\enspace\texttt{AllSymmetricFusionsOfIntersectionAlgebra({\mdseries\slshape CC})\index{AllSymmetricFusionsOfIntersectionAlgebra@\texttt{All}\-\texttt{Symmetric}\-\texttt{Fusions}\-\texttt{Of}\-\texttt{Intersection}\-\texttt{Algebra}!for IsIntersectionAlgebraObject}
\label{AllSymmetricFusionsOfIntersectionAlgebra:for IsIntersectionAlgebraObject}
}\hfill{\scriptsize (operation)}}\\
\textbf{\indent Returns:\ }
List of all fusions of CC 



 Returns a list of all association schemes (i.e symmetric relations) such that
each element of the list is a fusion of CC. Includes trivial fusions, i.e the
original homogeneous coherent configuration (if it is an association scheme),
and the coherent configuration resulting from the fusion of all non-identity
relations }

 }

 
\section{\textcolor{Chapter }{Intersection algebra}}\label{Chapter_Intersection_Algebra_objects_Section_Intersection_algebra}
\logpage{[ 4, 5, 0 ]}
\hyperdef{L}{X7EBD17797F391938}{}
{
  

\subsection{\textcolor{Chapter }{CentralIdempotentsOfIntersectionAlgebra (for IsIntersectionAlgebraObject)}}
\logpage{[ 4, 5, 1 ]}\nobreak
\hyperdef{L}{X85467C9F78AE489A}{}
{\noindent\textcolor{FuncColor}{$\triangleright$\enspace\texttt{CentralIdempotentsOfIntersectionAlgebra({\mdseries\slshape I})\index{CentralIdempotentsOfIntersectionAlgebra@\texttt{Central}\-\texttt{Idempotents}\-\texttt{Of}\-\texttt{Intersection}\-\texttt{Algebra}!for IsIntersectionAlgebraObject}
\label{CentralIdempotentsOfIntersectionAlgebra:for IsIntersectionAlgebraObject}
}\hfill{\scriptsize (attribute)}}\\
\textbf{\indent Returns:\ }
central idempotents 



 Returns the central idempotents of the intersection algebra. }

 

\subsection{\textcolor{Chapter }{CentralIdempotentsOfIntersectionAlgebraOverRationals (for IsIntersectionAlgebraObject)}}
\logpage{[ 4, 5, 2 ]}\nobreak
\hyperdef{L}{X81043F1E7EBF019E}{}
{\noindent\textcolor{FuncColor}{$\triangleright$\enspace\texttt{CentralIdempotentsOfIntersectionAlgebraOverRationals({\mdseries\slshape I})\index{CentralIdempotentsOfIntersectionAlgebraOverRationals@\texttt{Central}\-\texttt{Idempotents}\-\texttt{Of}\-\texttt{Intersection}\-\texttt{Algebra}\-\texttt{Over}\-\texttt{Rationals}!for IsIntersectionAlgebraObject}
\label{CentralIdempotentsOfIntersectionAlgebraOverRationals:for IsIntersectionAlgebraObject}
}\hfill{\scriptsize (attribute)}}\\
\textbf{\indent Returns:\ }
central idempotents 



 Returns the central idempotents of the intersection algebra over the
rationals. }

 

\subsection{\textcolor{Chapter }{MatrixOfDualEigenvalues (for IsIntersectionAlgebraObject)}}
\logpage{[ 4, 5, 3 ]}\nobreak
\hyperdef{L}{X8214E4828479DB33}{}
{\noindent\textcolor{FuncColor}{$\triangleright$\enspace\texttt{MatrixOfDualEigenvalues({\mdseries\slshape CC})\index{MatrixOfDualEigenvalues@\texttt{MatrixOfDualEigenvalues}!for IsIntersectionAlgebraObject}
\label{MatrixOfDualEigenvalues:for IsIntersectionAlgebraObject}
}\hfill{\scriptsize (attribute)}}\\
\textbf{\indent Returns:\ }
Q 



 Returns a the dual matrix of eigenvalues, $Q$, for a homogeneous coherent configuration CC. }

 

\subsection{\textcolor{Chapter }{MatrixOfEigenvalues (for IsIntersectionAlgebraObject)}}
\logpage{[ 4, 5, 4 ]}\nobreak
\hyperdef{L}{X79DF2613822843F2}{}
{\noindent\textcolor{FuncColor}{$\triangleright$\enspace\texttt{MatrixOfEigenvalues({\mdseries\slshape CC})\index{MatrixOfEigenvalues@\texttt{MatrixOfEigenvalues}!for IsIntersectionAlgebraObject}
\label{MatrixOfEigenvalues:for IsIntersectionAlgebraObject}
}\hfill{\scriptsize (attribute)}}\\
\textbf{\indent Returns:\ }
P 



 Returns a the matrix of eigenvalues (or character table), $P$, for the intersection algebra of a homogeneous coherent configuration CC. }

 

\subsection{\textcolor{Chapter }{FitMatrixOfEigenvalues (for IsIntersectionAlgebraObject, IsMatrix)}}
\logpage{[ 4, 5, 5 ]}\nobreak
\hyperdef{L}{X86A133187ED9D892}{}
{\noindent\textcolor{FuncColor}{$\triangleright$\enspace\texttt{FitMatrixOfEigenvalues({\mdseries\slshape A, P})\index{FitMatrixOfEigenvalues@\texttt{FitMatrixOfEigenvalues}!for IsIntersectionAlgebraObject, IsMatrix}
\label{FitMatrixOfEigenvalues:for IsIntersectionAlgebraObject, IsMatrix}
}\hfill{\scriptsize (operation)}}\\
\textbf{\indent Returns:\ }
P2 



 Checks if P is the matrix of eigenvalues of intersection algebra A, upto some
reordering of the columns. In such a case, P2, the reordered matrix is
returned. If not, returns fail. }

 }

 
\section{\textcolor{Chapter }{Metric schemes}}\label{Chapter_Intersection_Algebra_objects_Section_Metric_schemes}
\logpage{[ 4, 6, 0 ]}
\hyperdef{L}{X7EEAC69486AF989B}{}
{
  

\subsection{\textcolor{Chapter }{IsPPolynomial (for IsIntersectionAlgebraObject)}}
\logpage{[ 4, 6, 1 ]}\nobreak
\hyperdef{L}{X80974BA47FC6BF38}{}
{\noindent\textcolor{FuncColor}{$\triangleright$\enspace\texttt{IsPPolynomial({\mdseries\slshape CC})\index{IsPPolynomial@\texttt{IsPPolynomial}!for IsIntersectionAlgebraObject}
\label{IsPPolynomial:for IsIntersectionAlgebraObject}
}\hfill{\scriptsize (property)}}\\
\textbf{\indent Returns:\ }
true or false 



 Returns if the homogeneous coherent configuration CC is P-polynomial. }

 

\subsection{\textcolor{Chapter }{FirstPPolynomialOrdering (for IsIntersectionAlgebraObject)}}
\logpage{[ 4, 6, 2 ]}\nobreak
\hyperdef{L}{X7C84FB8B7CF27E91}{}
{\noindent\textcolor{FuncColor}{$\triangleright$\enspace\texttt{FirstPPolynomialOrdering({\mdseries\slshape CC})\index{FirstPPolynomialOrdering@\texttt{FirstPPolynomialOrdering}!for IsIntersectionAlgebraObject}
\label{FirstPPolynomialOrdering:for IsIntersectionAlgebraObject}
}\hfill{\scriptsize (attribute)}}\\
\textbf{\indent Returns:\ }
P-polynomial ordering or fail 



 Returns the first P-polynomial ordering admitted by the homogeneous coherent
configuration CC, and fail otherwise. }

 

\subsection{\textcolor{Chapter }{AdmitsPPolynomialOrdering (for IsIntersectionAlgebraObject)}}
\logpage{[ 4, 6, 3 ]}\nobreak
\hyperdef{L}{X85C393DD7D3AC351}{}
{\noindent\textcolor{FuncColor}{$\triangleright$\enspace\texttt{AdmitsPPolynomialOrdering({\mdseries\slshape CC})\index{AdmitsPPolynomialOrdering@\texttt{AdmitsPPolynomialOrdering}!for IsIntersectionAlgebraObject}
\label{AdmitsPPolynomialOrdering:for IsIntersectionAlgebraObject}
}\hfill{\scriptsize (property)}}\\
\textbf{\indent Returns:\ }
true or false 



 Returns if the homogeneous coherent configuration CC admits a P-polynomial
ordering. }

 

\subsection{\textcolor{Chapter }{IsMetric (for IsIntersectionAlgebraObject)}}
\logpage{[ 4, 6, 4 ]}\nobreak
\hyperdef{L}{X86FE341F8436B2F1}{}
{\noindent\textcolor{FuncColor}{$\triangleright$\enspace\texttt{IsMetric({\mdseries\slshape CC})\index{IsMetric@\texttt{IsMetric}!for IsIntersectionAlgebraObject}
\label{IsMetric:for IsIntersectionAlgebraObject}
}\hfill{\scriptsize (operation)}}\\
\textbf{\indent Returns:\ }
true or false 



 Alias for IsPPolynomial. }

 

\subsection{\textcolor{Chapter }{FirstMetricOrdering (for IsIntersectionAlgebraObject)}}
\logpage{[ 4, 6, 5 ]}\nobreak
\hyperdef{L}{X7EF0929E86278943}{}
{\noindent\textcolor{FuncColor}{$\triangleright$\enspace\texttt{FirstMetricOrdering({\mdseries\slshape CC})\index{FirstMetricOrdering@\texttt{FirstMetricOrdering}!for IsIntersectionAlgebraObject}
\label{FirstMetricOrdering:for IsIntersectionAlgebraObject}
}\hfill{\scriptsize (operation)}}\\
\textbf{\indent Returns:\ }
metric ordering or fail 



 Alias for FirstPPolynomialOrdering. }

 

\subsection{\textcolor{Chapter }{AdmitsMetricOrdering (for IsIntersectionAlgebraObject)}}
\logpage{[ 4, 6, 6 ]}\nobreak
\hyperdef{L}{X8234F6FC878D0573}{}
{\noindent\textcolor{FuncColor}{$\triangleright$\enspace\texttt{AdmitsMetricOrdering({\mdseries\slshape CC})\index{AdmitsMetricOrdering@\texttt{AdmitsMetricOrdering}!for IsIntersectionAlgebraObject}
\label{AdmitsMetricOrdering:for IsIntersectionAlgebraObject}
}\hfill{\scriptsize (operation)}}\\
\textbf{\indent Returns:\ }
true or false 



 Alias for AdmitsPPolynomialOrdering. }

 

\subsection{\textcolor{Chapter }{AllPPolynomialOrderings (for IsIntersectionAlgebraObject)}}
\logpage{[ 4, 6, 7 ]}\nobreak
\hyperdef{L}{X790D7CDB85570564}{}
{\noindent\textcolor{FuncColor}{$\triangleright$\enspace\texttt{AllPPolynomialOrderings({\mdseries\slshape CC})\index{AllPPolynomialOrderings@\texttt{AllPPolynomialOrderings}!for IsIntersectionAlgebraObject}
\label{AllPPolynomialOrderings:for IsIntersectionAlgebraObject}
}\hfill{\scriptsize (attribute)}}\\
\textbf{\indent Returns:\ }
L 



 Calculate the list $L$ of all P-polynomial orderings of a homogeneous coherent configuration. }

 

\subsection{\textcolor{Chapter }{AllMetricOrderings (for IsIntersectionAlgebraObject)}}
\logpage{[ 4, 6, 8 ]}\nobreak
\hyperdef{L}{X7B0635CC80BED05D}{}
{\noindent\textcolor{FuncColor}{$\triangleright$\enspace\texttt{AllMetricOrderings({\mdseries\slshape CC})\index{AllMetricOrderings@\texttt{AllMetricOrderings}!for IsIntersectionAlgebraObject}
\label{AllMetricOrderings:for IsIntersectionAlgebraObject}
}\hfill{\scriptsize (operation)}}\\
\textbf{\indent Returns:\ }
L 



 Alias for AllPPolynomialOrderings. }

 

\subsection{\textcolor{Chapter }{IntersectionArray (for IsIntersectionAlgebraObject)}}
\logpage{[ 4, 6, 9 ]}\nobreak
\hyperdef{L}{X7BE960BF807C8725}{}
{\noindent\textcolor{FuncColor}{$\triangleright$\enspace\texttt{IntersectionArray({\mdseries\slshape CC})\index{IntersectionArray@\texttt{IntersectionArray}!for IsIntersectionAlgebraObject}
\label{IntersectionArray:for IsIntersectionAlgebraObject}
}\hfill{\scriptsize (attribute)}}\\
\textbf{\indent Returns:\ }
List 



 Returns the intersection array if CC is P-polynomial. }

 

\subsection{\textcolor{Chapter }{ClassicalParameters (for IsIntersectionAlgebraObject)}}
\logpage{[ 4, 6, 10 ]}\nobreak
\hyperdef{L}{X7AFB848F7D33F5CA}{}
{\noindent\textcolor{FuncColor}{$\triangleright$\enspace\texttt{ClassicalParameters({\mdseries\slshape CC})\index{ClassicalParameters@\texttt{ClassicalParameters}!for IsIntersectionAlgebraObject}
\label{ClassicalParameters:for IsIntersectionAlgebraObject}
}\hfill{\scriptsize (attribute)}}\\
\textbf{\indent Returns:\ }
[d, b, $\alpha$, $\beta$] 



 Returns the classical parameters if the CC is metric with classical
parameters. }

 

\subsection{\textcolor{Chapter }{StronglyRegularGraphParameters (for IsIntersectionAlgebraObject)}}
\logpage{[ 4, 6, 11 ]}\nobreak
\hyperdef{L}{X83418B9878424CF5}{}
{\noindent\textcolor{FuncColor}{$\triangleright$\enspace\texttt{StronglyRegularGraphParameters({\mdseries\slshape CC})\index{StronglyRegularGraphParameters@\texttt{StronglyRegularGraphParameters}!for IsIntersectionAlgebraObject}
\label{StronglyRegularGraphParameters:for IsIntersectionAlgebraObject}
}\hfill{\scriptsize (attribute)}}\\
\textbf{\indent Returns:\ }
[d, b, $\alpha$, $\beta$] 



 Returns the parameters $\{n, k; \lambda, \mu \}$ if the CC is an association scheme with 2 classes. }

 }

 
\section{\textcolor{Chapter }{Cometric schemes}}\label{Chapter_Intersection_Algebra_objects_Section_Cometric_schemes}
\logpage{[ 4, 7, 0 ]}
\hyperdef{L}{X81F482747C1E7A48}{}
{
  

 

\subsection{\textcolor{Chapter }{AdmitsQPolynomialOrdering (for IsIntersectionAlgebraObject)}}
\logpage{[ 4, 7, 1 ]}\nobreak
\hyperdef{L}{X79607CE4875DB3EA}{}
{\noindent\textcolor{FuncColor}{$\triangleright$\enspace\texttt{AdmitsQPolynomialOrdering({\mdseries\slshape CC})\index{AdmitsQPolynomialOrdering@\texttt{AdmitsQPolynomialOrdering}!for IsIntersectionAlgebraObject}
\label{AdmitsQPolynomialOrdering:for IsIntersectionAlgebraObject}
}\hfill{\scriptsize (property)}}\\
\textbf{\indent Returns:\ }
true or false 



 Returns if the homogeneous coherent configuration CC admits a Q-polynomial
ordering. }

 

\subsection{\textcolor{Chapter }{AdmitsCometricOrdering (for IsIntersectionAlgebraObject)}}
\logpage{[ 4, 7, 2 ]}\nobreak
\hyperdef{L}{X7C1CE6DC85622EE0}{}
{\noindent\textcolor{FuncColor}{$\triangleright$\enspace\texttt{AdmitsCometricOrdering({\mdseries\slshape CC})\index{AdmitsCometricOrdering@\texttt{AdmitsCometricOrdering}!for IsIntersectionAlgebraObject}
\label{AdmitsCometricOrdering:for IsIntersectionAlgebraObject}
}\hfill{\scriptsize (operation)}}\\
\textbf{\indent Returns:\ }
true or false 



 Alias for AdmitsQPolynomialOrdering. }

 

\subsection{\textcolor{Chapter }{IsQPolynomial (for IsIntersectionAlgebraObject)}}
\logpage{[ 4, 7, 3 ]}\nobreak
\hyperdef{L}{X84E674CB7FDA999B}{}
{\noindent\textcolor{FuncColor}{$\triangleright$\enspace\texttt{IsQPolynomial({\mdseries\slshape CC})\index{IsQPolynomial@\texttt{IsQPolynomial}!for IsIntersectionAlgebraObject}
\label{IsQPolynomial:for IsIntersectionAlgebraObject}
}\hfill{\scriptsize (property)}}\\
\textbf{\indent Returns:\ }
true or false 



 Returns if the commutative coherent configuration CC is Q-polynomial. }

 

\subsection{\textcolor{Chapter }{IsCometric (for IsIntersectionAlgebraObject)}}
\logpage{[ 4, 7, 4 ]}\nobreak
\hyperdef{L}{X7A4AEBE3825C32B9}{}
{\noindent\textcolor{FuncColor}{$\triangleright$\enspace\texttt{IsCometric({\mdseries\slshape CC})\index{IsCometric@\texttt{IsCometric}!for IsIntersectionAlgebraObject}
\label{IsCometric:for IsIntersectionAlgebraObject}
}\hfill{\scriptsize (operation)}}\\
\textbf{\indent Returns:\ }
true or false 



 Alias for is Q-polynomial. }

 

\subsection{\textcolor{Chapter }{AllQPolynomialOrderings (for IsIntersectionAlgebraObject)}}
\logpage{[ 4, 7, 5 ]}\nobreak
\hyperdef{L}{X7F91B31B7EDCC312}{}
{\noindent\textcolor{FuncColor}{$\triangleright$\enspace\texttt{AllQPolynomialOrderings({\mdseries\slshape CC})\index{AllQPolynomialOrderings@\texttt{AllQPolynomialOrderings}!for IsIntersectionAlgebraObject}
\label{AllQPolynomialOrderings:for IsIntersectionAlgebraObject}
}\hfill{\scriptsize (attribute)}}\\
\textbf{\indent Returns:\ }
L 



 Calculate a list $L$ of all Q-polynomial orderings of a homogeneous coherent configuration. }

 

\subsection{\textcolor{Chapter }{AllCometricOrderings (for IsIntersectionAlgebraObject)}}
\logpage{[ 4, 7, 6 ]}\nobreak
\hyperdef{L}{X7B05438D7C4760D8}{}
{\noindent\textcolor{FuncColor}{$\triangleright$\enspace\texttt{AllCometricOrderings({\mdseries\slshape CC})\index{AllCometricOrderings@\texttt{AllCometricOrderings}!for IsIntersectionAlgebraObject}
\label{AllCometricOrderings:for IsIntersectionAlgebraObject}
}\hfill{\scriptsize (operation)}}\\
\textbf{\indent Returns:\ }
L 



 Alias for AllQPolynomialOrderings. }

 

\subsection{\textcolor{Chapter }{KreinArray (for IsIntersectionAlgebraObject)}}
\logpage{[ 4, 7, 7 ]}\nobreak
\hyperdef{L}{X83156F2F7C785E65}{}
{\noindent\textcolor{FuncColor}{$\triangleright$\enspace\texttt{KreinArray({\mdseries\slshape CC})\index{KreinArray@\texttt{KreinArray}!for IsIntersectionAlgebraObject}
\label{KreinArray:for IsIntersectionAlgebraObject}
}\hfill{\scriptsize (attribute)}}\\
\textbf{\indent Returns:\ }
List 



 Returns the Krein (or dual intersection) array if CC is Q-polynomial. }

 

\subsection{\textcolor{Chapter }{DualIntersectionArray (for IsIntersectionAlgebraObject)}}
\logpage{[ 4, 7, 8 ]}\nobreak
\hyperdef{L}{X81BB6D5B7BC9A829}{}
{\noindent\textcolor{FuncColor}{$\triangleright$\enspace\texttt{DualIntersectionArray({\mdseries\slshape CC})\index{DualIntersectionArray@\texttt{DualIntersectionArray}!for IsIntersectionAlgebraObject}
\label{DualIntersectionArray:for IsIntersectionAlgebraObject}
}\hfill{\scriptsize (operation)}}\\
\textbf{\indent Returns:\ }
List 



 Alias for KreinArray. }

 }

 }

   
\chapter{\textcolor{Chapter }{Examples}}\label{Chapter_Examples}
\logpage{[ 5, 0, 0 ]}
\hyperdef{L}{X7A489A5D79DA9E5C}{}
{
  

 
\section{\textcolor{Chapter }{Example 1 -- Constructing groups}}\label{Chapter_Examples_Section_Example_1_--_Constructing_groups}
\logpage{[ 5, 1, 0 ]}
\hyperdef{L}{X7A0C31A67CF49C08}{}
{
  In this example, we show how we can use coherent configurations to construct
an entriely different almost simple permutation group from another one. We
first show how $PSU(4,3)$ can be made out of its subgroup $PSL(3,4)$. 
\begin{Verbatim}[commandchars=!@|,fontsize=\small,frame=single,label=Example]
  !gapprompt@gap>| !gapinput@psl34 := PSL(3,4);;|
  !gapprompt@gap>| !gapinput@sylow3 := SylowSubgroup(psl34, 3);;|
  !gapprompt@gap>| !gapinput@normaliser := Normaliser(psl34, sylow3);;|
  !gapprompt@gap>| !gapinput@G := Image( FactorCosetAction(psl34, normaliser) );;|
\end{Verbatim}
 At this stage, we have constructed the unique permutation representation of
degree 280, for $PSL(3,4)$. 
\begin{Verbatim}[commandchars=!@|,fontsize=\small,frame=single,label=Example]
  !gapprompt@gap>| !gapinput@A := HomogeneousCoherentConfigurationByOrbitals(G);|
  7-class homogeneous coherent configuration of order 280
  !gapprompt@gap>| !gapinput@mat := RelationMatrix(A);;|
  !gapprompt@gap>| !gapinput@P := MatrixOfEigenvalues(A);;|
  !gapprompt@gap>| !gapinput@Print(P);|
  [ [ 1, 18, 18, 18, 72, 72, 72, 9 ], [ 1, 4, 4, 4, 16, -12, -12, -5 ],
   [ 1, -2, -2, 10, -8, 0, 0, 1 ], [ 1, -2, 10, -2, -8, 0, 0, 1 ], 
   [ 1, 10, -2, -2, -8, 0, 0, 1 ],
   [ 1, -2, -2, -2, 0, -8*E(7)^3-8*E(7)^5-8*E(7)^6, -8*E(7)-8*E(7)^2-8*E(7)^4, -3 ], 
   [ 1, -2, -2, -2, 0, -8*E(7)-8*E(7)^2-8*E(7)^4, -8*E(7)^3-8*E(7)^5-8*E(7)^6, -3 ],
   [ 1, -2, -2, -2, 7, -3, -3, 4 ] ]
\end{Verbatim}
 We now take a particular fusion of this coherent configuration to obtain a
2-class association scheme. 
\begin{Verbatim}[commandchars=!@|,fontsize=\small,frame=single,label=Example]
  !gapprompt@gap>| !gapinput@valency18 := Filtered([1..7], j -> Number(mat[1], i -> i = j) = 18);|
  [1,2,3]
  !gapprompt@gap>| !gapinput@fusions := List(Combinations(valency18,2), t -> |
  !gapprompt@>| !gapinput@		FusionOfHomogeneousCoherentConfiguration(A, [[0], t, |
  !gapprompt@>| !gapinput@			Difference([1..7],t)]) );;|
\end{Verbatim}
 Any of these three fusions will do: 
\begin{Verbatim}[commandchars=!@B,fontsize=\small,frame=single,label=Example]
  !gapprompt@gap>B !gapinput@autgroup := AutomorphismGroup( fusions[1] );;B
  !gapprompt@gap>B !gapinput@DisplayCompositionSeries( autgroup );B
  G (11 gens, size 26127360)
   | Z(2)
  S (4 gens, size 13063680)
   | Z(2)
  S (3 gens, size 6531840)
   | Z(2)
  S (2 gens, size 3265920)
   | 2A(3,3) = U(4,3) ~ 2D(3,3) = O-(6,3)
  1 (0 gens, size 1)
  !gapprompt@gap>B !gapinput@socle := Socle(autgroup);;B
  !gapprompt@gap>B !gapinput@StructureDescription(socle);B
  "PSU(4,3)"
\end{Verbatim}
 }

 
\section{\textcolor{Chapter }{Example 2 -- Dual polar spaces and their graphs}}\label{Chapter_Examples_Section_Example_2_--_Dual_polar_spaces_and_their_graphs}
\logpage{[ 5, 2, 0 ]}
\hyperdef{L}{X83AE03A179F223FD}{}
{
  For this example, we also use the package FinInG \cite{FinInG}. We will construct a metric association scheme coming from a dual polar
space. 
\begin{Verbatim}[commandchars=!@|,fontsize=\small,frame=single,label=Example]
  !gapprompt@gap>| !gapinput@LoadPackage("FinInG", false);;|
  !gapprompt@gap>| !gapinput@quadric := EllipticQuadric(7, 2);|
  Q-(7, 2)
  !gapprompt@gap>| !gapinput@points := AsList( Planes(quadric) );;|
  !gapprompt@gap>| !gapinput@mat := NullMat(Length(points), Length(points));;|
  !gapprompt@gap>| !gapinput@for i in [1..Length(points)] do|
  !gapprompt@>| !gapinput@	for j in [i+1..Length(points)] do|
  !gapprompt@>| !gapinput@		intersection := Meet( points{[i,j]} );|
  !gapprompt@gap>| !gapinput@		mat[i][j] := 2 - ProjectiveDimension( intersection );|
  !gapprompt@gap>| !gapinput@		mat[j][i] := mat[i][j];|
  !gapprompt@gap>| !gapinput@	od;|
  !gapprompt@gap>| !gapinput@od;|
\end{Verbatim}
 So far we have constructed the relation matrix arising from the dual polar
space. 
\begin{Verbatim}[commandchars=!@|,fontsize=\small,frame=single,label=Example]
  !gapprompt@gap>| !gapinput@a := HomogeneousCoherentConfiguration( mat );|
  3-class association scheme of order 765
  !gapprompt@gap>| !gapinput@P := MatrixOfEigenvalues(a);;|
  !gapprompt@gap>| !gapinput@Q := MatrixOfDualEigenvalues(a);;|
  !gapprompt@gap>| !gapinput@Display(P);|
  [ [    1,   28,  224,  512 ],
    [    1,   11,   20,  -32 ],
    [    1,   -7,   14,   -8 ],
    [    1,    1,  -10,    8 ] ]
  !gapprompt@gap>| !gapinput@Display(Q);|
  [ [       1,      84,     204,     476 ],
    [       1,      33,     -51,      17 ],
    [       1,    15/2,    51/4,   -85/4 ],
    [       1,   -21/4,  -51/16,  119/16 ] ]
  !gapprompt@gap>| !gapinput@IsPPolynomial(a);|
  true
  !gapprompt@gap>| !gapinput@IsQPolynomial(a);|
  true 
\end{Verbatim}
 A simpler way (perhaps) uses the automorphism group of the ambient polar
space: 
\begin{Verbatim}[commandchars=!@|,fontsize=\small,frame=single,label=Example]
  !gapprompt@gap>| !gapinput@cgroup := CollineationGroup(quadric);|
  PGO(-1,8,2)
  !gapprompt@gap>| !gapinput@G := Action(cgroup, points);|
  <permutation group with 3 generators>
  !gapprompt@gap>| !gapinput@a := SchurianAssociationScheme(G);|
  3-class homogeneous coherent configuration of order 765
  !gapprompt@gap>| !gapinput@IsPPolynomial(a);|
  true
\end{Verbatim}
 The automorphism group of the association scheme should be the same: 
\begin{Verbatim}[commandchars=!@|,fontsize=\small,frame=single,label=Example]
  !gapprompt@gap>| !gapinput@autgroup := AutomorphismGroup(a);;|
  !gapprompt@gap>| !gapinput@autgroup = G;|
  true
\end{Verbatim}
 Now (for the purist!) we see if there are interesting subsets. Take a
nondegenerate hyperplane section defining a parabolic quadric. 
\begin{Verbatim}[commandchars=!@|,fontsize=\small,frame=single,label=Example]
  !gapprompt@gap>| !gapinput@hyperplane := First(Hyperplanes(PG(7,2)), h -> |
  !gapprompt@>| !gapinput@		TypeOfSubspace(quadric, h) = "parabolic");|
  <a proj. 6-space in ProjectiveSpace(7, 2)>
  !gapprompt@gap>| !gapinput@insidehyp := Filtered(points, t -> t * hyperplane);;|
  !gapprompt@gap>| !gapinput@vector := CharacteristicVector(points, insidehyp);;|
  !gapprompt@gap>| !gapinput@dist := InnerDistribution(a, vector);|
  [ 1, 56, 64, 14 ]
  !gapprompt@gap>| !gapinput@macw := MacWilliamsTransform(a, dist);|
  [ 135, 630, 0, 0 ]
\end{Verbatim}
 Therefore, a hyperplane section gives rise to a design that is not a code, in
this association scheme. Now we produce the dual polar graph. 
\begin{Verbatim}[commandchars=!@|,fontsize=\small,frame=single,label=Example]
  !gapprompt@gap>| !gapinput@P := MatrixOfEigenvalues(a);;|
  !gapprompt@gap>| !gapinput@Display(P);|
  [ [    1,  224,  512,   28 ],
    [    1,   20,  -32,   11 ],
    [    1,   14,   -8,   -7 ],
    [    1,  -10,    8,    1 ] ]
  !gapprompt@gap>| !gapinput@position := Position(P[1], 28);|
  4
  !gapprompt@gap>| !gapinput@M := AdjacencyMatrices(a)[ position ];;|
  !gapprompt@gap>| !gapinput@graph := Graph(G, [1..Order(a)], OnPoints, {x,y} -> M[x][y] = 1);;|
  !gapprompt@gap>| !gapinput@IsDistanceRegular(graph);|
  true
  !gapprompt@gap>| !gapinput@GlobalParameters(graph);|
  [ [ 0, 0, 28 ], [ 1, 3, 24 ], [ 3, 9, 16 ], [ 7, 21, 0 ] ]
\end{Verbatim}
 }

 
\section{\textcolor{Chapter }{Example 3 -- Codes}}\label{Chapter_Examples_Section_Example_3_--_Codes}
\logpage{[ 5, 3, 0 ]}
\hyperdef{L}{X7F31FB6D7897A23B}{}
{
  For this example, we use the package Guava\cite{GUAVA} for its facility with block codes. We will see that the inner distribution
vector of a subset coincides with the weight enumerator of a code when the
association scheme is a Hamming scheme. 
\begin{Verbatim}[commandchars=!@|,fontsize=\small,frame=single,label=Example]
  !gapprompt@gap>| !gapinput@hammingscheme := HammingScheme(7,2);|
  7-class homogeneous coherent configuration of order 128
  !gapprompt@gap>| !gapinput@LoadPackage("Guava", false);;|
  !gapprompt@gap>| !gapinput@hammingcode := HammingCode(3, GF(2));|
  a linear [7,4,3]1 Hamming (3,2) code over GF(2)
\end{Verbatim}
 We now use an operation from Guava: 
\begin{Verbatim}[commandchars=!@|,fontsize=\small,frame=single,label=Example]
  !gapprompt@gap>| !gapinput@InnerDistribution(hammingcode);|
  [ 1, 0, 0, 7, 7, 0, 0, 1 ]
\end{Verbatim}
 From the association scheme perspective ... 
\begin{Verbatim}[commandchars=!@|,fontsize=\small,frame=single,label=Example]
  !gapprompt@gap>| !gapinput@codewords := List( hammingcode, VectorCodeword );;|
  !gapprompt@gap>| !gapinput@vector := CharacteristicVector( AsList(GF(2)^7), codewords );;|
  !gapprompt@gap>| !gapinput@Collected(vector);|
  [ [ 0, 112 ], [ 1, 16 ] ]
  !gapprompt@gap>| !gapinput@inndist := InnerDistribution( hammingscheme, vector);|
  [ 1, 0, 0, 7, 7, 0, 0, 1 ]
\end{Verbatim}
 The MacWilliams transform coincides with the distribution vector of the dual
code: 
\begin{Verbatim}[commandchars=!@|,fontsize=\small,frame=single,label=Example]
  !gapprompt@gap>| !gapinput@1/16 * MacWilliamsTransform( hammingscheme, inndist);|
  [ 1, 0, 0, 0, 7, 0, 0, 0 ]
  !gapprompt@gap>| !gapinput@dualcode := DualCode( hammingcode );|
  a linear [7,3,4]2..3 dual code
  !gapprompt@gap>| !gapinput@InnerDistribution( dualcode );|
  [ 1, 0, 0, 0, 7, 0, 0, 0 ]
\end{Verbatim}
 }

 
\section{\textcolor{Chapter }{Example 4 -- Using the library}}\label{Chapter_Examples_Section_Example_4_--_Using_the_library}
\logpage{[ 5, 4, 0 ]}
\hyperdef{L}{X869C42647960C7AC}{}
{
  In this package, we also have a library of all small homogeneous coherent
configurations, of order at most 38 (except 31, ,35, 36, 37), corresponding to \cite{Hanaki}. 
\begin{Verbatim}[commandchars=!@|,fontsize=\small,frame=single,label=Example]
  !gapprompt@gap>| !gapinput@for i in [5..20] do|
  !gapprompt@>| !gapinput@	Print(i,"    ",NumberOfHomogeneousCoherentConfigurations(i),"\n");|
  !gapprompt@gap>| !gapinput@od;|
  5    2
  6    6
  7    3
  8    16
  9    10
  10    11
  11    3
  12    54
  13    5
  14    14
  15    24
  16    208
  17    4
  18    90
  19    6
  20    90
  !gapprompt@gap>| !gapinput@order7 := List([1..3], i -> HomogeneousCoherentConfiguration(7, i));|
   1-class homogeneous coherent configuration of order 7, 
   2-class homogeneous coherent configuration of order 7, 
   3-class homogeneous coherent configuration of order 7 ]
\end{Verbatim}
 The first of these is trivial, so we look at the other two. The first arises
from the Paley graph of order 7. 
\begin{Verbatim}[commandchars=!@|,fontsize=\small,frame=single,label=Example]
  !gapprompt@gap>| !gapinput@a1 := order7[2];|
  2-class homogeneous coherent configuration of order 7
  !gapprompt@gap>| !gapinput@autgroup := AutomorphismGroup(a1);|
  Group([ (2,3,4)(5,7,6), (1,2,3,5,4,6,7) ])
  !gapprompt@gap>| !gapinput@StructureDescription(autgroup);|
  "C7 : C3"
\end{Verbatim}
 The last one is a 3-class association scheme: 
\begin{Verbatim}[commandchars=!@|,fontsize=\small,frame=single,label=Example]
  !gapprompt@gap>| !gapinput@a2 := order7[3];|
  3-class homogeneous coherent configuration of order 7
  !gapprompt@gap>| !gapinput@IsAssociationScheme(a2);|
  true
  !gapprompt@gap>| !gapinput@IsPPolynomial( a2 );|
  true
  !gapprompt@gap>| !gapinput@IsPrimitive(a2);|
  rue
  !gapprompt@gap>| !gapinput@Valencies(a2);|
  [ 1, 2, 2, 2 ]
  !gapprompt@gap>| !gapinput@autgroup := AutomorphismGroup(a2);|
  Group([ (2,3)(4,5)(6,7), (1,2)(3,4)(5,6) ])
  !gapprompt@gap>| !gapinput@StructureDescription(autgroup);|
  "D14"
  !gapprompt@gap>| !gapinput@P := MatrixOfEigenvalues(a2);;|
  !gapprompt@gap>| !gapinput@Display(P);|
  [ [              1,              2,              2,              2 ],
    [              1,  E(7)^3+E(7)^4,    E(7)+E(7)^6,  E(7)^2+E(7)^5 ],
    [              1,  E(7)^2+E(7)^5,  E(7)^3+E(7)^4,    E(7)+E(7)^6 ],
    [              1,    E(7)+E(7)^6,  E(7)^2+E(7)^5,  E(7)^3+E(7)^4 ] ]
  !gapprompt@gap>| !gapinput@AllPPolynomialOrderings(a2);|
  [ [ 0, 1, 2, 3 ], [ 0, 2, 3, 1 ], [ 0, 3, 1, 2 ] ]
  !gapprompt@gap>| !gapinput@AdmitsQPolynomialOrdering(a2);|
  true
  !gapprompt@gap>| !gapinput@AllQPolynomialOrderings(a2);|
  [ [ 0, 1, 3, 2 ], [ 0, 2, 1, 3 ], [ 0, 3, 2, 1 ] ]
\end{Verbatim}
 }

 
\section{\textcolor{Chapter }{Example 5 -- Constructing HS (advanced example)}}\label{Chapter_Examples_Section_Example_5_--_Constructing_HS_advanced_example}
\logpage{[ 5, 5, 0 ]}
\hyperdef{L}{X7F8954C27C3CCAAC}{}
{
  We redo an example that appears in Section 3.6 of Peter Cameron's "Permutation
Groups" book \cite{cameron} and construct the Higman-Sims group. \\
\\
 First we construct the Hoffman-Singleton graph from the alternating group of
degree 7. 
\begin{Verbatim}[commandchars=!@|,fontsize=\small,frame=single,label=Example]
  !gapprompt@gap>| !gapinput@A7 := AlternatingGroup(7);;|
  !gapprompt@gap>| !gapinput@Pi := [ [ 1, 2, 4 ], [ 1, 3, 7 ], [ 1, 5, 6 ], |
  !gapprompt@>| !gapinput@ [ 2, 3, 5 ], [ 2, 6, 7 ], [ 3, 4, 6 ], [ 4, 5, 7 ] ];;|
  !gapprompt@gap>| !gapinput@OnSetsRecursive := function(x,g) |
  !gapprompt@>| !gapinput@	if not IsSet(x) then|
  !gapprompt@>| !gapinput@		return x^g; |
  !gapprompt@gap>| !gapinput@	else |
  !gapprompt@>| !gapinput@		return Set(x,y->OnSetsRecursive(y,g));|
  !gapprompt@gap>| !gapinput@	fi;|
  !gapprompt@gap>| !gapinput@end;;|
  !gapprompt@gap>| !gapinput@triples := Combinations([1..7], 3);;|
  !gapprompt@gap>| !gapinput@allFanos := Orbit(A7, Pi, OnSetsSets);;|
  !gapprompt@gap>| !gapinput@fifty := Concatenation(triples, allFanos);;|
  !gapprompt@gap>| !gapinput@A7action := Action(A7, fifty, OnSetsRecursive);|
  <permutation group with 2 generators>
  !gapprompt@gap>| !gapinput@orbitals := Orbits(A7action, Combinations([1..50],2), OnSets);;|
  !gapprompt@gap>| !gapinput@List(orbitals, Size);|
  [ 210, 315, 70, 420, 105, 105 ]
\end{Verbatim}
 We will now make a homogeneous coherent configuration from scratch, from these
orbitals. 
\begin{Verbatim}[commandchars=!@|,fontsize=\small,frame=single,label=Example]
  !gapprompt@gap>| !gapinput@mat := NullMat(50,50);;|
  !gapprompt@gap>| !gapinput@for i in [1..50] do|
  !gapprompt@>| !gapinput@	for j in [i+1..50] do|
  !gapprompt@>| !gapinput@		pos := First([1..Length(orbitals)], k -> [i,j] in orbitals[k]);|
  !gapprompt@gap>| !gapinput@		mat[i][j] := pos;|
  !gapprompt@gap>| !gapinput@		mat[j][i] := pos;|
  !gapprompt@gap>| !gapinput@	od;|
  !gapprompt@gap>| !gapinput@od;|
\end{Verbatim}
 This is not a CC yet. We will fuse the relations of valency 3 and 4: 
\begin{Verbatim}[commandchars=!@|,fontsize=\small,frame=single,label=Example]
  !gapprompt@gap>| !gapinput@l := Collected(mat[1]);|
  [ [ 0, 1 ], [ 1, 12 ], [ 2, 18 ], [ 3, 4 ], [ 4, 12 ], [ 5, 3 ] ]
  !gapprompt@gap>| !gapinput@to_fuse := Filtered([1..Length(l)], t -> l[t][2] in [3,4])-1;|
  [ 3, 5 ]
  !gapprompt@gap>| !gapinput@to_fuse2 := Difference([1..6],to_fuse);|
  [ 1, 2, 4, 6 ]
  !gapprompt@gap>| !gapinput@poly := InterpolatedPolynomial(Rationals, Concatenation([0], to_fuse, |
  !gapprompt@>| !gapinput@	to_fuse2), [0,1,1,2,2,2,2] );;|
  !gapprompt@gap>| !gapinput@newmat := List(mat, row -> List(row, x -> Value(poly,x)));;|
  !gapprompt@gap>| !gapinput@Collected(newmat[1]);|
  [ [ 0, 1 ], [ 1, 7 ], [ 2, 42 ] ]
\end{Verbatim}
 This now leads us directly to the Hoffman-Singleton graph: 
\begin{Verbatim}[commandchars=!@|,fontsize=\small,frame=single,label=Example]
  !gapprompt@gap>| !gapinput@cc := HomogeneousCoherentConfiguration( newmat ); |
  2-class association scheme of order 50
  !gapprompt@gap>| !gapinput@autHoffSing := AutomorphismGroup( cc );|
  <permutation group with 7 generators>
  !gapprompt@gap>| !gapinput@StructureDescription( autHoffSing );|
  "PSU(3,5) : C2"
\end{Verbatim}
 We will now construct the Mesner-Higman-Sims graph 
\begin{Verbatim}[commandchars=!@|,fontsize=\small,frame=single,label=Example]
  !gapprompt@gap>| !gapinput@vals := Valencies(cc);|
  [ 1, 7, 42 ]
  !gapprompt@gap>| !gapinput@adjmat := AdjacencyMatrices(cc)[ Position(vals, 42) ];;|
  !gapprompt@gap>| !gapinput@graph := Graph(autHoffSing, [1..50], OnPoints, {x,y} -> adjmat[x][y]=1);;|
  !gapprompt@gap>| !gapinput@one_coclique := CompleteSubgraphsOfGivenSize(graph, 15)[1];;|
  !gapprompt@gap>| !gapinput@all_cocliques := Orbit(autHoffSing, |
  !gapprompt@>| !gapinput@	Set(VertexNames(graph){one_coclique}), OnSets);;|
  !gapprompt@gap>| !gapinput@Size(all_cocliques);|
  100
  !gapprompt@gap>| !gapinput@G := Action(autHoffSing, all_cocliques, OnSets);;|
  !gapprompt@gap>| !gapinput@a := SchurianAssociationScheme(G);|
  4-class homogeneous coherent configuration of order 100
\end{Verbatim}
 Now fuse the relations with valencies 7 and 15 (and the complement) 
\begin{Verbatim}[commandchars=!@|,fontsize=\small,frame=single,label=Example]
  !gapprompt@gap>| !gapinput@vals := Valencies(a);|
  [ 1, 35, 42, 15, 7 ]
  !gapprompt@gap>| !gapinput@to_fuse := Filtered([1..Length(vals)], t -> vals[t] in [7,15])-1;;|
  !gapprompt@gap>| !gapinput@to_fuse2 := Difference([1..4], to_fuse);;|
  !gapprompt@gap>| !gapinput@fusion := FusionOfHomogeneousCoherentConfiguration(a, [[0], to_fuse,|
  !gapprompt@>| !gapinput@	to_fuse2]);|
  2-class association scheme of order 100
  !gapprompt@gap>| !gapinput@autgroup2 := AutomorphismGroup(fusion);|
  <permutation group with 10 generators>
  !gapprompt@gap>| !gapinput@StructureDescription(autgroup2);|
  "HS : C2"
\end{Verbatim}
 }

 }

   
\chapter{\textcolor{Chapter }{Appendix}}\label{Chapter_Appendix}
\logpage{[ 6, 0, 0 ]}
\hyperdef{L}{X86E649828230497D}{}
{
  

 
\section{\textcolor{Chapter }{AssociationSchemes Links}}\label{Chapter_Appendix_Section_AssociationSchemes_Links}
\logpage{[ 6, 1, 0 ]}
\hyperdef{L}{X8780F3D87DD387DA}{}
{
  

 
\begin{itemize}
\item  Homepage: http://www.jesselansdown.com/AssociationSchemes 
\item  Issue tracker: https://github.com/jesselansdown/AssociationSchemes/issues 
\item  DOI: 10.5281/zenodo.2634955 
\end{itemize}
 

 }

 
\section{\textcolor{Chapter }{\textsf{GAP} Links}}\label{Chapter_Appendix_Section_GAP_Links}
\logpage{[ 6, 2, 0 ]}
\hyperdef{L}{X8122776087DEC60B}{}
{
  

 
\begin{itemize}
\item  Homepage: http://gap-system.org 
\item  NautyTracesInterface: https://github.com/sebasguts/NautyTracesInterface 
\end{itemize}
 }

 }

 \def\bibname{References\logpage{[ "Bib", 0, 0 ]}
\hyperdef{L}{X7A6F98FD85F02BFE}{}
}

\bibliographystyle{alpha}
\bibliography{AssociationSchemes.bib}

\addcontentsline{toc}{chapter}{References}

\def\indexname{Index\logpage{[ "Ind", 0, 0 ]}
\hyperdef{L}{X83A0356F839C696F}{}
}

\cleardoublepage
\phantomsection
\addcontentsline{toc}{chapter}{Index}


\printindex

\immediate\write\pagenrlog{["Ind", 0, 0], \arabic{page},}
\newpage
\immediate\write\pagenrlog{["End"], \arabic{page}];}
\immediate\closeout\pagenrlog
\end{document}
